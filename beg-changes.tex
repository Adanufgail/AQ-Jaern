\chapter{Changes Made in this Version}
The following are areas that I (the editor of this revised edition, Michael Lubert) felt were either no longer in keeping with the world that I played, were wholly missing, or were conflicting within the text:
\begin{enumerate}[leftmargin=12pt]
\item Slavery: In the original version text, slavery is both depicted as a form of punishment-based indentured servitude, and as a chattel version of slavery in which slaves remain in servitude for life. Additionally, the original text includes the conflicting statements that children cannot be slaves, and that they can be born into slavery or saddled with someone else's slave-debt. As slave labor was often relegated to the background of scenes when I played, and slavery was only utilized as a punishment, I will be removing much of the supporting text for it and updating it to be more akin indentured labor, with the punishment for crimes not being transferable to kin, save for the withholding of inheritance to cover debts. 

References to "slave" will be replaced with "\indy{prisoner}," which fits with their circumstance as someone who is \textit{temporarily} obligated to perform work as a condition of criminal punishment. 
\item \indy[weapon]{Weapons}: Many of the weapons seem to hold nonsensical values with regard to their (sparse) descriptions, often conflicting with historical (and other game's versions) of the weapons. I will be making efforts to update the weapon table to make sense.
\item \indy{Souls}: Much of the writings of nomadic, divine, and elemental magic systems involve souls and those who have them. There are entire branches of necromancy devoted to it. However, there are odd gaps when it comes to elves. Additionally, there is some confusion on the difference between the mind/soul, specifically in regards to memory and personality (important distinctions for both undead and necromancy). As a result, I have made the following clarifications/changes:
\begin{enumerate}
	\item \indy{Spells} and effects which \indy[soul!remove]{remove} or \indy[soul!destroy]{destroy} a soul do not kill the target's body.
	\item \indy{Memory} and \indy{personality} of a creature with a soul are stored in the soul, and are stored in the mind for a creature without. This means that a person or creature who loses their soul loses their memories and personality, but are still capable of creating new memories and may develop a similar or radically different personality (similar to amnesia). Additionally, a person or creature who is able to move their soul to another body (which is without a soul) will retain all of their memories, but none of the being whose body they now inhabit.
	\item Where the phrase \indy[soul!soulless husk]{"husk"} is used, it can be interpreted as the following effect:

	\listing{Husk}\\
	The creature is in a nearly lifeless stupor, unaware of the world around it and incapable of any actions beyond the basic processes needed to continue life (ie breathing, maintaining heartbeat). This condition will continue for \tcdieroll{10d20} hours, determined by the GM. After leaving this condition, the creature will retain none of their memories or personality of their previous life. If the affected creature was an \indy{adventurer}, it is up to the GM to determine whether the player should continue playing their new life, or if they become a GM-acted character. If a creature effected by this effect has a soul implanted (whether their own or a new one) during the duration, the effect clears and their soul takes over functions.
\end{enumerate}
\item Karfelon: Much of the 2010 version of the manual references Karfelon, including characters, history, locations, and lore. Karfelon was a massive city in a valley surrounded by a man-made seawall extending from the bottom of Lojem. Karfelon was destroyed following the destruction of the seawall in the late 1990s or early 2000s (the AQ website adventure summaries from 2002 already reference Rougtero, the city founded in the wake of the destruction where surviving refugees rebuilt). As it had been destroyed for nearly a decade (Earth time) by the time I began playing, I never had any attachment to it beyond as a source for lore and a potential place to send adventurers to dive down to for a mysterious treasure. I will be updating the relevant chapters and characters to match ones from Rougtero (perhaps copying some of the more interesting ones from Karfelon).
\item Pimping: Similar to slave handling, this skill was never utilized, and additionally is just pretty gross. Additionally, the skill basically encompasses Teaching, Business Management (a newly created skill), and Courtesan skills. 
\item Pronoun Gender: Gender neutral pronouns are used where applicable, updating from the previous version's masculine pronoun usage.
\item Adamantine vs Adamantite: Both versions of this material appear in the text. It appears that "adamantite" was utilized in the 1st edition of D\&D, which \aq is based on, and then changed in later versions to "adamantine," which is a Greek concept for diamond (and thus an obvious inspiration the hard and durable nature of the material). I have opted for "adamantine" and will be updating any instances of "adamantite." Additonally, there is almost no actual information on the availability/utility of the material or its properties beyond the material cost modifier and that it's apparently durable but rare and hard to work with. I first gained familiarity with the material from the video game Dwarf Fortress, where it's known for being exceedingly light, and is crafted into lightweight chainmail or sharp blades, but is useless for hammers because of how light it is. I will be including similar material properties for various woods, leather and scales, metals, stones, magical materials, mythical materials, etc. and how they could affect weapon and armor weight, durability, performance, or resistance to damage types.
\item \indy{Scrogg}: Scrogg was created as a joke, but eventually given some level of legitimacy. By my time, they were referred to not as the "God of Sensual Pleasures" but as the "God of Earthly Pleasures," which had been expanded to include music and food. I will be reworking most of the spells, history, and structure of this priesthood to reflect that, as well as removing many of the frankly disturbing aspects of Scrogg. Having 4 different spells that make people want to have sex with you is excessive, and it'd be more interesting, for instance, to have a spell that makes someone think that eating stew that's about to go off tastes like the best meal they've ever had, or to make a tone-deaf drunk's wailing sound like Frank Sinatra. 
\item Restructuring - Having gone now through roughly 1/3 of this book (as well started work on an NPC generator script), I can safely say that there are definitely aspects of this book that can be re-arranged. I don't have concrete ideas yet, but I as a start I will put all chapters necessary for playing an adventurer before boat combat and information about the Onivero and Jaern. Also, I will likely split up the "Playing an Adventurer" chapter to 
\end{enumerate}
\section{To Do}
\begin{itemize}
	\item Finish copying remaining old text. Prioritize player-utilized chapters first (Nomad Incants, Elemental Spells, Priest Spells, Glossary, Tables)
	\item Redo all formatting.
	\item Create Player Model template and import data.
	\item Update gender of remaining old text.
	\item Update/remove "slavery" and pimping references.
	\item Correct any logical inconsistencies.
	\item Fix line wrap with highlighting to not screw with spacing so much.
	\item Come up with a better indexing system that doesn't require all lowercase.
	\item Update Mets/Feet/Mile/Kilometer charts to not all start on a new page (maybe drop to 200?).
\end{itemize}
\section{Changelog}
\begin{itemize}[leftmargin=12pt]
\item DATE-TBD: Initial version
\end{itemize}
\chapter{How to Read This Book}
\section{Dice Notation}
Dice come in many different sizes, and when a die roll is required, the type and number are expressed like this:

\begin{normboxc}[Dice Notation]
\textbf{(\# of dice) d (sides of dice) [+ modifier]}
\end{normboxc}

This notation means to take so many of a kind of dice, roll them, and add the results of each die, with an optional modifier added to the end.

Examples:
\begin{itemize}
\item 3d6 - Roll three dice which each have six sides.
\item 1d4 - Roll one die with four sides.
\item 5d8+6 - Roll five dice with eight sides, add the totals, and then add 6 to the result.
\end{itemize}

\begin{normboxc}
Always assume six-sided dice if the number of sides per die is not specified.
\end{normboxc}

\subsection{Die Types}
There are numerous kinds of die available. Most players new to playing tabletop role playing games are likely familiar with the standard d6, which is present in many board games. Eventually, you will learn the die by their shape, but they sheer number of die types can be confusing. If you're ever in doubt, look for the largest number you can see on a die, or ask another player.

Players of \aq can expect to use the following dice regularly:
\begin{itemize}
\item d20 - Players will need 1x twenty sided die. This is used to roll for combat attacks.
\begin{itemize}
\item Note: There are two kinds of d20s: "Normal" and "Spindown." Normal die have each consecutive number as far as possible from the next number in the series (thus 1 and two are on nearly opposite sides of the die). Spindown die, often used as counters, have consecutive number touching the previous one, making finding the next number easy for someone using it as a placeholder. Various people have conducted testing of the probabilities and have determined that there is a greater variance between any two die of the same kind than of one type over another. Thus either variety can be used.
\end{itemize}
\item d10 - Players will need 2x ten sided dies. These are used for certain weapons and for rolling "percentiles"
\begin{itemize}
\item Note: There are two kinds of d10s: ones that are numbered 1-10, and ones that are numbered 00-90 (by 10s). It may be beneficial to have one of each t9o role percentiles (described below), but either can be used in place of a d10 for a standard role (with the understanding that 00 is "10"). It's a good idea if your d10s are the same kind to have two different colored ones.
\end{itemize}
\item d6 - Players will need at least 3x six sided dies. These are used for all skill checks.
\end{itemize}

Players will potentially need one or more of the following, depending on what weapons and skills they have chosen:
\begin{itemize}
\item d4
\item d8
\item d12
\end{itemize}

\subsection{Substituting Die}
Die can be substituted for one another, provided the probabilities are equal. For instance, a player who needs to roll a d10 but does not have one, can roll a d20 and cut the result in half. Thus:
\begin{itemize}
\item 1-2 = 1
\item 3-4 = 2
\item 5-6 = 3
\item 7-8 = 4
\item 9-10 = 5
\item 11-12 = 6
\item 13-14 = 7 
\item 15-16 = 8
\item 17-18 = 9
\item 19-20 = 10
\end{itemize}

Similarly, a d12 can be used in place of a d6. Dice can not be multiplied (ie you can't roll 2d6 in place of 1d12, as there is no way to roll a 1).

\subsection{Percentiles}

When the player rolls a critical hit, or under other circumstances, they will need to roll "percentiles." This is a roll from 1-100. There are several ways this can be acomplished:

\begin{itemize}
\item 1d100 - There are 100-sided dies that are sold. It is not recommended that you purchase one unless you really want to.
\item 2d10 - The player can roll two (differentiated) d10 die, and treat one as the tens digit, and one as the ones. Players should state which is which before the roll, if it is not obvious. Each die is treated as 0-9, with the 10 result being 0. The only exception is if both die are 0, in which case the result is 100.
\item 2d20/1d10+1d20 - players can substitute a d20 for one or both die and divide the result of each in half.
\end{itemize}

Examples using 2d10 labeled as "1-10": 

\begin{itemize}
\item Example 1: The 10's die is a 6 and the 1's is a 4. Their result is 64.
\item Example 1: The 10's die is a 6 and the 1's is a 10. Their result is 60.
\item Example 2: The 10's die is a 10 and the 1's is a 4. The 10's die result is considered 0, and their result is 4.
\item Example 2: The 10's die is a 10 and the 1's is a 10. In this, and only this instance, the result is 100.
\end{itemize}


\section{What Order To Read}
This books serves as a players guide, GM's guide, and general reference. You are free to read any and all you wish, in whatever order.

Information necessary to create a character is located in chapters...
	Creating
	Races
	Backgrounds

Information on how to play a character is located in chapters ...
	Skills
	Combat
	Boats
	Weapons and Armor
	Magic
		Nomad
		Elemental
		Divine
		Creating Magic items

Information about the world this game takes place is located in chapters ...
	Historical
		Torandor
		Onivero
	Jaern
		Lojem
		Rougtero

Information about how to run a game is located in chapters...
	Rules
	Running a Campaign
	Example NPCs
	Example Magic Items
	Creatures
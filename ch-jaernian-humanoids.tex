\chapter{Jaernian Humanoids}
\label{ch:jaern-humanoids}
\setlength{\columnsep}{\defcolwidth}\begin{multicols*}{2}
Five races of intelligent beings coexist on Jaern,
each physically and mentally different. Any of the following
races can be used as adventurers. It is important to remember
their characteristics and abilities when you play the role of
various human and non-human races.
\section{Humans}
Humans make up most the population of Jaern.
They came to this place approximately 27 centuries ago on
the Kaaren of Destruction from their home planet Torandor
just before it was destroyed. Humans usually look upon non-
humans with suspicion, distrust, or fear. Humans are more
versatile and flexible than any other intelligent race. They
ha ve m ore abili ty to adapt t o th eir e nvironment. This is
represented by giving them a second Placed Roll when they
are originally generated. Also they have no disadvantages to
overcome or cope with. Humans generally live to the ages
between 60 and 84.
\section{Elves}
Elves are a race of tall, slender, elegant humanoids,
ble sse d wit h height en ed senses o f percepti on, sigh t, and
he ari ng. They c an ju dge v is i ble distances with uncanny
accuracy. Elves are creatures of the wild, and become very
uneasy when they cannot see the sky. While they do possess
life force, they do not have souls, which prevents them from
being brought back from the dead.
\subsection{History}
According to elven history, the elf folk were the
fir st hu manoid s to deve l op sentience on Torandor. What
they lacked in the sciences, they made up for in the social
graces, and the lack of competition allowed them to flourish
and multiply. T hey develo ped a sophisticated culture that
produced a planet of happy, fulfilled, and contented people.
Situations like this rarely stay stable. One night a
large meteor crashed to the ground in the Jelwah province. It
carried a life form infected with a disease that came to be
ca lled Elvesbane, becau se it was fatal to the elven folk.
Millions died, and it looked like the fate of the elven race was
sealed.
But one elf in a thousand was resistant to elvesbane.
T he surviv ors di scov ered t ha t the dise ase had someho w
c ha nge d th e ir nat ure in se ver al way s . T hey n o l on ger
appeared to age. In fact, once they reached puberty, they aged
one year for every five that passed. Also, their ability to bear
young was greatly diminished.
A n o ther e f f e c t o f el v e s b a n e a n d t h e i r c l o s e
connection to nature, is that elves only recover from damage
and exhaustion by placing themselves in a trance rather than
going to sleep. An elf requires 12 full hours to regain the lost
d a mag e po i nt s a nd e lem e n tal u n it s t h a t a h u man can
regenerate in eight hours.
Today’s elves are a happy race with much frivolity.
They enjoy playing practical jokes on visit ors, which has
made them the natural enemies of orcs. War does not come
naturally to elves, but they make fierce fighters when pressed.
\subsection{Appearance}
Elves average six feet in height. Males and females
are built similarly to humans, except that they are generally
more slender, lighter, and less muscled. Elven blood is green
in colo r, w hich gives them a light, green ish complexion.
Their ears point upward, and their hair is thinner than other
races.
\subsection{Technology}
Elven technology is no more advanced than that of
the other races. They tend to use things found in their natural
state rather than go to the trouble of making something from a
new technology.
\subsection{Transportation}
Elves enjoy the land, and prefer to walk. They will
travel by horseback or ship if the distance is great or speed is
important.
\subsection{Cities and Architecture}
El v e n c i ti e s a r e com m onl y f o und in fo r e s ts .
Buildings are well-lit, and all rooms have windows because
elves are racially claustrophobic. Their houses are usually
simple platforms, or huts, suspended high in the trees. What
little furniture they use is typically made of wood.
\subsection{Agriculture}
Elves enjoy hunting for meat, and gathering nuts,
roots, and berries from the forests and fields. Their carefree
nature l eaves them little t ime , o r inc lina tion, to plant or
harvest crops.
\subsection{Medicine}
E l ven med i c in e s a re c o mpo s e d of h e rb s a n d
poultices. They are not superior to those of other races, but
illustrate elven ways. Elves generally live to an age between
200 and 280.
\subsection{Language}
Elvish is a very melodious and harmonic language.
Elves enjoy teaching their tongue to others, and prefer to
speak their native tongue when possible. Written elvish is a
stylish script, very elegant to look at but difficult to read.
\subsection{Art}
Elven paintings depict nature and the environment,
a nd t he ir h istory ca n b e fou nd in t hei r ar tist ry . Th e ir
depictions of love and war are glamorous and heroic, not
gruesome or realistic.
Elven dances are graceful to behold, with smooth
mot ions, gestures , an d m ovement s. El ve n m usic is very
so othing and nat ur al soundi ng, and is often mistaken for
natural woodland sounds.
\subsection{Sports}
Elves are more interested in playing games than
participating in fierce competitions. Games such as hide and
seek are very popular. They enjoy sneaking up on an animal
and touching it, rather than killing it for pleasure.
\subsection{Religion}
Elves are free to worship any god or goddess they
desire. Many worship Ra, and Isis is highly favored for her
benevolent and kind nature.
\subsection{Economy}
Elves are very communistic, and sharing is very
popular. They do not have a good sense of prices, since they
do not use money amongst themselves, and they value the
possession of historic artifacts above all else.
\subsection{Government}
E l v e n gove r n m e n t s a r e v e r y o r g a n i z e d and
heirarchal. While they rarely have a set abode or physical
locat i o n , e lv en n o b le s m e e t pe r i o d ica l l y to a i r t h e ir
grievances, adjudicate differences, punish wrongdoers, and
speak of the greater events in the outside world. Important
events, like wars or natural disasters, cause elves to gather
from all over to discuss plans and policies.
\subsection{Traditions}
Elven society is patriarchal. Elven fathers pass their
names, titles, and possessions on to their first born sons.
Elven women are always protected, and prized as wives by
other races for their beauty and pleasant natures.
\subsection{Elven Abilities}
\subsubsection{Exceptional PER}
Keen senses possesed by most elves entitle them to
one rank of Exceptional PER. Any time an elf needs to make
a resistance check or a stat check against his PER, he may
attempt it at one less die than normal.
\subsubsection{Distance Judgment}
If an elf desires, he can know the exact distance
from him to any object he can see.
\subsubsection{Missile Skill}
Being very good at knowing distances allows an elf
to shoot missiles more accurately. Add the numb er of an
adventurer’s elvish grandparents to all rolls “to strike” when
he uses his missile modifier.
\subsubsection{Soulless}
Having no soul is both a curse and a benefit to
elves. Without a soul they can not be brought back from the
dead. Sleep is a renewal of the soul, and because of this elves
do not need to sleep. Instead they go into a translike state
while their body regenerates. In this trance they are not aware
of their surroundings, Their body heals and, if t hey have
learned to use it, regenerates elemental energy.
Elve s are i mmu ne to spell s an d ma te r ia ls th at
induce a forced sleep. Also elves are immune to love potions,
as love is an affair of the soul. In general, while elves can be
affectionate of others, they consider actual love a weakness to
which they are not prone.
\section{Dwarves}
Dwarves are a short, stout humanoid race that has
e volved within subterranea n chambers . They average fou r
and a half feet in height and are usually heavier than their size
w ou ld di ctate. D warf ma l es an d fem ales a re bu ilt very
similarly, except that the females do not sport the beards
common to all males have after adolescence.
The Dwarves that escaped Torandor’s destruction
were not entirely pleased at their arrival on Jaern. Dwarves
hate water, and the prospect of living on a planet covered
almost entirely water made their disposition even grumpier
than usual.
Dwarves are a stern race that sees humor as an
unnecessary frivolity. When working, work is the only thing
o n t h e i r m i n d . T h e y t a k e e n o r m o u s p r i d e i n t h e i r
craftsmanship, and all other considerations come second to
this.
Living very structured lives does not mean they do
not have a lighter side. When the time to work has ended,
they leave all thoughts of work behind them, and make a
serious job of relaxing. Many of them can be found in local
bars, telling old war stories and making inept passes at the bar
maids.
\subsection{History}
Dw arves evolved from a race of c ave dwelli ng
humanoid s. They lived be neath the surface for c entu ries ,
having an extr eme cult ural fear of open spaces. Hu mans
mistakenly believed that dwarves were horrible monsters that
only came out at night to steal children and eat them. It was
considered good form for humans to hunt down and slaughter
dwarves.
Eventually, a brave human captured one of these
heathe n monsters to try to lear n more about the m. After
spending six months learning the dwarf’s language, the man
learned that dwarves weren’t really bad people at all. The
dwarf learned that being above the surface was not the terror
he believed it would be. This dwarf returned to his people and
slowly lead them into the open. Dwarves were persecuted by
oth er h umano i ds for man y cen turies aft er t hat , but they
eventually earned their place in society.
\subsection{Technology}
Dwarves have a good reputation of working with
metal and stone. They are the builders among the races of
Jaern. They are often sought for metal forging, since they
understand the properties of metal in all its forms and can
make items from metal with surpassing ease. An all day job
for a human blacksmith is a light morning’s work for a dwarf.
\subsection{Transportation}
Dwarves have trouble riding horses because of their
squat stature. Walking also takes longer, so they prefer to ride
wagons and carriages instead.
Dwa r ve s develo ped a rail sys tem, usin g m ul e-
pulled ore cars, to move ore out of the mines. They also use
the cars to descend into the mines.
\subsection{Cities and Architecture}
Dwarven cities are commonly found on sides of
mountains and volcanos. The homes and buildings in these
ci ties sh ow t he d w arve s ’ gr eat s ki l l an d pri de i n t heir
craftsmanship. The detailing used in their architecture is very
intricate and detailed. Dwarves do not need as much light as
other races, so their buildings appear dimly lit.
Furniture is typically made of wood or stone, and
serves as another excellent venue of dwarven artistry and
comfort.
\subsection{Agriculture}
Dwarves do not like raising plants, considering it
beneath their dignity as craftsmen. They often exchange their
crafts for foods instead of coinage. If unavoidable, dwarves
will hunt for their food.
\subsection{Medicine}
Medic ine s are ra r el y u sed among dw arves , not
th r ou gh ignoran ce bu t through lack of need. Their h igh
stamina and health help deal with most diseases and injuries
a t an astonis hing r ate . Dwarves gene rally liv e to an age
between 140 and 180.
\subsection{Language}
The dwarven language is very powerful and deep
s o un d in g . T hey ar e so m ewh a t r e lu c tan t to t e a ch th e ir
language to other races. Dwarven writing is composed of
runes that represent ideas and concepts, and is very difficult
for others to learn.
\subsection{Art}
Dwarven artistry springs forth in their stone and
metalwork. Typical themes are of war and dwarven history.
They can spend years detailing their works.
They enjoy telling tales of their heritage in song
and verse. Their eloquence often conjures visions of the past
in their listener’s minds.
\subsection{Sports}
T her e ar e f e w s po r t s i n w h ich dwar ves wi ll
participate. Their activities during their free time are chiefly
drinking contests and arm wrestling. They are also fond of
barroom brawls, often started by someone commenting on
their height.
\subsection{Religion}
Most dwarves commonly worship Osiris, since she
is the mother of nature and the earth. T’or is also revered for
his warlike and structured nature.
\subsection{Economy}
Dwarves take such pride in their workmanship that
they will only part with their creations at a reasonable profit.
Dwarves are very c apitalistic and value gems and crafted
materials highly.
\subsection{Government}
Dw arves are m onarc hi al , an d chieftain cies an d
kingships are hereditary. When a king or chief dies with no
heir, ranking nobles pick the dwarf with the most valor in
battle to fill the vacancy. General social status is determined
by accomplishments, prowess, and courage in battle.
\subsection{Dwarven Abilities}
\subsubsection{Exceptional HEA}
Hardy bodies and fine toned muscles possesed by
most Dwarves entitle them to one rank of Exceptional HEA.
Any time a dwarf needs to make a resistance check or a stat
check against his HEA, he may attempt it at one less die than
normal.
\subsubsection{Knowledge of Material Composition}
A dw arf can oft en ide nti f y stone a nd metalli c
materials which they have a familiarty with. They do this by
simply handling the object. This ability will not work for very
unusual or magically enchanted objects.
\subsubsection{Armor Construction}
A dwarf’s detailed knowledge of armor materials
and construction enables him to strike armored opponents
mor e eas il y t h a n o t her s . W hen a tt a c k i ng a n ar m or e d
o p p o n e n t , a d w a r f c a n a d d h i s n u m b e r of d w a r v en
grandparents to the “to strike” roll.
\subsubsection{Great Durability}
Dwarves recover from wounds more quickly than
any other race. A full night’s rest restores their HEA/2,
rounded down, in lost damage points. This healing ability
directly conflicts with magic, so healing magic has no effect
on dwarves.
\section{Orcs}
Orcs are a sho r t, hea vy hu ma noid ra c e. T he y
average at five feet in height and are usually heavy in build.
Orcs males and females are built very much like humans.
They have lar ge, protruding canines and lower bicuspids.
They have flat noses, and are considered very ugly by human
standards.
Orcs are uncouth. Th ey do not bathe often, but
ironically they have a very well-developed sense of smell.
Other peoples usually steer clear of orcs due due to their
sm ell. Scuffles and disagrem ents with others, and amo ng
themselves, are common since orcs are incredibly stubborn,
both mentally and physically.
This stubborn s treak is evident in their deal i ngs
w it h oth e r s . The y a r gu e fi er cel y wh e n barg ai ni ng , a nd
invariably believe they have won any verbal exchange. An
argument between orcs is a truly impressive sight. Orcs are
usually avoided by the other races because of th eir slow,
vulgar wits and body odor.
Orcs are energetic and temperame nta l creatures.
Their high level of physical activity must be driven by a good
diet. All orcs require at least one pound of freshly killed meat
per day to maintain this level of activity. For each day they do
not eat fresh meat, they temporarily loose one rank of STR,
c um ul at iv e. Wh e n t heir STR r eac hes z e r o, t hey d ie of
starvation.
\subsection{Technology}
Orcs are very primitive and warlike in nature. Their
greatest achievement is in the area of torture. They will steal
any technology they can find, and any devices that might help
them in combat.
\subsection{Transportation}
Orcs like traveling in wagons or in sedan chairs.
Orcs tend to be lazy, and subjugate weaker people into doing
the hauling, be it carrying the sedan chairs or harnessing them
like mules to their wagons.
\subsection{Cities and Architecture}
Orcs build haphazardly, but in their eagerness they
often over-engineer, so their strange looking abodes are very
sturdy. Just wher e t hey pu t th em i s often co nfused , but
eventually enough houses are close enough to each other to
be mistaken for an orcan town or city.
\subsection{Agriculture}
Orcs dislike farming and raising animals because it
is too complex. Adolescent orcs often hunt for food to fill the
larder and work out their aggressions on something other than
each other.
\subsection{Medicine}
Orc medical skills are rudimentary at best, and there
is a high death rate from disease. Orcs generally live to an age
between 40 and 64.
\subsection{Language}
Orcish is a rude, vulgar language. It is littered with
curses and vulgarities, which usually mean the opposite of
what is said. To compliment an orc, for example, one would
say “Yo u are the fi l thiest, mos t sickenin g piec e of horse
manure I’ve seen ever to come out the rear passages of a
lizard .” A t ypical orc greeting h as been kn own to c ause
women to faint an d to redd en the ears of even the most
h ard en ed ma r ine. O r cs h ave n o writ t en la ng ua ge , t ha nk
goodness.
\subsection{Art}
Orcs have little use for art, and find it very amusing
that other races would waste time on such things as painting,
dance, music, singing, and writing.
However, one popula r pastime inv olve s cr eati ve
and unusual methods of procreation. Orcs often keep score
while competitors compete in groups of two or more. They
consider this an artform.
\subsection{Sports}
Orcs enjoy war games and are fierce competitors.
Often the losers lose more then the event. They are commonly
branded as weak, and exiled from the village or enslaved until
they can prove themselves worthy of a better station in life.
\subsection{Religion}
Orcs commonly worship Orus, for he allows them
to c lea rly exp ress their war lus t and an ger. Due to their
fascination with death, some follow Anubis.
\subsection{Economy}
Orcs be liev e that po ssessio n is n in e tent hs of
ownership. Many will take whatever they can get away with
without causing too much trouble.
\subsection{Government}
Orcish government is ruled by their war generals,
and is highly militaristic. The formalities of order usually
break down during times of war.
\subsection{Orcish Abilities}
\subsubsection{Exceptional WIL}
The overbearing stubborness possesed by most orcs
entitle them to one rank of Exceptional WIL. Any time an
orc needs to make a resistance check or a stat check against
his WIL, he may attempt it at one less die than normal.
\subsubsection{Sense of Smell}
Orcs can detect, by smell, the condition of any food
or drink. T hey can often tell if food is edible, rotten, or
poisoned.
\subsubsection{Physical Viciousness}
Orcs are incredibly vic i ous when gr appling, and
rarely “play fair.” Their abilities to use holds and grapples is
rarely matched by non-orcs. Adventurers may add the number
of orcan grandparents to all their rolls “to grapple.”
\subsubsection{Mental Stubbornness}
An orc’s grasp on life is very strong. He only needs
to roll for unconsciousness when his current DP total falls
under 4 damage points, rather than 6. He then uses a d4 for
the roll rather than a d6. If an attack would take him from 1 to
3 points below zero, he is taken to zero points instead and left
unconscious.
\section{Lizards}
\subsection{History}
A race of humanoids lives in relative isolation deep
beneath the ocean’s waves. Evolved from the denizens of the
deep, lizards are native to Jaern. When Jaern’s original sun
went nova, catapulting the planet on its intergalactic journey,
most of the lizards expired. But many were frozen at the
bottom of the sea, and when Jaern took up orbit in the Onra
system and its seas thawed, so did the lizards.
\subsection{Physical Description}
A strange and reclusive race, lizards rarely leave the
deep waters to walk on land. Most lizards stand six to seven
feet tall, with sca ly, hairless bodie s and long tails . Their
tongues are forked, and they have a snout rather than a nose.
Their ears are just small holes in the sides of their heads,
often covered by a flap of skin, and their eyes are larger than
those of most humans.
Male and female lizards are very similar in most
respects, and can only be distinguished by lizards and others
that have spent several years in their company. Lizards are
cold blooded, and have gills that allow them to live beneath
the sea indefinitely. They also have primitive lungs that allow
them to breathe air normally, like other humanoids.
Lizards must immerse themselves in water at least
once every 24 hours or suffer one damage point every three
hours as they dehydrate.
\subsection{Reproduction}
Lizard men and women pair up, forming lifelong
bonds, when they reach adulthood. Approximately once per
year, the female feels the urge to bear young. If she and her
mate decide to bear, the male impregnates the female at the
proper time. Unlike most reptiles, the young gestate within
the female’s body for five months, and are then born live.
The young are cared for and brought up by their
parents for the first four years of their lives. On their fourth
birthday they are brought to a local Creche, where they spend
most of their childhood with other lizards their age.
\subsection{Technology}
Lizards are good ship builders. They are also good
c ar to gr ap h er s , at l e ast f or c o as t li n es. Th e l a ck of fi r e
underwater has slowed their technology and prevented them
from learning how to forge metals. They operate underwater
mines for other races in exchange for finished products. One
of the ores they have found is Pho’ dite, a phosphorescent
element used for lighting underwater. Lizards do not trade
this ore, and keep it hidden when non-lizards are present.
\subsection{Transportation}
Lizards utilize ships for their long range voyages.
They do not use other means of transportation, preferring to
swim or walk from place to place.
\subsection{Cities and Architecture}
L izard citi es a r e fo und un de r wa ter in seas and
lakes. Their buildings are made of stone, and are very sturdy
to withstand tidal forces and currents. Buildings are poorly lit;
there are rumors of large illuminated cities under the sea, but
these stories are unconfirmed.
Furniture is typically made of stone or coral. Chairs
are backless, to accommodate their tails. Designing furniture
and interior dividers by carefully growing and training corals
has been raised to a high art form by lizards.
\subsection{Agriculture}
Lizards commonly farm fish and grow vegetation.
A few l izards, ch oosing to l ive above wa ter, also enjoy
growing crops. They never raise land animals.
\subsection{Medicine}
Medical technology is no more advanced then that
of other races. Their medicine comes from kelp and other sea
plants. Lizards generally live to an age between 80 and 104.
\subsection{Language}
Lizardish is spoken with a lisp, and the letters are
often slurred due to the shape of their tongues. During the
years of separation, the lizards developed two distinct dialects
of Selict. The most common is spoken on land and is easily
spoken and understood by the other races. The other is only
spoken under-water, is difficult to understand, and even more
difficult to speak, without drowning, by non-lizards. They
have no written language.
\subsection{Art}
Lizard artistry lies in the designs of their sea craft.
Most lizards share a racial tendency to use all their skills in an
artful manner, adding flare to such routine tasks as farming,
food preparation, and interior design.
\subsection{Sports}
There are many sports that lizards enjoy, usually
involving swimming, diving, surfing, and racing. They enjoy
racing other underwater creatures, and competing against land
humanoids in water sports.
\subsection{Religion}
Although Lizards are free to worship any god or
goddess they commonly worship Neptune, the god of the seas
and oceans. Osiris is also revered because of the lizards’ love
of nature.
\subsection{Economy}
Lizards highly prize their works, and are very eager
to barter their handicrafts. Lizards are very materialistic, and
would rather trade than sell. Lizards hoard a large portion of
the world’s wealth, which they have recovered from sunken
ships.
\subsection{Government}
Lizards are communal by nature, with no formal
leaders. They gather together whenever a major issue must be
settled. A vote is called, each attender being entitled to one
vote. Lizards find very few things important enough to vote
on, prefe r r ing to ta ke appro priate a cti ons o n th eir own .
Separate villages may sometimes hold such gatherings and
select a lizar d to re pr esen t them at dista nt ga therings. A
decision of such importance has only been made twice in
recent Jaernian history.
\subsection{Lizard Abilities}
\subsubsection{Exceptional AGI}
The quick reptilian movments possesed by most
lizards entitle them to one rank of Exceptional AGI. Any
time an lizard needs to make a resistance check or a stat check
a gain st hi s AGI, he may atte mpt it at o ne less die t han
normal.
\subsubsection{Quickness}
Liza r ds are ve r y quick an d i nstin ct ive in th ei r
actions. If fighting non-lizards, and if the lizard desires, he
gets initiative during combat, even if his companions do not.
\subsubsection{Water Breathing}
Lizards can breathe and move freely under water.
They automatically have swimming skill at rank nine.
\end{multicols*}
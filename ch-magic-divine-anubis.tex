\chapter{Anubis - Lord of the Dead}
\label{ch:divine-anubis}
\section{Domain}
Anubis is the guardian and protector of all souls, guiding them through life and into the true existence of death.
\section{History}
When man was first made by the gods, it is told that there was no death and old age was unknown. When people were injured, their bodies simply healed. Only the occasional hero or martyr would be taken by the gods and brought to Cielo, an infinitely large place of peace and beauty.
Man grew fruitful and multiplied greatly. Over the next few centuries gradually the world was filled up. Then things started to go downhill. Crowding and overpopulation caused strife, hunger, and pain. Since none could die, whole populations were held captive and forced to exist in pain in large refugee camps. The very land became sterile, and food even more scarce.
A young man, a hero who had rescued an entire nation from captivity, was rewarded by the gods by being sent to Cielo. He asked, “Why are so few granted this boon? Why do you revered gods and goddesses ignore the suffering and agony of your worshipers?” The gods told him to be still and take his reward without objection. A rather heated argument ensued, ending by sending the young man summarily to Cielo, after silencing him by changing his head into that of a jackal.
He found other heros and martyrs living in Cielo, and spent the next few years creating a language of gestures and movements to communicate with them. The others did this to satisfy their curiosity as to the origin of this unusual creature. Once he could be understood, he told them his story, and of the plight of the peoples of the world.
Enlisting the aid of history’s heroes, he stormed the gate to the real world and forced the guardian deity there to let them pass. The hoard of heros roamed the world, seeking out the most deserving to send onward to Cielo.
The gods gathered to punish this presumptious mortal who had ruined their paradise. They caused the land to open and swallow up Cielo beneath the ground, now calling it Infero. Its beauty became bleakness instead. The heros there degenerated and become mere specters of their former selves, unable to enjoy life for theirs had been taken. Their brash leader was branded Anubis, “he who destroys,” and was exiled to lead the dead within Infero.
The gods soon realized the benefits of a much smaller population. While the number of worshipers was much smaller, the resulting prosperity actually increased the gods’ powers. Having learned this lesson, they relented on Anubis, elevating him to godhood, and charging him to cull
the old and rescue the miserable from life. He was to maintain the population down at a manageable size, and was given absolute power over the dead to assist him in accomplishing this task.
\section{Motivation}
"Life is a shadow of true existence"
Believing that real existence does not start until death, the followers of Anubis hold that life is a place to train the soul and prepare it for its existence after death. It is the highest honor to be sent to Anubis via human sacrifice (Mind you, many followers don’t take this too literally.)
\section{Aspects}
Unlike the other gods, Anubis is rarely seen outside his home in the underworld. The existence of live souls about him pains him and causes him distress. He delegates the tasks that would take him above Infero to his various minions.
He usually appears as an 80 foot tall, jackal-headed human within his citadel. He sits on an enormous adamantine throne at the front of a huge chamber, four mets long by one met wide. There he grants an audience to all the newly dead, and passes judgement on them before assigning them their place in Infero. His pronouncements can be quite cruel, but Anubis is always just. While he can speak, he usually uses Tusparol, a sign language, to communicate with his priests.
\section{Structures of the Priesthood/Temple}
The priests and priestesses of Anubis’ Sepulchers perform all the holy and administrative tasks needed. Prisoners are assigned all duties involving manual labor. Non-priest freedmen are not allowed to work within the Sepulcher, as they do not understand Tusparol and therefore cannot communicate with the priests.
Priests do not specialize in one style of service. Instead, each priest holds a holy office, is responsible for a certain administrative duty, and must participate in the religious discussions of the Morto-Vojo, their holy books.
\subsection{Organization}
\subsection{Requirements}
\subsubsection{Apprenticeship}
\subsubsection{Initiation}
\subsubsection{Duties}
\subsection{Advancement}
\subsection{Dogma}
\subsection{Traditions}
\subsubsection{Clothing}
\subsubsection{Appearance}
\subsubsection{Speech and Gestures}
\section{Worship}
\subsection{Sacrifices}
\subsection{Donations}
\subsection{Obligations}
\subsection{Penance}
\subsection{Advice}
\section{Prayers}
\section{Holidays and Feast Days}
\section{Relationship to Other Dieties}


\section{Spells}
\renewcommand{\labelenumii}{\arabic{enumii}.}
\begin{tabular}{@{} p{0.25\linewidth} p{0.25\linewidth}}
\begin{enumerate}
	\item Item
	\begin{enumerate}
		\item Subitem
	\end{enumerate}
\end{enumerate} &
\begin{enumerate}
	\setcounter{enumi}{1}
	\item Item 2
	\begin{enumerate}
		\item Subitem
	\end{enumerate}
\end{enumerate} 
\end{tabular}
\pagebreak

\section{Tomboloko}

\spellentry{Find Dead}{1}{Time to Cast: 1 R}{Resist Check: none}{Target: none}{Duration: 10 + 5/F M}{Area: 100 + 50/F' radius}{Effect: locate dead}{Casting this spell causes any dead or undead bodies within the area of effect to radiate a cool white light visible only to the caster. This light can be seen through any material other than iron or adamantine.}
\spellentry{Tombstone}{2}{Time to Cast: 1 M}{Resist Check: none}{Target: touch}{Duration: permanent}{Area: 1 grave marker}{Effect: finishes marker}{Casting this spell on a block of rough-cut stone allows the caster to quickly fashion a finished and inscribed grave marker. The marker can contain any markings which the caster would have been capable of enscribing with the proper enscribing tools.}
\spellentry{Grave Sight}{3}{Time to Cast: 1 M}{Resist Check: none}{Target: touch}{Duration: 1 T}{Area: 1 grave}{Effect: view grave contents}{Casting this spell on a grave marker or a grave will cause a vision to appear to the caster of the contents of the grave.}
\spellentry{Preserve Dead}{4}{Time to Cast: 1 M}{Resist Check: 4d6 vs HEA negates}{Target: 10 + 5/F'}{Duration: 24 + 12/F H}{Area: one body}{Effect: preserves dead tissue}{The caster keeps dead tissue from further decay by casting this spell. If the tissue is animate (as in undead) it can avoid the effects by making a successful RC.}
\spellentry{Grave}{5}{Time to Cast: 1 M}{Resist Check: none}{Target: 10 + 5/F'}{Duration: 10 + 5/F M}{Area: 1 grave}{Effect: opens grave}{Any non-rock ground will split open in a 7 foot by 4 foot rift of up to 6 feet deep when affected by this spell. The caster may close the rift only during the spell's duration.}
\spellentry{Grave Lock}{6}{Time to Cast: 1 M}{Resist Check: none}{Target: touch}{Duration: 20 + 10/F weeks}{Area: one grave}{Effect: protects grave}{This spell allows the caster to protect a grave from grave robbers. Any attempt to open or desecrate the grave will cause 3d8 damage points to the violator.}
\spellentry{Vervakadavro}{7}{Time to Cast: 5 R}{Resist Check: none}{Target: 30 + 10/F'}{Duration: 2 + 1/F H}{Area: one dead body}{Effect: allows movement}{A dead, but whole, body can be given movement by this spell. After the casting, the dead body will follow the caster's simple orders involving movement. The animated body cannot manipulate objects or be given orders about the future, as the spell is only }
\spellentry{Coffin}{8}{Time to Cast: 10 M}{Resist Check: none}{Target: touch}{Duration: instantaneous}{Area: 1 coffin}{Effect: crafts coffin}{Given a sufficient amount of wood, this spell will quickly fashion a box suitable for internment of a body. The workmanship will be equivalent to what the caster could do normally with the proper tools, but the magic crafts the coffin quickly and efficien}
\spellentry{Grave Ward}{9}{Time to Cast: 2 M}{Resist Check: none}{Target: touch}{Duration: 40 + 20/F weeks}{Area: one grave}{Effect: protects grave}{This spell allows the caster to protect a grave from grave robbers. Any attempt to open or desecrate the grave will cause 6d6 damage points to the violator.}
\spellentry{Regenerate Dead}{10}{Time to Cast: 10 M}{Resist Check: none}{Target: touch}{Duration: instantaneous}{Area: 1 body}{Effect: restores decay}{The caster can take a whole, but decomposed, dead body and cause its tissues to regenerate, leaving the body in a healthy, but still dead, state. This spell cannot be used on the living or undead.}
\spellentry{Shrine}{11}{Time to Cast: 10 M}{Resist Check: none}{Target: touch}{Duration: instantaneous}{Area: 1 monument}{Effect: make grave marker}{Cast upon a suitable amount of loose rocks and stones, this spell will fashion an appropriate monument for a grave. The workmanship will be only what the caster is capable of, but the monument will be completed by the end of the spell.}
\spellentry{Grave Curse}{12}{Time to Cast: 10 M}{Resist Check: none}{Target: touch}{Duration: special}{Area: special}{Effect: curses defiler}{The priest says this warding over a recently (less than one year) buried person. While touching the dirt of the grave, incanting the deceased's name, and visualizing the circumstances of death, the priest places a ward upon the grave. When anyone attempts}
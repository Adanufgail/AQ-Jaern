\chapter{Marines for Hire}
\label{ch:marines}
%\setlength{\columnsep}{\defcolwidth}
%\begin{multicols*}{2}
Jaern lacks land masses large enough to support large armies, nor is there a single island with enough excess income to afford one. Soldiers who traveled to Jaern on the Kaaren of Destruction learned to take to the seas to ply their services. Cities, towns and powerful lords hire, or even sponsor, marine troops to defend their homelands, to attack rivals, or to act as a deterrent to their enemies.

Occasional bouts of peace have forced them to be adaptable. When not on hire, they haul cargoes between ports. While they must compete with commercial merchants for normal cargoes, and nomad Rondos for more exotic cargoes, marines are often used to transport cargoes of great worth or risk.

\section{Organization}

Each marine ship, or Maraujo, is a modified galley, usually about 100 to 150 feet from stem to stern. The crew complement varies from 60 to 120 marines. The marine commanding the maraujo is called The Cefo, and their word is law to the marines under their command. Directly under their
command are the chiefs of the four shipboard departments.

The Engineer heads the engineering department, and is responsible for the repair and general condition of the ship. During seaborne combat, they are also responsible for the firing the ship’s catapults and ballistas. The marines under them are called Gunner. They are skilled in building and repairing mechanical systems, and artillery machinery. The senior-most gunner is called the Chief Gunner and is responsible for task assignment. He reports to the Engineer.

The Navigator heads the navigation department. Piloting the ship, and using and maintaining the rigging and sails are done by the pilots and riggers in this department. The Chief Rigger and the Chief Pilot report directly to the Navigator.

The Battle Chief heads the battle department. The dolphineers, lead by the Chief Dolphineer, are responsible for fighting, scouting, cargo hauling, and message delivery. This is the largest department, making up the main battle force of the maraujo. Dolphineers are trained to fight on ship, in the water and on land. The First Trader heads the trade department. They are in charge of sales and purchases of cargo, and the resupply of the ship at each port. The First Trader also serves as the personnel officer, responsible for recruiting new marines and trading prisoners. Traders under their direction assist in sales while in port, and in directing the maraujo’s complement of prisoners. The prisoners cook, clean, do laundry, and generally do
any task too menial for a marine. They are generally treated well, and are important to the steady function of maraujo.

A marujo of one hundred marines is usually divided into the following divisions:

\begin{normbox}[Marine Jobs]
\small
\begin{tabular}{@{} l l l}
Cefo & Maraujo Chief & 1 \\
\midrule
Engineer & Dept head & 1\\
Chief Gunsman & Duty officer & 1\\
Gunmen & Artillery/repair & 5\\
\midrule
Navigator & Dept head & 1\\
Chief Pilot & Duty officer & 1\\
Pilots & Navigation & 5\\
Chief Rigger & Duty officer & 1\\
Riggers & Ship handling & 11\\
\midrule
Battle Chief & Dept head & 1\\
Chief Dolphineer & Duty officer & 1\\
Dolphineers & Fighting/Scouting & 60\\
Dolphineers & Message running & 3\\
\midrule
First Trader & Dept head & 1\\
Traders & Cargo sales & 4\\
Managers & Ships services/slaves & 4\\
\midrule
 & \bf{TOTAL} & \bf{100}\\
 \midrule
Prisoners & Menial jobs & 10\\
\end{tabular}
\end{normbox}

\section{Work at Sea}

In port, a maraujo flies a red and green flag to signify that it is available for hire. A maraujo may be hired
for many reasons: to haul expensive, risky or dangerous cargoes; take on contracts to defend islands, cities and strongholds; or hired to be an attack element in larger force. If hired to fight, the maraujo flies a red flag. If hired to haul cargo, a green flag is flown.

The Cefo insists on honesty from their employer, and will demand know all that a contract entails up front. Broken contracts have been the downfall of many clients, for the contract is more sacred to the marines than any cause. Wars between fleets of maraujo have halted as one client neglects to pay their navy, resulting in their downfall when the marines combined forces and attacked their erstwhile employer.

\section{Combat at Sea}

Marines fight aboard ships, in the water, and on land. Troop tactics have finely evolved over time, and their use of dolphin mounts for mobility and amphibian landing tactics have made marines az formidable threat. Constant drilling while at sea maintains the fighting edge of the dolphineers.

Bludgeoning and edged weapons are not effective below the water, and piercing weapons are not very effective above water, so the marines developed a weapon optimized to their style of combat. This is the maroglave, a cloth glove with the fingers left exposed. A leather strip runs down the
upper side of the hand. Attached to it, by three small metal braces, is a blade, triangular in crossection, with the edge
facing upward. It tapers to a point about eight inches past the wrist.

Underwater the marine thrusts with weapon, as if they were punching. On land, the marine backhands with their maroglave, drawing the edged blade across their opponent. A sheath allows a trained marine to reach across their abdomen, thrust their hand into the sheath, pull the drawstrings and cinch them around a metal hook, and withdraw the weapon, ready for combat, within one round.

\section{Requirements}

A maraujo looking to replace lost marines, or to expand its fighting complement, will fly a blue flag while in
port. Prospective marines inquire as to the departure time of the ship, and gather on the dock just before it leaves. The First Trader announces the number of apprentices required, and as the ship pulls out, calls for the prospective marines to follow. These men and woman jump into the sea and swim
after the maraujo.

This is a test of endurance and strength, but not fatally so. Dolphineers follow behind the swimmers, and as each falters, rescues  and returns them to shore. When the number is down to that required the ship stops, and the recruits are allowed to climb aboard. Since the recruits can only bring what they can swim with, thay rarely have anything in the way of personal possessions. A trader assigns them quarters, and requisitions them clothing, weapons, and any other needed personal items.

Over the next few days each new recruit is interviewed by the chief of the department he aspires to join, to find a berth suitable for his skills and training. Finally all brought to the Cefo, who formally invites them to join the maraujo.

\section{Apprenticeship}

Apprenticeship aboard the marujo is not much different from the tasks and duties of the marines. Recruits are expected to train and drill with the other marines as they learn the use of their weapons, and learn the skills of the department they have joined. Recruits are not allowed to fight for the maraujo, except in defending the ship if it is attacked directly. Time spent as a recruit is usually six months to two
years, depending on the department and the skill of the recruit.

\section{Initiation}

The night before the induction, the recruit and their shipmates consume mass quantities of liquor and become incredibly intoxicated. In the morning, at the crack of dawn, the recruit is roused, and must make their way on deck where the Cefo awaits to induct them into the crew. It has often been a test of will for the greatly hungover recruit to make it through the induction ceremony without incident.

\section{Duties}

Each marine is responsible to the chief of their section to perform all their assigned duties. They must also perform any orders given him by any other officer, or the Cefo. The duty cycle is usually eight hours on duty, four hours training, four hours free, and eight hours of rest. The duty cycles are overlapped in three groups: Morning Crew, Evening Crew and Night Crew. Marines are rotated from one crew to another every couple of months to even out the
different kinds of duties, and give them experience working with all the officers and crew.

\section{Advancement}

Command advancement is regulated strictly by seniority. Time served aboard the maraujo is recorded by the First Trader, who is responsible for assigning promotions when posts are vacated.
Valor in combat is rewarded by awards. The Battle Chief records the number and type of awards given to each marine, and these are used to determine the marine’s income and his split in combat bonuses.

When a marine transfers to another maraujo, they take an automatic four year seniority and a 10\% pay cut, unless the transfer was done as a direct trade between maraujos. Such trades are often done to restore balance between departments. Occasionally two war depleted crews will combine on the better maraujo. There is always an occasion of much negotiation and adjustment, until the new maraujo functions as one.

Occasionally a very full and established maraujo captures another ship, or commissions one to be built. A fraction of their complement, usually the younger marines, transfers to the new ship, bringing a new maraujo into existence.

\section{Discipline}

Discipline aboard ship is tight, yet adaptable. Orders must be carried out, without hesitation. However, the officers understand the crew’s need to release the tension of being confined to the ship. Officers rarely give orders about things that are not strictly needed. Drunken excesses, minor brawls, and wild behavior is tolerated if it does not interfere with ship functions. Social gatherings are often scheduled to allow the crew to relax. A good supply of liquor and minor relaxants is made available through the Trader’s Office at reasonable prices.

When a marine violate orders, the Battle Chief may assign them extra duty hours, suspend their Trader Office privileges, or restrict them to ship while in port. Unlike many other Jaernian institutions, the marines do not believe incorporal punishment or humiliation.

If the offense is grave, the Battle Chief may recommend to the Cefo that the marine be discharged. If there were no deaths involved, the marine is stripped of their seniority by carefully cutting off their left ear, along with all of their rank earrings. All their awards and wealth (except 10 sp) is confiscated, and they are left at the next port. If there was a death involved, the maraujo sails at least 30 mets from any land, and the Cefo tosses the offender into the sea, without weapons or equipment, and sails away. The offender is left to the mercy of the sea; it is rare that any one survives this ordeal.

\section{Traditions}

Maraujos are normally called by the name of their vessel. If a crew survives the destruction of their ship, and has the resources to acquire another, it is always rechristened with the same name as their lost ship. These ships are usually named after heroic men and women of the past. No two ships may hold the same name without inciting a battle between their crews. The honor and lineage of a maraujo is given by the heroism and age of the hero by which it is named.

\subsection{Clothing}

Being in the water as often as they are, marines disdain most clothing that might slow them down while swimming. Non-officer marines usually wear a loincloth, and tight fitting cotton net shirts on deck. Thin, well fitted leather moccasins, with an additional one inch of leather webbing,
assist their speed swimming, while not impeding movement on land.

\subsection{Appearance}

Spending much of their time in the sun, marines are typically deeply tanned. While they will wear little to impede their movement in the water, they do wear earrings to show seniority and honors. Each copper earring in their left ear signifies one year of service. Each silver earring indicates five years of service, while each gold earring indicates twenty years of service. Department heads wear a specially designed earring for their department, and the maraujo captain wears a diamond in their left ear. Honors for valor are different gemstone earrings worn in the right ear. The value of the gemstone is related to the degree of valor being rewarded. These awards are given and paid for through the Trader’s Office at the direction of the Cefo.

\section{Religion}

With their profession offering ample opportunities for a quick demise, marines are often more religious than others. Most marines look to Neptune for spiritual guidance, but some revere Ra. Priests of either faith are often on board as marines themselves. The Cefo and officers always allow time for these priests to hold worship services and give benedictions prior to battle.

%\end{multicols*}
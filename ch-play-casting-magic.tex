\chapter{Casting Magic}
\label{ch:casting-magic}
%\setlength{\columnsep}{\defcolwidth}\begin{multicols*}{2}
\section{Using Magic}
There are two broad classes of magic: elemental and divine. Elemental power is derived from the four elements present in the physical environment: Earth, Fire, Air, and Water. The adventurer chooses one element in which to specialize and serves as a conduit for the power of that element. Magical effects are manifested by calling forth, manipulating, and controlling raw elemental power. The caster's expertise in their craft is measured in Elemental Units (EU).

Each time a caster buys a rank in a spell group, they gains one Elemental Unit (EU). This power may be applied to any Spell Group the caster has purchased the knowledge to use; it is not limited to any specific spell group. Thus a caster who has purchased up through the third rank spells in five spell groups has 15 EU, and may use them to cast any spell they have acquired, even the same spell requiring three EU five times. Elemental Unit and Divine Unit totals must be kept separate as elemental power cannot activate divine spells and vice versa.

Divine magical power derives directly from the Jaernian deities. The adventurer is beholden to a specific God and must perform the duties of their office and serve the cause of their god to receive the power to invoke magical effects. Priests perform their magical effects by manipulating the divine power granted them by their gods. Their mastery of their calling is also measured in Divine Units (DU).
\subsection{Casting and Terminating Skills}
spell!castingspell!terminating
To cast a spell, declare to the GM which spell your adventurer intends to cast. Your adventurer then begins to gesture, with a single hand if the casting time of the spell is a single round, or both hands for longer spells. They also speak out the key word or words that activate the spell. Any spell may be terminated by the caster before
the normal end of its duration by expending one unit. This counts as the adventurer's action for the round.
\subsection{Recovering Elemental and Divine Units}
When a spell is cast, the required units are temporarily deducted from the caster's total. Expended units may be recovered by resting. These units regenerate at a rate of the caster's PWR stat in units for each 8 hours rest, or 12 of meditation for an elf.

For example, a caster with a PWR of 13 recovers units at the rate of 13 units/full rest. 

Divine and elemental unit totals are kept separate, and an adventurer recovers their PWR in units for both types if they have purchased both styles of magic. Priests of Ra (see \chpage{ch:ra}) and Rudri (see \chpage{ch:rudri}) recover spells in unusual ways.
\subsection{Restrictions on Spell Casting}
spell!restrictions
If your adventurer's hands are damaged or restricted, they may be prevented from casting spells. One impaired hand prevents casting any spell with a casting time greater than 1 round; two impaired hands prevents any casting. A damaged or non-functional voice also prevents casting, but a magical silence does not, as the vocal component of a spell!vocal componentspell is more a concentration device than a method of summoning magical power. 

If a spell caster has the spell!one-hand castingOne-Hand Casting skill, they can cast spells longer than one round by making a check. spelL!non-verbal castingNon-verbal casting allows a caster to cast spells without using their voice. See \chpage{ch:skills} for more information.

The properties of the metal Terisium, consume spell energy. If a caster is encircled by this metal, their current EU and DU totals eventually drops to 0 units; the adventurer may recover the units, by resting, once the metal is removed. 

Prisoners capable of spell-casting are often made to wear manacles, collars, and leg irons.
\subsection{Spell Interruption}
spell!interruption
All spells have a fixed casting time. If your adventurer begins a spell and then becomes the target of an attack before the spell is completed, the spell is ruined and they lose the units put into the spell. Spells with a one round casting time may not be interrupted, except by your adventurer's companions. 

Of course, if a fellow adventurer disrupts the spell, they may no longer be a friend.
\subsection{Spell Duration}
spell!duration
Once a spell has been cast and is active, the caster only needs to concentrate on it when they desires to change the spell effect. For example, an Arise spell requires no concentration to hover, but does to lower or raise the target. A Fly spell requires no concentration to move straight at a constant velocity, but does to turn, slow down, or speed up. A conjuration takes no concentration to maintain, but the caster must concentrate on it to make any changes within the conjuration.

Concentrating on the spell restricts the caster's actions to a slow walk (1/5 normal movement rate) for any non-movement spell, and the appropriate movement for movement spells. Also they must maintain line-of-sight (LOS) on the spell effect to be changed. The caster may not speak, nor perform other actions while concentrating.
\subsection{Stressing PWR for Units}
stressingspell!stressing
An adventurer may sacrifice 1 point of PWR for 8 EU or DU by stressing the stat. This may be done at any time and does not count as an action. The caster may use these newly acquired units as they sees fit. The experience cost to replace a point of PWR is quite high, so this is not an action to be taken lightly.
\subsection{Overloading the Spell Group}
spell!overloading
Normally you state your adventurer is casting an acquired spell, expend the necessary unit (deducting them from their current total), and the spell effect is adjudicated by the GM. If the caster lacks the required number of units, the spell may not be cast as they lose all remaining units. However, there are instances where an adventurer can stretch their ability to (and beyond) the limit.

This happens when the total cost of a spell is higher than the caster's rank in a spell group, and they have sufficient units to cast that spell. The caster is extrapolating their knowledge of the gestures and control that may be required by trying to skip one or more necessary steps (spells) in the training process. The adventurer may cast spells above what is allowed normally by overloading. They may not cast any spell 7 ranks or more higher than their highest purchased rank in the spell group; attempting to do so only drains their unit total to 0 DU or EU and no spell effect occurs. Nor may the attempted spell rank be higher than the caster's PWR stat.

If the overload attempt is from 1 to 6 ranks above the caster's highest purchased rank, the attempted spell may work, but other effects are possible considering the uncertainties of the power involved. The required units are deducted from the caster's total despite what happens. Subtract the caster's rank in the spell group from the rank of the attempted spell and add +6 to the result. 

This is the number the player must roll or exceed on 2d6 for the overload to work. The spell fails if the roll comes up short; check the roll against the Overload Effect Table for additional effects. The table only goes up to 11 because if the required roll is 12 and a 12 is rolled, the overload is successful.\\
\begin{normboxc}[Overloading Effect Table]
\small
\begin{tabular}{@{}l l}
\textbf{Roll} & \textbf{Effect}\\
\midrule
2 & caster suffers (units)d4 DP\\
3 & caster drained of all remaining units\\
4 & random spell (from ANY group) falls on caster\\
5 & caster looses consciousness for 1d4 hours\\
6 & caster suffers 1d10 DP\\
7 & no other effects\\
8 & lose one rank in spell group\\
9 & lose two ranks in spell group\\
10 & lose one INT/CSE point permanently\\
\makecell[tl]{11} & \makecell[tl]{lose two INT/CSE points permanently\\(INT for elemental/CSE for divine)}\\
\end{tabular}
\end{normboxc}

Malvern has bought up to rank four in the Fire Magics group, but wishes to cast the eighth ranked spell, Fireball. He expends 8 EU to cast the spell, and the player must roll a 10 or higher (8-4+6=10) on 2d6 for the Fireball to succeed. The roll comes up as 11, meaning the Fireball functions as normal.

Gondo has bought up to sixth rank in the Water Magics group, but wants to cast Ocean Cold, the twelfth ranked spell. He expends 12 EU and needs to roll a 12 (12-6+6=12) on 2d6 for the overload to work. Unfortunately, he rolls an 11, meaning that the spell fails, and he loses two points of INT. He may buy his INT back, but it cannot regenerate on its own. One must be cautious when using spells.
\subsection{Finessing Spells}
spell!finess
The spells any caster learns have been developed over centuries of trial and much error. What has been learned is that when a certain amount of power is called forth and, through specific gestures and words manipulated in such a way, a certain effect happens. Magic is thus more an art than a science due to the vagaries of the raw power, elemental or divine, with which the caster must work.

This is not to say that experimentation is dead; on the contrary, most spells were serendipitously discovered when magicians and priests attempted to refine, or finesse, a known spell effect by judiciously applying a little more power to alter the range, duration, area of effect, or the effect itself. The EU or DU cost required to alter a spell component is always one, and no spell or spell component may be finessed more than 4 times. The sum of the spell rank and the finesses may not exceed the caster's PWR stat.

Finessable parameters within the spell descriptions are denoted by giving their values in two parts. The first part is the base number, followed by a plus sign, and then the
amount that the base number may be modified by each finesse. No number may be modified to less than 0 of any unit by finessing.

For example, the area of effect of a spell could be given as 20 + 10/F' radius. This means the spell normally occupies a 20 foot radius sphere, but each finesse can add or subtract up to 10 feet to this radius. 

To determine if the finesse is successful, add 1 unit for each spell parameter the caster wishes to alter to the base cost of the spell. If the total cost does not exceed the caster's rank in that spell group, the finesse works. If the total cost exceeds the caster's rank, they are overloading the spell group (see above); subtract the caster's rank in the group from the total cost of the spell and add +6 to find the number or more to be rolled on 2d6.

Tolfirion wishes to finesse two parameters of a 2 EU spell. The total cost is 4 EU (2+1+1=4), but the caster has only bought up to the second spell . He therefore is overloading the spell group and must roll 8 or more (4-2+6=8) on 2d6 for the finessed spell to work. If the roll is 7 or less, check the Overload Table for the result.

If the caster finds they lack the required units to meet the total cost, the spell never gets started and the caster loses all remaining units. Pay attention to the costs and your adventurer's current unit totals!

Malvern has bought up to the fourth rank in a spell group and wishes to finesse two parameters of a 2 EU spell. The total cost is 4 EU, but he only has 2 EU left. The spell fizzles and the caster loses his remaining 2 EU, unless he stresses his PWR to gain EU.
\subsection{Limitations on Finessing}
spelL!finess limitations
How much may a spell be altered before it, in essence, becomes a new spell effect that must be researched? No spell, or single parameter of a spell, may be finessed more than 4 times. This could be 1 parameter four times, 2 parameters twice, 2 parameters once and 1 parameter twice, etc. Each spell description shows which parameters may be finessed and the change per finesse.

For example, an adventurer wishes to increase the range of a spell by 2 steps, the duration by 1 step. This is a total of 3 finesses and is possible. If they wished to increase the range twice and the duration twice, it would be possible, as well. But if they wanted to increase the range 3 times and the duration 3 times that is a total of 6 finesses and is beyond the capabilities of the spell.
\subsection{Finessing and Overloading}
spell!finess and overload
This is possible, but obviously very chancy. This occurs when the adventurer wants to cast a spell above their rank in a group, and finesses it. The deleterious effects of lacking the basic spell ability and finesse ability are additive.

For example, a caster wishes to alter a fifth ranked spell so that it is 2 steps smaller but does the same damage as the normal spell. The finesse cost for this would be 4 EU (reduce the area twice (2 EU) and increase the damage twice (2 EU). This is a total of 4 finesses (within the limit) and 9 EU (5+2+2=9). But the caster only has rank 4 in this group. The total cost for this spell exceeds the caster's rank by 5. They must roll 11 or more on 2d6 (9-4+6=11) for the finessed spell to succeed; if they roll 10 or less, check the Overload Table for the grizzly results.
\subsection{Powerful Spells}
spell!powerful
Casting any spell with a base rank 12 or more (before finesses) causes the caster to permanently lose 1 rank in that spell group. The only way to recover this rank is to purchase the rank back with experience points, just as it was originally bought.
\section{Targeting}
target
Targeting is the directing of magical spell energy, and is as important as the spell itself. There are seven targeting methods which determine what is the spell target. Some affect an object, entitling that object to a resistance check to reduce or eliminate the spell effect. Other methods affect an area and are always successful. Each spell description lists the targeting method for that spell.
\subsection{Caster}
target!casterTarget: caster\\
Spells which specify caster as a target can only
affect the person or creature casting the spell.
\subsection{Touch}
target!touchtouch
Target: touch\\
Spells labeled touch require the caster to actually touch the intended target. Only a single object, person, or creature can be affected by this type of spell. If cast during combat at a mobile target, the caster must successfully strike the target to deliver the spell. If the caster attempts to strike and fails, the spell is never cast and the spell energy is not expended. If the target is an unwilling person or creature,
or any object, it is entitled to a resistance check against the spell if one is listed.
\subsection{Multitouch}
target!multitouchTarget: multitouch\\
While a spell labeled MultiTouch is being cast, the caster touches each target they want to affect, during the rounds used to cast the spell. Thus a spell with a target of MultiTouch, which takes three rounds to cast, indicates the caster touches as many targets as they can (or wish) to in those three rounds, and when the casting time is complete, all those touched are affected. If the targets are unwilling persons or creatures, or any objects, they are entitled to a resistance check against the spell if one is listed.
\subsection{Hearing}
target!hearingTarget: Hearing\\
This targeting method involves an audible casting magic, which affects any creatures or persons capable of hearing it. In a large, open area with no other sounds,  creatures or people within a distance of 240 feet of the caster can be affected. Other sound, wind, and obstructions may modify this distance, as adjudicated by the GM. Simply covering the ears does not stop the sound! The targets must have effective earplugs, which stop all other noises as well, to avoid being affected by the spell. If the targets are unwilling persons or creatures, they are entitled to a resistance check against the spell if one is listed.
\subsection{Memorized Location}
target!memlocmemorized locationTarget: MemLoc\\
This targeting method is generally used for spells which move the caster or an object to a distanct place, or let the caster scry or communicate at a distance. To memorize a location the player must state that their adventurer is specifically memorizing a location. The adventurer must spend at least 10 minutes to complete the memorization, and may not memorize more locations than their INT attribute. The adventurer can only remember the fine details needed to target to the memorized location for a period of 4 weeks. Since there is no target object, no resistance checks are needed for these spells.
\subsection{Direction/Distance}
target!directionTarget: X + Y/F unit\\
Spells using this method contain only a distance in the Target: field. The caster specifies the direction the spell is to travel, and the distance at which it will activate. The spell then travels in that direction and activates at the stated distance X units, or at the first intervening object in the indicated direction. Since there is no intended target object, there is no resistance check which could prevent the spell from activating. However, there may be a resistance check against the spell effect. The distance can be finessed by Y units per finess.
\subsection{Line of Sight}
target!line of sightTarget: LOS X + Y/F unit\\
LOS stands for Line of Sight. These spells are cast at an object. The object must be within the listed distance X units, and there must be an unobstructed, straight path from the caster to the object. The distance limitation is based on the details needed for the caster to successfully target the spell. Any intervening objects, glass, water, opaque gases, or darkness prevent these spells from succeeding. LOS spells may not be cast through scrying spells unless the spell specifically states otherwise. Distances can be increased by the amount Y units for each finess.

These spells can be banked off of well-formed mirrors and other optics, but will malfunction in strange ways (GM's discretion) if banked off flawed surfaces. Spells which affect vision also affect the ability to cast LOS spells. For example, Long Eyes increases LOS spell ranges proportionally. Heat Vision allows LOS spells to function in the dark. There are no resistance checks against the activation of these spells, but any listed RC applies to the resulting spell effect.
\section{Areas of Effect}
target!area
As well as understanding how to target a spell, you also need to know how to define what is affected by the spell. In general, spells affect areas, objects, or groups of objects.

Let's deal with areas first. An area is defined by giving a specific size to the spell effect. If the effect is meant to occur to objects within the area, then every object within it is entitled to the resistance check listed in the spell description. If the area itself is to be affected, there is no resistance check. Areas can be expressed as:
\subsection{Radius}
Area: X unit radius\\
This affects a spherically-shaped area with a radius of X units from the point at which the spell is targeted. Intervening objects within the area may partially or fully shield other objects from the spell effect (GM's discretion). Once the spell is activated, the GM may use normal laws of physics to determine how the effect acts, if it's a physical effect.
\subsection{Volume}
Area: X cubic unit\\
This spell affects a particular volume of size X units, whose shape is specified by the caster. No single dimension of this volume may by more than 4 times larger than any other dimension. All objects within the volume can be affected by the spell, and resistance checks may be listed, if appropriate.
\subsection{Cone}
Area: X x Y unit cone\\
This spell affects a conical area Y units long with a X unit diameter base. The point of the cone is at the caster's fingertip. Intervening objects within the area may partially or fully shield other objects from the spell effect (GM's discretion). Once the spell is activated, the GM may use normal laws of physics to determine how the effect acts, if it's physical in nature.
\subsection{Line}
Area: X x Y unit line\\
This area of effect is defined by drawing a line from the caster's finger tip Y units toward the spell target. All objects within a column whose radius is one half of the width (X/2 units) can be affected by the spell. Intervening objects within the area may partially or fully shield other objects from the spell effect (GM's discretion). Once the spell is activated, the GM may use normal laws of physics to determine how the effect acts, if it's physical in nature.
\section{Objects}
An object is a person, a creature or a thing. When a spell affects an object, further restrictions limit what kind or type of object can be affected by the spell.

Area: caster\\
This limits the spell effect to the caster.

Area: single creature\\
This limits the target of the spell to one living creature or person.

Area: single marine creature\\
This type of area further restricts the target to a creature which primarily lives beneath the sea. Many other restrictions, such as living, dead, humanoid or non-intelligent, can be applied in this way.

Area: single plant\\
Yes, plants can be affected by some spells as well.

Area: X unit\\
This limits the spell effect to a single object of no more than X units.

Area: X unit radius\\
This limits the spell to affecting that portion of an object which is within X units of the target point of the spell.

Area: ferromagnetic object\\
The target of this spell is only effected if it can be magnetized. Other classifications, such as transparent, non-metallic, frozen or red can be used in this way.
\section{Groups of Objects}
Often a group of several objects can and will be considered as a single object. If all the objects in the group fit within the limits and restrictions of the spell being cast, and they are all physically touching, the spell will affect the group of objects as though they are one. 

An adventurer, their clothes, backpack, and enclosed objects within the backpack, is considered a single object. A wall, with all of its boards, nails, enclosed wiring, and paint is considered a single object. A brick wall, with bricks and mortar is considered a single object. A ship's hull, with its enclosed superstructure, decking and rigging is considered a single object. A group of more than one persons, creatures, or plants is not considered a single object.

In short, anything constructed as a permanent structure, and any creature carrying non-living objects, are considered as a single object when examining the area of effect of spells.
%\end{multicols*}
\section{Changes Made in this Version}

The following are areas that I felt were either no longer in keeping with the world that I played, were wholely missing, or were conflicting within the text:
\begin{enumerate}[leftmargin=12pt]
\item \textbf{Slavery}: In the original version text, slavery is both depicted as a form of punishment-based indentured servitude, and as a chattel version of slavery in which slaves remain in servitude for life. Additionally, the original text includes the conflicting statements that children cannot be slaves, and that they can be born into slavery or saddled with someone else's slave-debt. As slave labor was often relegated to the background of scenes when I played, and slavery was only utilized as a punishment, I will be removing much of the supporting text for it and updating it to be more akin endentured labor, with the punishment for crimes not being transferable to kin, save for the withholding of inheritance to cover debts. 

References to "\textbf{slave}" will be replaced with "\textbf{prisoner}," which fits with their circumstance as someone who is \textit{temporarily} obligated to perform work as a condition of criminal punishment. 
\item Weapons: Many of the weapons seem to hold nonsensical values with regard to their (sparse) descriptions, often conflicting with historical (and other game's versions) of the weapons. I will be making efforts to update the weapon table to make sense.
\item Souls: Much of the writings of nomadic, divine, and elemental magic systems involve souls and those who have them. There are entire branches of necromancy devoted to it. However, there are odd gaps when it comes to elves. Additionally, there is some confusion on the difference between the mind/soul, specifically in regards to memory and personality (important distinctions for both undead and necromancy). As a result, I have made the following changes:
\begin{enumerate}
	\item Spells and effects which remove or destroy a soul do not kill the target.
	\item Memory and personality of a creature with a soul are stored in the soul, and are stored in the mind for a creature without. This means that a person or creature who loses their soul loses their memories and personality, but are still capable of creating new memories and may develop a similar or radically different personality (similar to amnesia). Additionally, a person or creature who is able to move their soul to another body (which is without a soul) will retain all of their memories, but none of the being whose body they now inhabit.
\end{enumerate}
\item Karfelon: Much of the 2010 version of the manual references Karfelon, including characters, history, locations, and lore. Karfelon was a massive city in a valley created by a man-made sea-wall extending from the bottom of Lojem. Karfelon was destroyed in the late 1990s or early 2000s (the AQ website adventure summaries from 2002 already reference Rougtero, the city founded in the wake of the destruction where surviving refugees rebuilt). As it had been destroyed for nearly a decade (Earth time) by the time I began playing, I never had any attachment to it beyond as a source for lore and a potential place to send adventurers to dive down to for a mysterious treasure. I will be updating the relevant chapters and characters to match ones from Rougtero (perhaps copying some of the more interesting ones from Karfelon).
\item Pronoun Gender" Gender neutral pronouns are in use where applicable, updating from the previous version's masculine pronoun usage.
\end{enumerate}
\section{Conventions Used}
In this version I have included multiple differences to fonts, spacing, and color, which I will make efforts to keep constant across the work, and are documented here:
\subsection{Colored Highlighting}
\begin{enumerate}[leftmargin=12pt]
\item Definitions of terms, concepts, and calculations are \textbf{bold} and highlighted in a \custhl{soulDodgerBlue}{light blue} and wrtiten in the format that best suits them, with units where it makes sense. Numbers will be written out.
	\begin{itemize}\item Example: The base cost for DP is \tcdefine{twenty-five EP}\end{itemize}
\item Calculated values (ie the values that are added to a roll), or are used to derive values, (such as multipliers), are \textbf{bold} and highlighted in \custhl{soulCyan}{cyan}, with units where possible. Numbers will be written in numerical form.
	\begin{itemize}\item Example: multiply the total by \result{10} to determine your adventurer's starting money\end{itemize}
\item Chapter and page number references are \textbf{bold} and highlighted in a \custhl{soulCrimson}{light salmon} and written in a "Ch \#: Title" format for chapters and a "Page \#"" format for pages.
	\begin{itemize}\item Example: Aging is covered in detail in \chref{ch:jaern-humanoids} on \tcpage{ch:jaern-humanoids}.\end{itemize}
\item Die roll which the player makes are \textbf{bold} and highlighted in a \custhl{soulGreenYellow}{light green-yellow} and wrtiten in a "\#d\#"" format.
	\begin{itemize}\item Example: Roll \tcdieroll{4d6} and throw any one die out.\end{itemize}
\item Measurements are bold and highlighted in \custhl{soulKhaki}{light tan} a and wrtiten in a "number unit" format.
	\begin{itemize}\item Example: A ballista is generally \measure{8 to 10 feet} in length and breadth\end{itemize}
\item Quips are use to add flavor to the descriptions, and are \textit{italic} and highlighted in \custhl{soulLightGreen}{light green}.
	\begin{itemize}\item Example: \quip{Shooting your friends in the back is a good way to earn a quick and violent death.}\end{itemize}
%\item are highlighted in a and wrtiten in a format.
%	\begin{itemize}\item Example: \end{itemize}
\end{enumerate}
\chapter{Jaernian Humanoids}
\label{ch:jaern-humanoids}
\setlength{\columnsep}{\defcolwidth}\begin{multicols*}{2}
Five races of intelligent beings coexist on Jaern, each physically and mentally different. Any of the following races can be used as adventurers. It is important to remember their characteristics and abilities when you play the role of various human and non-human races.
\section{Humans}
Humans make up most the population of Jaern. They came to this place approximately 27 centuries ago on the Kaaren of Destruction from their home planet Torandor just before it was destroyed. Humans often look upon non-humans with suspicion, distrust, or fear. Humans are more versatile and flexible than any other intelligent race. They have more ability to adapt to their environment. This is represented by giving them a second Placed Roll when they are originally generated. Also they have no disadvantages to overcome or cope with. Humans generally live to the ages between 60 and 84 years.
\makeline
\section{Elves}
Elves are a race of tall, slender, elegant humanoids, blessed with heightened senses of perception, sight, and hearing. They can judge visible distances with uncanny accuracy. Elves are creatures of the wild, and become very uneasy when they cannot see the sky. While they do possess life force, they do not have souls, which prevents them from being brought back from the dead.
\subsection{History}
According to elven history, the elf folk were the first humanoids to develop sentience on Torandor. What they lacked in the sciences, they made up for in the social graces, and the lack of competition allowed them to flourish and multiply. They developed a sophisticated culture that produced a planet of happy, fulfilled, and contented people.

Situations like this rarely stay stable. One night a large meteor crashed to the ground in the Jelwah province. It carried a life form infected with a disease that came to be
called Elvesbane, because it was fatal to the elven folk. Millions died, and it looked like the fate of the elven race was sealed.

But one elf in a thousand was resistant to elvesbane.
The survivors discovered that the disease had somehow changed their nature in several ways. They no longer appeared to age. In fact, once they reached puberty, they aged one year for every five that passed. Also, their ability to bear young was greatly diminished.

Another effect of elvesbane and their close connection to nature, is that elves only recover from damage and exhaustion by placing themselves in a trance rather than going to sleep. An elf requires 12 full hours to regain the lost damage points and elemental units that a human can regenerate in eight hours.

Today's elves are a happy race with much frivolity. They enjoy playing practical jokes on visitors, which has made them the natural enemies of orcs. War does not come naturally to elves, but they make fierce fighters when pressed.
\subsection{Appearance}
Elves average 6 feet in height. Males and females are built similarly to humans, except that they are generally more slender, lighter, and less muscled. Elven blood is green in color, which gives them a light, greenish complexion. Their ears point upward, and their hair is thinner than other races.
\subsection{Culture}
\subsubsection{Technology}
Elven technology is no more advanced than that of the other races. They tend to use things found in their natural state rather than go to the trouble of making something from a
new technology.
\subsubsection{Transportation}
Elves enjoy the land, and prefer to walk. They will travel by horseback or ship if the distance is great or speed is important.
\subsubsection{Cities and Architecture}
Elven cities are commonly found in forests. Buildings are well-lit, and all rooms have windows because elves are racially claustrophobic. Their houses are usually simple platforms, or huts, suspended high in the trees. What little furniture they use is typically made of wood.
\subsubsection{Agriculture}
Elves enjoy hunting for meat, and gathering nuts, roots, and berries from the forests and fields. Their carefree nature leaves them little time, or inclination, to plant or  harvest crops.
\subsubsection{Medicine}
Elven medicines are composed of herbs and poultices. They are not superior to those of other races, but illustrate elven ways. Elves generally live to an age between 200 and 280 years.
\subsubsection{Language}
Elvish is a very melodious and harmonic language. Elves enjoy teaching their tongue to others, and prefer to speak their native tongue when possible. Written elvish is a stylish script, very elegant to look at but difficult to read.
\subsubsection{Art}
Elven paintings depict nature and the environment, and their history can be found in their artistry. Their depictions of love and war are glamorous and heroic, not gruesome or realistic.

Elven dances are graceful to behold, with smooth motions, gestures, and movements. Elven music is very soothing and natural sounding, and is often mistaken for natural woodland sounds.
\subsubsection{Sports}
Elves are more interested in playing games than participating in fierce competitions. Games such as hide and seek are very popular. They enjoy sneaking up on an animal and touching it, rather than killing it for pleasure.
\subsubsection{Religion}
Elves are free to worship any god or goddess they desire. Many worship Ra, and Isis is highly favored for her benevolent and kind nature.
\subsubsection{Economy}
Elves are very communistic, and sharing is very popular. They do not have a good sense of prices, since they do not use money amongst themselves, and they value the possession of historic artifacts above all else.
\subsubsection{Government}
Elven governments are very organized and hierarchical. While they rarely have a set abode or physical location, elven nobles meet periodically to air their grievances, adjudicate differences, punish wrongdoers, and speak of the greater events in the outside world. Important events, like wars or natural disasters, cause elves to gather from all over to discuss plans and policies.
\subsubsection{Traditions}
Elven society is patriarchal. Elven fathers pass their names, titles, and possessions on to their first born sons. Elven women are always protected, and prized as wives by other races for their beauty and pleasant natures.
\subsection{Elven Abilities}
\subsubsection{Exceptional PER}
Keen senses possessed by most elves entitle them to 1 rank of Exceptional PER. Any time an elf needs to make a resistance check or a stat check against their PER, they may attempt it at 1 die less than normal.
\subsubsection{Distance Judgment}
If an elf desires, they can know the exact distance from themselves to any object they can see.
\subsubsection{Missile Skill}
Being very good at knowing distances allows an elf to shoot missiles more accurately. Add the number of the adventurer's elvish grandparents to all rolls "to hit" when they uses their missile modifier.
\subsubsection{Soulless}
Having no soul is both a curse and a benefit to elves. Without a soul they cannot be brought back from the dead. Sleep is a renewal of the soul, and because of this elves
do not need to sleep. Instead they go into a trance-like state while their body regenerates. In this trance they are not aware of their surroundings, Their body heals and, if they have learned to use it, regenerates elemental energy. Having no souls with which to offer, they cannot serve as priests to any God and are incapable of learning divine magic. Similarly, they are also unable to develop the soul-bond that allows nomads to draw energy from the Kurago.

Elves are immune to spells and materials that induce a forced sleep. Also elves are immune to love potions, as love is an affair of the soul. Elves form deep, long lasting, and meaningful bonds. but they do not experience love in the way races do.
\makeline
\section{Dwarves}
Dwarves are a short, stout humanoid race that has evolved within subterranean chambers. They average 4.5 feet in height and are usually heavier than their size would dictate. Dwarf males and females are built very similarly, except that the females do not sport the beards common to all males have after adolescence.

The Dwarves that escaped Torandor's destruction were not entirely pleased at their arrival on Jaern. Dwarves hate water, and the prospect of living on a planet covered almost entirely water made their disposition even grumpier than usual.

Dwarves are a stern race that sees humor as an unnecessary frivolity. When working, work is the only thing on their mind. They take enormous pride in their craftsmanship, and all other considerations come second to this.

Living very structured lives does not mean they do not have a lighter side. When the time to work has ended, they leave all thoughts of work behind them, and make a serious job of relaxing. Many of them can be found in local bars, telling old war stories and making inept passes at the bar maids.
\subsection{History}
Dwarves evolved from a race of cave dwelling humanoids. They lived be neath the surface for centuries, having an extreme cultural fear of open spaces. Humans mistakenly believed that dwarves were horrible monsters that only came out at night to steal children and eat them. It was considered good form for humans to hunt down and slaughter dwarves.

Eventually, a brave human captured one of these heathen monsters to try to learn more about them. After spending six months learning the dwarf's language, the man learned that dwarves weren't really bad people at all. The dwarf learned that being above the surface was not the terror he believed it would be. This dwarf returned to his people and slowly lead them into the open. Dwarves were persecuted by other humanoids for many centuries after that, but they eventually earned their place in society.
\subsection{Culture}
\subsubsection{Technology}
Dwarves have a good reputation of working with metal and stone. They are the builders among the races of Jaern. They are often sought for metal forging, since they understand the properties of metal in all its forms and can make items from metal with surpassing ease. An all day job for a human blacksmith is a light morning's work for a dwarf.
\subsubsection{Transportation}
Dwarves have trouble riding horses because of their squat stature. Walking also takes longer, so they prefer to ride wagons and carriages instead. 

Dwarves developed a rail system, using mule-pulled ore cars, to move ore out of the mines. They also use the cars to descend into the mines.
\subsubsection{Cities and Architecture}
Dwarven cities are commonly found on sides of mountains and volcanoes. The homes and buildings in these cities show the dwarves' great skill and pride in their craftsmanship. The detailing used in their architecture is very intricate and detailed. Dwarves do not need as much light as other races, so their buildings appear dimly lit. Furniture is typically made of wood or stone, and serves as another excellent venue of dwarven artistry and comfort.
\subsubsection{Agriculture}
Dwarves do not like raising plants, considering it beneath their dignity as craftsmen. They often exchange their crafts for foods instead of coinage. If unavoidable, dwarves will hunt for their food.
\subsubsection{Medicine}
Medicines are rarely used among dwarves, not through ignorance, but through lack of need. Their high stamina and health help deal with most diseases and injuries at an astonishing rate. Dwarves generally live to an age between 140 and 180 years.
\subsubsection{Language}
The dwarven language is very powerful and deep sounding. They are somewhat reluctant to teach their language to other races. Dwarven writing is composed of runes that represent ideas and concepts, and is very difficult for others to learn.
\subsubsection{Art}
Dwarven artistry springs forth in their stone and metalwork. Typical themes are of war and dwarven history. They can spend years detailing their works. They enjoy telling tales of their heritage in song and verse. Their eloquence often conjures visions of the past in their listener's minds.
\subsubsection{Sports}
There are few sports in which dwarves will participate. Their activities during their free time are chiefly drinking contests and arm wrestling. They are also fond of barroom brawls, often started by someone commenting on their height.
\subsubsection{Religion}
Most dwarves commonly worship Osiris, since she is the mother of nature and the earth. T'or is also revered for his warlike and structured nature.
\subsubsection{Economy}
Dwarves take such pride in their workmanship that they will only part with their creations at a reasonable profit. Dwarves are very capitalistic and value gems and crafted materials highly.
\subsubsection{Government}
Dwarves are monarchical, and titles are hereditary. When a monarch or chief dies with no heir, ranking nobles pick the dwarf with the most valor in battle to fill the vacancy. General social status is determined by accomplishments, prowess, and courage in battle.
\subsection{Dwarven Abilities}
\subsubsection{Exceptional HEA}
Hardy bodies and fine toned muscles possessed by most Dwarves entitle them to 1 rank of Exceptional HEA. Any time a dwarf needs to make a resistance check or a stat check against their HEA, they may attempt it at 1 die less than normal.
\subsubsection{Material Sense}
A dwarf can often identify stone and metallic materials which they have a familiarity with. They do this by simply handling the object. This ability will not work for very unusual or magically enchanted objects.
\subsubsection{Armor Construction}
A dwarf's detailed knowledge of armor materials and construction enables them to strike armored opponents more easily than others. When attacking an armored opponent, add the number of the adventurer's dwarven grandparents to all rolls "to strike."
\subsubsection{Great Durability}
Dwarves recover from wounds more quickly than any other race. A full night's rest restores their HEA/2, rounded down, in lost DP. This healing ability directly conflicts with magic, so healing magic has no effect on dwarves.
\makeline
\section{Orcs}
Orcs are a short, heavy humanoid race. They average at 5 feet in height and are usually heavy in build. Orcs males and females are built very much like humans. They have large, protruding canines and lower bicuspids. They have flat noses, and are considered very ugly by human standards.

Orcs are uncouth. They do not bathe often, but ironically they have a very well-developed sense of smell. Other peoples usually steer clear of orcs due due to their smell. Scuffles and disagrements with others, and among themselves, are common since orcs are incredibly stubborn, both mentally and physically.

This stubborn streak is evident in their dealings with others. They argue fiercely when bargaining, and invariably believe they have won any verbal exchange. An argument between orcs is a truly impressive sight. Orcs are usually avoided by the other races because of their slow, vulgar wits and body odor.

Orcs are energetic and temperamental creatures. Their high level of physical activity must be driven by a good diet. All orcs require at least 1 pound of freshly killed meat per day to maintain this level of activity. For each day they do not eat fresh meat, they temporarily loose 1 rank of STR, cumulative. When their STR reaches zero, they die of starvation.
\subsection{Culture}
\subsubsection{Technology}
Orcs are very primitive and warlike in nature. Their greatest achievement is in the area of torture. They will steal any technology they can find, and any devices that might help them in combat.
\subsubsection{Transportation}
Orcs like traveling in wagons or in sedan chairs. Orcs tend to be lazy, and subjugate weaker people into doing the hauling, be it carrying the sedan chairs or harnessing them like mules to their wagons.
\subsubsection{Cities and Architecture}
Orcs build haphazardly, but in their eagerness they often over-engineer, so their strange looking abodes are very sturdy. Just where they put them is often confused, but eventually enough houses are close enough to each other to be mistaken for an orcish town or city.
\subsubsection{Agriculture}
Orcs dislike farming and raising animals because it is too complex. Adolescent orcs often hunt for food to fill the larder and work out their aggression on something other than each other.
\subsubsection{Medicine}
Orc medical skills are rudimentary at best, and there is a high death rate from disease. Orcs generally live to an age between 40 and 64 years.
\subsubsection{Language}
Orcish is a rude, vulgar language. It is littered with curses and vulgarities, which usually mean the opposite of what is said. To compliment an orc, for example, one would say "You are the filthiest, most sickening piece or horse manure I've seen ever to come out the rear passages of a lizard." A typical orc greeting has been known to cause women to faint and to redden the ears of even the most hardened marine. Orcs have no written language, thank goodness.
\subsubsection{Art}
Orcs have little use for art, and find it very amusing that other races would waste time on such things as painting, dance, music, singing, and writing. However, one popular pastime involves creative and unusual methods of procreation. Orcs often keep score while competitors compete in groups of two or more. They consider this an artform.
\subsubsection{Sports}
Orcs enjoy war games and are fierce competitors. Often the losers lose more then the event. They are commonly branded as weak, and exiled from the village or enslaved until they can prove themselves worthy of a better station in life.
\subsubsection{Religion}
Orcs commonly worship Orus, for he allows them to clearly express their war lust and anger. Due to their fascination with death, some follow Anubis.
\subsubsection{Economy}
Orcs believe that possession is nine tenths of ownership. Many will take whatever they can get away with without causing too much trouble.
\subsubsection{Government}
Orcish government is ruled by their war generals, and is highly militaristic. The formalities of order usually break down during times of war.
\subsection{Orcish Abilities}
\subsubsection{Exceptional WIL}
{willpower!WIL}
The overbearing stubbornness possessed by most orcs entitle them to 1 rank of Exceptional WIL. Any time an orc needs to make a resistance check or a stat check against
their WIL, they may attempt it at 1 die less than normal.
\subsubsection{Enhanced Smell}
Orcs can detect, by smell, the condition of any food or drink. They can often tell if food is edible, rotten, or poisoned.
\subsubsection{Physical Viciousness}
Orcs are incredibly vicious when grappling, and rarely "play fair." Their abilities to use holds and grapples is rarely matched by non-orcs. Adventurers may add the number of the adventurer's orcish grandparents to all rolls "to grapple."
\subsubsection{Mental Stubbornness}
An orc's grasp on life is very strong. They only need to roll for unconsciousness when their current DP total falls under 4 DP, rather than 6. They then use 1d4 for the roll rather than 1d6. If an attack would take them to between -1 to -3 DP, they are taken to 0 DP instead and left unconscious.
\makeline
\section{Lizards}
\subsection{History}
A race of humanoids lives in relative isolation deep beneath the ocean's waves. Evolved from the denizens of the deep, lizards are native to Jaern. When Jaern's original sun went nova, catapulting the planet on its intergalactic journey, most of the lizards expired. But many were frozen at the bottom of the sea, and when Jaern took up orbit in the Onra system and its seas thawed, so did the lizards.
\subsection{Physical Description}
A strange and reclusive race, lizards rarely leave the deep waters to walk on land. Most lizards stand 6 to 7 feet tall, with scaly, hairless bodie sand long tails . Their
tongues are forked, and they have a snout rather than a nose. Their ears are just small holes in the sides of their heads, often covered by a flap of skin, and their eyes are larger than
those of most humans. 

Male and female lizards are very similar in most respects, and can only be distinguished by lizards and others that have spent several years in their company. Lizards are cold blooded, and have gills that allow them to live beneath the sea indefinitely. They also have primitive lungs that allow them to breathe air normally, like other humanoids.

Lizards must immerse themselves in water at least once every 24 hours or suffer 1 DP every 3 hours as they dehydrate.
\subsection{Reproduction}
Lizard men and women pair up, forming lifelong bonds, when they reach adulthood. Approximately once per year, the female feels the urge to bear young. If she and her mate decide to bear, the male impregnates the female at the proper time. Unlike most reptiles, the young gestate within the female's body for five months, and are then born live.

The young are cared for and brought up by their parents for the first four years of their lives. On their fourth birthday they are brought to a local Creche, where they spend most of their childhood with other lizards their age.
\subsection{Culture}
\subsubsection{Technology}
Lizards are good ship builders. They are also good cartographers, at least for coastlines. The lack of fire underwater has slowed their technology and prevented them
from learning how to forge metals. They operate underwater mines for other races in exchange for finished products. One of the ores they have found is Pho'dite, a phosphorescent element used for lighting underwater. Lizards do not trade this ore, and keep it hidden when non-lizards are present.
\subsubsection{Transportation}
Lizards utilize ships for their long range voyages. They do not use other means of transportation, preferring to swim or walk from place to place.
\subsubsection{Cities and Architecture}
Lizard cities are found underwater in seas and lakes. Their buildings are made of stone, and are very sturdy to withstand tidal forces and currents. Buildings are poorly lit; there are rumors of large illuminated cities under the sea, but these stories are unconfirmed.

Furniture is typically made of stone or coral. Chairs are backless, to accommodate their tails. Designing furniture and interior dividers by carefully growing and training corals has been raised to a high art form by lizards.
\subsubsection{Agriculture}
Lizards commonly farm fish and grow vegetation. A few lizards, choosing to live above water, also enjoy growing crops. They never raise land animals.
\subsubsection{Medicine}
Medical technology is no more advanced then that of other races. Their medicine comes from kelp and other sea plants. Lizards generally live to an age between 80 and 104 years.
\subsubsection{Language}
Sel'ict is spoken with a lisp, and the letters are often slurred due to the shape of their tongues. During the years of separation, the lizards developed two distinct dialects of Sel'ict. The most common is spoken on land and is easily spoken and understood by the other races. The other is only spoken under-water, is difficult to understand, and even more difficult to speak, without drowning, by non-lizards. They have no written language.
\subsubsection{Art}
Lizard artistry lies in the designs of their sea craft. Most lizards share a racial tendency to use all their skills in an artful manner, adding flare to such routine tasks as farming, food preparation, and interior design.
\subsubsection{Sports}
There are many sports that lizards enjoy, usually involving swimming, diving, surfing, and racing. They enjoy racing other underwater creatures, and competing against land humanoids in water sports.
\subsubsection{Religion}
Although Lizards are free to worship any god or goddess they commonly worship Neptune, the god of the seas and oceans. Osiris is also revered because of the lizards' love of nature.
\subsubsection{Economy}
Lizards highly prize their works, and are very eager to barter their handicrafts. Lizards are very materialistic, and would rather trade than sell. Lizards hoard a large portion of the world's wealth, which they have recovered from sunken ships.
\subsubsection{Government}
Lizards are communal by nature, with no formal leaders. They gather together whenever a major issue must be settled. A vote is called, each attender being entitled to one vote. Lizards find very few things important enough to vote on, preferring to take appropriate actions on their own. Separate villages may sometimes hold such gatherings and select a lizard to represent them at distant gatherings. A decision of such importance has only been made twice in recent Jaernian history.
\subsection{Lizard Abilities}
\subsubsection{Exceptional AGI}
The quick reptilian movements possessed by most lizards entitle them to one rank of Exceptional AGI. Any
time an lizard needs to make a resistance check or a stat check
against their AGI, they may attempt it at 1 die less than normal.
\subsubsection{Quickness}
Lizards are very quick and instinctive in their actions. If fighting non-lizards, and if the lizard desires, they get initiative during combat, even if their companions do not.
\subsubsection{Water Breathing}
Lizards can breathe and move freely under water. They automatically have swimming skill at rank 9.
\end{multicols*}
\textbf{INTRODUCTION}

\textbf{Adventure Quest\textsuperscript{TM}} (\textbf{AQ} for short) is a role playing system in which you, through your game persona (adventurer), can experience all the thrills and perform deeds of derring-do in a fantasy world. It is like being the hero in an adventure novel, only, instead of just reading about what happens, your actions and decisions direct the storyline. You can destroy evil maidens, rescue fair dragons, or even be a knight in very dull armor. Your imagination is the only limit to what you can do while playing Adventure Quest.

As a player, you create an adventurer which you control. Another person, called the Game Master (GM), presents to you and other players a fantasy world of cities, towns, creatures, oppressive overlords, demanding temples, and lots of magic and treasure. You tackle adventures in this world to satisfy the personality and motives of your adventurer. Adventure Quest tm provides adventure in a variety of different settings (Games), each with its own history, customs, inhabitants, villains, and deities. 

This Game covers adventuring in JAERN, a distant fantasy world far in our future. Other Adventure Quest games include AQ/BRITANNIA, describing a world similar to the British Isles in t he mid 1200’s; AQ/KHEMET, providing adventure in a land akin to ancient Egypt; AQ/FREEZONE, a coorporate ruled gangland in the near future; and AQ/SPACE, for adventuring in the outer reaches of Interstellar Space among the Pan-Human Hegemony.

\textbf{Realism and Playability}

Adventure Quest/Jaern is a complete game; you do not have to buy any other books before beginning play. It contains all the necessary information for players to create and play their adventurers, and for Game Masters to design and maintain a campaign. Any game such as this must strike some kind of balance between realism and playability. The mechanics used in this manual lean heavily towards the latter, with the idea that you should spend your time roleplaying your creations, be you a player or Game Master, rather than wading through very complex rules for the sake of realism.

That said, we realize that some of you might be willing to make a different tradeoff. Where appropriate, optional rules are included offering different, but more complex, mechanics that arguably provide greater realism. The players and Game Master may choose which options to include to tailor the game to their liking. The cornerstone of \textbf{Adventure Quest\textsuperscript{TM}} games are flexibility. Much of the game book deals with the creation of personalities, creatures, magical items, etc. Examples are provided that you can use as is, but more importantly we tell you how to create your own that will automatically be balanced with the system.

\textbf{About Role Playing}

Playing Adventure Quest, like any role playing game, should be a fun and exciting experience. Your adventurer will likely encounter many unusual, exotic, and strange situations, people, and activities. Your adventurer may end up in conflict with, or allied to, an array of intelligent beings and creatures, many of which we might consider strange or even evil by today’s standard and mores. Please remember that this is "just a game." The authors in no way endorse or suggest that you act out any game-related actions or methods in the real world. Practice safe gaming, and leave the game and any enemies you make there behind you at the gaming table.

\textbf{How to Use this Book}
\begin{itemize}
\item All players and Game Masters should read Chapters 1 through 4 which deal with the creation and playing of adventurers.
\item Chapters 5 through 10 describe the world of Jaern, the setting for this game, and is therefore also pertinent for both players and Game Masters.
\item Chapters 11 through 27 present the magic available in AQ/Jaern. 
\begin{itemize}
\item Chapter 11 discusses nomadic mystiscism. 
\item Chapters 12 through 16 deal with elemental magic, and are therfore of primary interest to players whose adventurers use magician spells.
\item Chapters 17 through 27 deal with divine magic. Each deity has its own chapter, so these are of interest to any player whose adventurer follows a particular god or goddess.
\end{itemize}
\item Chapters 28 through 35 are meant primarily for the Game Master. They discuss creation of actors, creatures, and treasures, designing interesting and exciting adventures, adjudicating adventures, and how to maintain a campaign.
\end{itemize}

\textbf{Pronoun Gender}

Gender neutral pronouns are in use where applicable, updaing from the previous version's masculine pronoun usage.

\textbf{Updates Made in This Version}

The following are areas that I felt were either no longer in keeping with the world that I played, were wholely missing, or were conflicting within the text:
\begin{enumerate}
\item Slavery: In the original text, slavery is both depicted as a form of punishment (akin to an endentured labor) and as a chattel version of slavery in which slaves remain in servitude for life. Additionally, the original text includes statements both that children cannot be slaves and that they can be born into slavery. As slave labor was often relegated to the background of scenes when I played, I will be removing much of the supporting text for it and updating it to be more in line to be a limited time frame of endentured labor, with the punishment for crimes not being transferable to kin, save for the withholding of inheritance to cover debts. References to "slave" will be replaced with "prisoner."
\item Weapons: Many of the weapons seem to hold nonsensical values with regard to their (sparse) descriptions. I will be making efforts to update the weapon table to make sense.
\item Souls: Much of the writings of nomadic, divine, and elemental magic systems involve souls and those who have them. There are entire branches of necromancy devoted to it. However, there are odd gaps when it comes to elves. As a result, I have made a determination that spells and effects which remove or destroy a soul do not kill the target. Additionally, as there is some confusion on the difference between the mind/soul, specifically in regards to memory and personality, I have made the determination that those are part of the soul. This means that a being or creature who is able to move their soul to another body (which is without a soul) will possess all of their knowledge and skills, but is still bound by the physical characteristics of their new form (meaning they may be unable to wield a weapon they are proficient in or cast certain spells beyond their WIL).
\end{enumerate}

\textbf{Original Acknowledgements}

The list below is really just the beginning. Many people have contributed in different ways at different stages of this project. We would especially like to thank Mark Shoemaker for lots of zany ideas and style over many years, Bob Ferguson for his devotion in filling out thousands of forms, to Scott Delaney for fixing all our cars, to Tony Charlesworth for his endless time researching a world full of information, to Greg Mowzko for not letting a single error problem by no matter how insufferable it was, to Microsoft for their Access product that holds all of our databases, and to our good roleplaying friends in Lake Geneva, for providing us the motivation.

Robert J. Blake, my coauthor of this system, created most of the elemental spells, a lot of creatures, many skill descriptions and provided a sounding board for all the basic concepts behind our system. He provided endless encouragement to bring this project to pass. Robert ran the AD\&D Open Tournement at the Gencon Gaming convention for over a decade, overseeing uncountable details of scenario design and game master coordination. It was his experience which made it possible for us to create this system. Also our work on these concepts found its place in improving other systems in many ways. Sadly, we lost Robert at the beginning of the new millenium. He will be greatly missed.

\textbf{Design assistance:}\\
Daniel Lawrence\\
Robert J. Blake\\
Dana Hoggatt\\
Eric Delaney\\
Christopher Stanley\\
Steve Ames\\
Greg Mowczko\\

\begin{tabular}{ l l }
\textbf{Cover Art:} & Terry Pavlet\\
\textbf{Interior Art:} & Sean Cannon\\
 & Kirstein Jacobus\\
 & Scott Starkey\\
\textbf{Logo Design:} & Karen Klutzke\\
\textbf{Database Design:} & Daniel Lawrence\\
\end{tabular}

\textbf{Play Testers:}\\
\begin{tabular}{ l l l l }
Aaron K Alexander & Andrew Grey Smith & Andrew M Luers & Anthony C Brown\\
Anthony Charlesworth & Benjamin Wai & Bill Dorell & Brent D McLean\\
Brett Riester & Brian Cash & Calvin Krug & Charles Voelkel\\
Chris Esser & Daniel Lawrence & Terry Pavlet & Sean Cannon\\
Kirstein Jacobus & Scott Starkey & Karen Klutzke & Daniel Lawrence\\
Chris Normand & Craig Waylan & Dana Shea & David H Morse\\
David Moore & David Ward & Eric Starkey & Erich Shultz\\
Gregory Mowczko & Jeff Hartwig & Jim Ivey & Joe Gregorovich\\
Jose Berrios & John Flournoy & John Haley & John W. Fisher\\
John R. Prewitt III & Kat Price & Kathy Janney & Kirk Monsen\\
Lee Kenyon & Lyle H Janney & Mark Muller & Matthew W Somers\\
Michael Ballard & Michael Brobst & Patrick Collins & Patrick Perrone\\
Paul Bernard & Robert Tertocha & Robert Thelan & Robert Westerman\\
Ryan Gavigan & Ryan Parker & Scott Starkey & Sean L McLane\\
Steve Ames & Terran D Lane & Tessy E Fitzpatrick
\end{tabular}
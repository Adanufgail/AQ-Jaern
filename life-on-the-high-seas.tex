\chapter{Life on the High Seas}
\label{ch:high-seas}
\setlength{\columnsep}{\defcolwidth}
\begin{multicols*}{2}
A very mature planet, the land masses of Jaern have been slowly eroded away, leaving most of its surface covered in a relatively shallow ocean. Thousands of small islands poke above the surface, leading to a proliferation of very diverse biologies and cultures. The ocean itself has a very low salt content and is home to the majority of Jaernian life. Humans and their kin ply the surface of the ocean with a plethora of different vessels, traveling from island to island, or even living at sea.
\section{Ocean Creatures}
\subsection{Fish}
\indx{fish}
The variety of fish swimming the seas of Jaern seems endless. Cold blooded, finned, spined, colored, poisonous and not, there is no end to their sizes shapes and appearances. Fish form the diet staple of most people, and employment for over one fifth of all humanoids in their harvesting. Dangerous fish include vicious \indy{sarko}, and the very poisonous \indy{quezl}.
\subsection{Coelenterates}
\indx{coelenterates}
Almost as numerous as all the other life in the seas, Jaern supports a bewildering variety of these creatures. Small \indy{Atomo}, each less than a hundredth of an inch in size, form the food for most of the grazing fish . Larger jellyfish in myriads of colors float in the top twenty feet of the water, slowly converting light to food by different chemical  processes. The largest of these creatures, the \indy{Flugofiso} can grow up to 60 feet in radius, and generate gases that allow it to escape the waters and float high in the air. Many superstitions surround the sighting of these strange \indy{coelenterates}. 
\subsection{Dolphins}
\indx{dolphin}
These creatures have evolved into highly intelligent creatures. Because of their environment and lack of manipulative limbs they are not tool builders, however they have developed a diverse and complex culture.

Physically they are very similar to Earth's dolphins, they differ in a few important ways. A dorsal stripe of color, blue for females, and yellow for males, assists in their communications by changing shade subtly, indicating general mood and inclinations. A slightly larger skull is protected by thicker bone and fleshy bumps which protect the dolphin's brain from blows and the extreme temperature found in ocean currents. A larger tail fin and a more sleek body allow these creatures to move through waters at speeds approaching \mets{40}{hour}{15}{24}. The life expectancy of these dolphins average from 30 to 40 years.

Dolphins communicate with each other in a highly developed language consisting of whistles, clicks and guttural stops. Some of these are outside the range of \indy{human} (but not \indy{lizard}) hearing. The language itself is weak in complex verbs and time based tenses, but very extensive in descriptive terms describing water and the objects found in the oceans. Some humans can, after extensive training, understand and speak some of this speech, but the dolphin must make a conscious effort to slow its speaking speed.

Forming small groups of 8 to 16 based around social needs and friendship, dolphins are rarely seen alone. They will pair once, making a lifetime commitment to a mate. If they are capable, each pair will bear young live, in pairs and triples, once every 5-10 years. These young are raised by the group until they are old enough to interact with others close to their age, at which time they leave and form new groups.

The dolphins consider Jaern their world, and land based creatures as strange aliens. Generally they do not concern themselves with events on land, however they will relentlessly hunt down those who damage or poison the seas and those whom would capture and enslave dolphins for their own ends. The dolphins will seek a way to punish these humans for these crimes. As for others living beneath the sea, dolphins live in peace with lizardkind, and while they can not ordinarily communicate well, will often be seen in lizard's company.

Occasionally, a young dolphin will meet one outside their own race and form a friendship. If the other wins the dolphin's trust and affection, the dolphin may decide to pair outside their race. This kind of relationship remains one of emotion and friendship. Maraujos often seek out places where many young dolphins swim, looking to find those whom would pair with its youngest marines. The dolphins and the marines, over time , have learned the benefits of such a pairing, which has resulted in the dolphins tolerating the nearby presence of these humans.

Once paired in this way, the dolphin will expect nearly daily attention from his partner. In return, he will devote his time, energies and abilities to help his partner. Over time, the two develop an almost telepathic sense of the other's needs and desires. Many dolphins, and dolphineers have sacrificed even their own lives to help save their partner from harm.
\subsection{Lizards}\indx{lizard}
Evolved in the deeps of the oceans, these intelligent creatures group together forming gatherings. They communicate verbally and sense objects and motion mainly via sound, which travels well under water. Finding dry land uncomfortable, they largely avoid interacting with the humans and humanoids above the water.
\subsection{Balenoigajos}\indx{Balenoigajos}
Mammalian creatures of large size, these  herbivorous sea dwellers subsist on \indy{Atomo} strained through their bodies as they swim. While it is very profitable to hunt
and slay these creatures for their meat and other body products, this profession has been virtually eliminated by the Onivero whom consider these creatures close friends and allies.
\subsection{Oorn}\indx{Oorn}
Evolved from land trees, these plants grow branches along the water's surface and thrust large leafy growths upward toward the light. A single floater can reach sizes of up to a few hundred feet in length. Posing a  navigational hazard to ships, standard shipping lanes in the southern waters that the Oorn usually grow, are regularly swept clean of these plants. Especially large Oorn are sometimes used as the base of small villages inhabited by the \indy{Onivero}, or by some nomads whom have emulated them.
\section{Oceangoing Vessels}
The seas of Jaern are plied by innumerable vessels of many different sizes. Ocean going transportation ranges from that for the single person, to entire ocean dwelling communities.
\subsection{Sail Boards}
\indx{vessel!sail board}
Only suitable for short journeys between nearby islands, these boards are about \measure{6 feet long} and \measure{2 feet wide}. In the center, 2 feet from the front is a socket which excepts a 7 foot tall mast, bearing a single cross spur and a triangular sail, 6 foot wide at the base, coming to a point at the top of the mast. The user of this device stands behind the sail and holds the cross spur at the proper angle to catch the waves. They use their body to steer the board.

This very active and athletic way of travel restricts the journey time to about two to three hours for even the most ambitious athlete. At a speed of up to \mets{10}{hour}{4}{6}, this restricts the normal range of this device to \mets{30}{}{11}{18}.

Marines, trained in sailboarding, will use sailboards as a last resort when their \indy{Maraujo} has sunk. Carrying several days food, they can often go up to a week, traveling \mets{30}{day}{11.3}{18.3}, to attempt to find a place to land. Navigating while using a sailboard is particularly difficult since they will only keep a course when they are held steady by their riders.
Sailboards without a mast and sail are called \indy[vessel!surfboard]{Surfboards} and are often used by marines to land during amphibious assaults. A maraujo will sit just over the horizon from its target, and its dolphineers will ride these surfboards to shore to catch their target by surprise. Then the maraujo will close and disgorge attack boats to pick up the marines after the assault.
\subsection{Dolphins}
\indy[dolphin]{Dolphins} will rarely concede to being used like steeds by humanoids. Normally only dolphineers and a few scattered nomads spend enough time at sea to have an opportunity to meet and pair with a dolphin. Even these people must treat their dolphin as a partner rather than a mount.

While riding a dolphin, the human lies above the dolphin and wraps their arms and legs around the dolphin. Since their arms will rarely reach, they will hold on to a length of leather in each hand, or tied about their wrists. Dolphineers will often have a suitable leather thong attached to their maroglave for this purpose. As the dolphin swims, the rider must match their body movements to the flexing of the dolphin as its tail and back arc up and down. A dolphin and rider traveling like this can achieve speeds approaching \mets{35}{hour}{13.25}{21.3}. Dolphins have good senses about currents, depths, and direction. They will be able to find land with little effort.

More barbaric and oppressive riders have been known to use a leather harness on the dolphin. Such a rider places their feet in the stirrups and attached their harness to the dolphin's. Such a harness is uncomfortable for the dolphin and will injure it in time.
\subsection{Skiffs}
\indx{vessel!skiff}
These small boats are generally about \measure{10 to 12 feet long} and \measure{3 to 4 feet wide}. They are generally constructed of wood planking and have a rudder operated manually at the rear. A mast with a single spur sports a 10 foot high triangular sail. A skiff can generally hold \tcdefine{6 to 8 people} and can travel at speeds of \mets{10}{hour under sail}{3.8}{6}, or \mets{4}{hour rowed}{1.5}{2.4}.

Skiffs are generally used for line of sight travel between nearby islands, and as auxiliary craft aboard larger crafts. The handle only relatively calm seas and will swamp
easily with waves larger than a couple of feet tall.
\subsection{Attack Boats}
\indx{vessel!attack boar}
Larger than skiffs, these boats are generally \measure{25 to 30 feet long}, \measure{8 to 10 feet wide}, and hold up to \tcdefine{30 men}. These boats are generally constructed of wood planking, pegged to form, and covered in many layers of a heavy shellac to waterseal and reduce friction. With no mast, they are driven by six sets of oars mounted midship. At capacity, these boats can be rowed at speeds of up to \mets{8}{hour}{3}{4.9}. They are generally used by larger vessels to aid in debarking and boarding. A Maraujo will often have up to four of these to transport non dolphin riding warriors. Merchant vessels will use these boats as life boats, and to land in places not equipped with a proper dock.
\subsection{Kurujo}
\indx{vessel!kurujo}
Used mainly as couriers and light cargo and passengers, these vessels are generally \measure{35 to 50 feet} long and displace \measure{3,000 to 6,000 tons}. With a crew of \tcdefine{10 to 20 sailors}, they can carry a few passengers, or a limited amount of cargo. Two masts hoist large triangular sails, with a few smaller sails for maneuverability. An unladen Kurujo can travel at speeds of up to \mets{15}{hour}{5.7}{9.1}.
\subsection{Metioujo}
\indx{vessel!metioujo}
These wooden vessels are deep sea ships capable of traveling long distances with large cargos. Used generally by merchants, these carry the majority of traffic at sea. Metioujo are sail driven, and are normally armed to ward off attacks by privateers.

A metioujo's displacement varies from \measure{10,000 to 20,000 tons}, and their length from \measure{80 to 120 feet}. Width, at the center, is usually one third of the vessel's length. There is normally a large cargo deck accessible from one or more cargo hatches on the main deck. Above the cargo deck is the crew deck, containing quarters for the crew, storage for food and operating equipment, and the crew's mess. Above this is the main deck, open to the sea in the middle. At the front of the ship is the forecastle, housing the officer's quarters. At the
rear is the aftcastle, containing the chart room and the weapons locker. Above the aftcastle is the pilot's deck, where the rudder wheel is manned, and the aft ballista is usually
located. Above the forecastle is the \indy[cefo]{cefo's} deck, from where the ship is normally commanded, and the fore ballista or catapult is manned.

Three masts carry a variety of square and triangular sails, and a crows nest tops the center mast. An intricate webbing of rigging allows the ship's rigger's to control the trim of the sails. The sails allow these vessels to travel as fast as \mets{12}{hour}{4.5}{7.3}.

Mounted at the port and starboard rails at center ship, two to four skiffs serve as lifeboats in emergencies.
\subsection{Maraujo}
These are the vessels inhabited by groups of \indy[marine]{Marines}. Similar to \indy{Metioujos}, these ships are usually narrower, lighter and faster than their mercantile counterparts. Like the \indy{Kurujo}, these vessels can travel as fast as \mets{15}{hour}{5.7}{9.1}. Much of what would be cargo space in a Metioujo is used in the Maraujo for supply storage, ammunition storage and dolphin tanks.

Usually sporting two or three large ship to ship weapons on the deck, Maraujos are rigged for speed and maneuverability. Two attack boats can be lowered from their stowage position near the rear of the ship. Boarding ramps are hinged into the port and starboard decking.
A 30' long and 10' wide tank at they very keel allows the ships compliment of dolphins to ride within the maraujo. A waterlock with two interlocked doors, slightly left of the keel, allows the dolphins access to the ocean. A third safety door can be lowered from ropes at three places on the ship, protecting the ship against sinking should the waterlock be damaged. Closing this safety door also protects the ship against boarders during a battle.

Crews of these vessels are proud, well trained warriors whom are honer bound to protect their crew mates, and their ship. Most marines will scuttle their Maraujo rather than let it fall into enemy hands. More information about these marines can be found in \chref{marines-for-hire} on page \tcpage{marines-for-hire}.
\subsection{Onivero Skim Boats}
Built and crewed by the \indy{Onivero}, these small boats are very unique. Only about \measure{50 feet long} and \measure{10 feet wide}, these boats have four masts using over one hundred and eighty sails. At first, the rigging seams inhabited by hoards of small sea creatures, but on closer examination it appears that the creatures ARE the rigging! The Onivero communicate with them telepathically, and together they control the skim boat. One or more groups of dolphins also travel with each skim boat.

Two hydrofoils are attached to the hull, a few feet under the waterline at each side. When the skim boat is in the open water, the dolphins form a bow wave, pulling the ship,
and the sails delicately take best advantage of each gust of wind, speeding the skim boat forward. At a crucial speed, the the boat leaps above the water, rising on its hydrofoils, and increases speed to \mets{150}{hour}{68}{110} Once skimming, the dolphins no longer need to pull it forward, and simply trail the boat, catching up to it as they can.

The Onivero will rarely allow others on board, as they have little space, and little patience for humans.
\subsection{Platforms}
Land is a rare commodity. With the largest and most farmable isles being well populated in recent times, many people have turned to living off of some of the smaller
and less desirable lands. In the last century, a new kind of migratory farmer has emerged. Rather than being tied to one plot, they have taken their entire household, established it on a large barge like raft, and moves from isle to isle. Tending different crops with different growing seasons, they maximize their ability to produce foodstuffs.

At first small groups of these farmers banded together, lashing their barges to each other, and traveling, en masse, from one site to another. As these groups grew, they
started needing specialized services, people to build and service tools, people to process the raw crops, merchants and traders to sell the results to others. Eventually these grew into entire towns and cities. Today many of these cities lay claim to a number of home sites, traveling from one to another as the growing season progresses.
\section{Building and Buying Ships}
\subsection{Design}
Before the first beam can be laid, a complete plan of the craft must be made. Shipwrights have the knowledge to create such plans and direct the construction. The average fee
for hiring a shipwright is approximately \result{10x} times their \indy{Ship Building} skill in silver pieces per day. The size of the ship will determine the time needed to draw the designs.

\begin{normboxc}[Ship Design Time]
\small
\begin{tabular}{@{} l l}
\textbf{Ship type} & \textbf{Days to Design}\\
\midrule

Sail Board & 1\\
Skiff & 5\\
Attack Boat & 8\\
Kurujo & 14\\
Metioujo & 30\\
Maraujo & 45\\
\end{tabular}
\end{normboxc}
\subsection{Dry Dock Fees}
With the initial design complete, the next step is to
rent drydock space, and hire the workers to begin construction of the ship's hull. Drydocks are usually owned by the municipality where the construction is to take place. Most drydocks must be scheduled six months to two years in advance of its usage. Penalties are levied on construction time overruns, as others ahead on the schedule must delay their construction. Drydocks are located in the prime docking areas, are reasonably expensive to build, and take a certain staff to maintain. All these factors go into their rather steep rental fees.

Sail boards are usually made in the shipwright's workspace, and do not require drydock fees. Skiffs and attack boats are made in smaller drydocks, usually requiring a fee of \tcdefine{40 silver} per day. Metioujos and Maraujos are made in full size drydocks, and require a fee of \tcdefine{400 silver} per day of construction.
\subsection{Workers}
Long experience has taught the shipwright the proper number of workers to accomplish their task most expediently. Less workers slows down the job, while more will simply get in each other's way. Ship workers have long since had their wages fixed at \tcdefine{25 silver} pieces per day. Adventurers with a \indy{Ship Building} skill of at least \tcdefine{rank 7} can replace these workers, bringing down the total ship cost by volunteering their labor.

\begin{normboxc}[Shipbuilding Labor Costs]
\small
\begin{tabular}{@{}l l l l}
\textbf{Hull Style} & \textbf{\makecell{Number of\\Workers}} & \textbf{\makecell{Days to\\Complete}} & \textbf{\makecell{Average Total\\Labor Cost}}\\

\midrule
Sail Board & 1 & 2 & 50\\
Skiff & 4 & 8 & 800\\
Attack Boat & 10 & 12 & 3,000\\
Kurujo & 14 & 40 & 14,000\\
Metioujo & 30 & 80 & 60,000\\
Maraujo & 40 & 120 & 120,000\\
\end{tabular}
\end{normboxc}
\subsection{The Hull}
The first element of the ship to choose is its hull. Hulls can be built in a variety of different sizes and styles. The quality of the construction material also will effect the cost of this phase.

\begin{normboxc}[Hull Costs]
\small
\begin{tabular}{@{}llll}
\textbf{Hull Style} & \textbf{\makecell{Days to\\Build}} & \textbf{\makecell{Costs of\\ Materials}} & \textbf{\makecell{Cargo\\Tonnage}}\\
\midrule
Sail Board & 1 & 200 & none\\
Skiff & 6 & 500 & none\\
Attack Boat & 8 & 2,000 & none\\
Kurujo & 32 & 30,000 & 150\\
Metioujo & 60 & 120,000 & 500\\
Maraujo & 90 & 150,000 & 200\\
\end{tabular}
\end{normboxc}

Modifications to the basic hull will effect the price. Adding copper sheathing reduces the wear and maintenance, and affords some additional protection against weapons, but increases the price by a factor of \result{2x}. The cost of maintaining such a ship is \result{1/3x} the normal cost.

Extra internal reinforcements can be placed within the hull to strengthen it. This makes it better able to withstand weapon fire and stressed placed on the hull from sandbars, storms and bad piloting. Hull bracing adds \result{1/2x} of the hulls original cost, and reduces its cargo space by \result{1/3x}.
\subsection{Masts}
Masts must be made from a hard wood like oak or walnut. The trees for these are specially cultivated over a period of years. The trees are bound with growing frames and
protected from insects and animals. Then they are cut, stripped of bark and planed to smoothness. Circular bands of iron are bound around the wood every few feet to increase its
resistance to bending and cracking. The wood is then varnished with several different layers to protect it from the water, wind and sun.

Then, the potential mast is fitted with the metal fixtures for mounting booms and stays. A metal cap which must be custom fitted to the deck and supports of the target ship is placed on the end. Fitting and initial rigging is then done at drydock.

A sailmaker is called in after the initial design is complete and he and his staff begin the task of preparing sails for the vessel. Generally, two identical custom sets are created, one to fit and a second for repairs. When the masts are up, the sailmaster works with the rigging crew to fit and retailor the original sails. The times and costs below include the preporation, placement, rigging and fitting of sails.

\begin{normboxc}[Shipbuilding Sails]
\small
\begin{tabular}{@{}lll}
\textbf{Hull Style} & \textbf{\makecell{Days to\\Fit Mast}} & \textbf{\makecell{Costs of\\Materials}}\\
\midrule
Sail Board & 1/5 & 100\\
Skiff & 3 & 500\\
Attack Boat & 0 & 0\\
Kurujo & 10 & 10,000\\
Metioujo & 25 & 40,000\\
Maraujo & 30 & 50,000\\
\end{tabular}
\end{normboxc}

\subsection{Rowing Deck}
The installation of rowing decks on large ships allows them the flexibility of moving under power in windless or other adverse conditions. This also causes a corresponding loss in cargo space or living quarters. Only Metioujo and Maraujo hulls have sufficient space for rowing decks. Each of the two possible decks cost an additional \tcdefine{25,000 silver} and reduces the cargo space by \tcdefine{100 tons}.
\subsection{Weapons}
Most weapons on a ship are deck mounted engines which project missiles of different types. These weapons vary in size (tonnage), damage inflicted, rate of fire and range. Each weapon is designed for a particular ammunition and can not be used with a different ammunition unless specifically noted.

A \indy{Ballista} is a device which projects large wooden bolts at high velocities. These bolts are of short range, since the must stay fairly level in flight and hit point first. The bolts generally have fins of stiff cloth or leather to help stabilize them in flight.

A ballista is generally \measure{8 to 10 feet in length and breadth}. A non-torsion ballista consists of a wooden track where the bolt is placed, a bow at right angles to the track which propels the bolt, a bowstring connected to the ends of the bow, and a trigger which holds the bolt and bowstring until fired. Other ballistas, called torsion ballistas, employ two arms connected to a box containing wound sinew or hair to propel the bolt instead of a bow. Both styles of ballistas are mounted on a swivel base for easy targeting.

A \indy{Catapult} throws large (\measure{5–10lb}) stones at high velocities. Because these stones cause damage just by impacting, they can be thrown in optimal arcs, allowing them to be used at longer ranges. Also mounted on swivel bases,
this weapon has a long throwing arm which has has a large weight at on end providing the propelling force. An attached winch is used to raise the weight to prepare the catapult for firing.

A \indy{Flamer} is a weapon developed for use on a ship carrying an experienced fire mage. Mounted on a swivel base is a \measure{6 foot long} tube of iron which starts about two feet thick and narrows to an aperture of two inches. The walls of the tube are \measure{3 inches thick}, and polished smooth. The large end has a small opening \measure{1 inch round}. Inside is a delicate mechanism which covers the opening with a plate of \indy{adamantine} at the slightest temperature rise.

The Engineer operating the device aims it at an appropriate target. Then the fire mage casts a Fireball spell through the small aperture, which closes immediately after from the heat of the spell passing through. The fireball explodes within the weapon, and all of its force and heat get channeled into a \measure{2 inch} stream projecting from the weapon's mouth. This stream expands to about \measure{1 foot} wide at its maximum range of \measure{60 feet}. The weapon must then be immediately doused with water before it can be used again. This short range weapon is very deadly to its targets, delivering both a very forceful concussion and a very damaging heat and fire stream. 

An \indy{Onager} is a catapult like device with one throwing arm which is powered by a twisted spring of sinew and hair, similar to a torsion ballista. The ability of this device to deliver large targets to great distances makes it a good choice for larger ships. Its solid framework is braced directly to the ship's deck, and it is aimed by turning the ship. Normally throwing large (\measure{20-30lb}) stones, it can also be used
to deliver other ammunition. Some favorites include fire bombs, made of bales of burning oil soaked hay, glass jars filled with poisonous snakes, and small \indy{Terisium} pellets embedded in an iron sphere to use to target different kinds of magic.

An \indy{Acceler} is a \measure{6 foot long} and \measure{1 foot wide tube} of nickel or other non-ferrous metal. The inner surface of the tube is covered with a tightly wound spiral of copper draw into a thin wire. The engineer opens an access panel on the read end and dumps in an amount of metallic shot. Closing the panel, a mage casts a \indy{Lightning Bolt} spell at the copper terminal at the rear end of the machine. The power of the lightning flows up the tube, attracting the jagged shot, and finally gives its charge to the now quickly moving
ammunition. The shot travels to its target, wildly spinning,
physically tearing and damaging what it hits, as well as releasing a portion of the energy used to propel it. Human beings in the way are generally torn to bits.
\vspace{3pt}
\setlength{\columnsep}{-13pt}
\begin{multicols*}{2}
\begin{normbox}[\# Weapon Mounts]
\begin{tabular}{@{} ll}
\textbf{\makecell{Hull Style}} & \textbf{\makecell{Available\\Weapon\\ Mounts}}\\
\midrule
Sail Board & 0\\
Skiff & 0\\
Attack Boat & 0\\
Kurujo & 1\\
Metioujo & 2\\
Maraujo & 4\\
\end{tabular}
\end{normbox}
\begin{normbox}[Ship Weapon Costs]
\small
\begin{tabular}{@{} l l l}
\textbf{Weapon} & \textbf{\makecell{Days to\\ Mount}} & \textbf{Cost}\\
\midrule
Ballista &  & \\
\quad Non-torsion  & 2,000 & 4\\
\quad Torsion & 2,000 & 4\\
Catapult & 3,000 & 5\\
Flamer & 8,000 & 6\\
Onager & 10,000 & 6\\
Acceler & 15,000 & 8\\
\end{tabular}
\end{normbox}
\end{multicols*}

\subsection{Auxiliary Craft}
Larger vessels usually mount small boats to act as lifeboats, and to give access to coasts where there is no dock. \indy{Kurujo} generally carry one \indy{skiff}, while \indy{Metioujo} carry two. A \indy{Maraujo} will normally carry two large \indy{attack boats}. Doubling capacity during an emergency, each skiff can carry \tcdefine{16 people} to safety, while an attack boat could potentially carry \tcdefine{50 people} in a smooth sea. These boats are mounted near the rear of the ship, on either side, a few feet above the waterline. A canvas chute runs from the deck down into the boat, and the rigging allows it to be dropped into the water with a single pull. A well drilled Maraujo crew can assemble, board and cast off in less than a minute.
\subsection{Defense Value}
Any constructed ship needs to have assigned to it a \indy{Artillery Defense Value} or \ADV for short to determine how well it can void enemy fire. To do this follow each of these steps, and then record the ship's \ADV.
\subsubsection{Mobility}
If your ship is operational, and is not fettered or restricted from moving, start with an \ADV of \tcdefine{+3}.
\subsubsection{Maneuverability}
If your helm is manned, and riggers or rowers in place, each ship then adds to this \ADV according to the maneuverability of that hull.

\begin{normboxc}[Ship Maneuvering Defense]
\small
\begin{tabular}{@{} l l}
\textbf{Ship} & \textbf{\makecell{Maneuvering\\bonus}}\\
\midrule
Sail Board & 6\\
Skiff (rowed) & 4\\
Skiff (sailed)  & 3\\
Attack Boat & 4\\
Kurujo & 2\\
Metioujo & 1\\
Maraujo & 2\\
\end{tabular}
\end{normboxc}

\subsubsection{Hull Condition}
A fully undamaged hull counts as an additional \result{+6} to that ship's \ADV. Using the ships \DP total as guide to its condition, this gets reduced when the ship loses \DP. \textit{A ship starting with 80 DP which has been reduced to 42 DP gets (6 x (42 / 80)) rounded down to 3 to be added to its ADV.}
\subsubsection{Hull Reinforcements}
A reinforced hull has extra bracing to strengthen the structure of the ship's hull, making it more able to withstand impact damage. If your ship's hull is reinforced, add \result{+3} to your ship's \ADV.
\subsubsection{Armor}
Copper sheathing increase your ship's defense. If your ship's hull is sheathed in \indy{copper}, add \result{+1} to your ship's \ADV.
\subsection{Initial Supplies}
When the ship is constructed, it needs to be stocked with the supplies and equipment needed by its crew. On the average, for each crewmember, \tcdefine{200 sp} must be spent for this initial equipment. This does not include any consumables like food or lamp oil or replacement equipment.
\subsection{Construction Costs}
Construction costs on a new ship are the sum of all
the various steps. For example, if you wish to construct a
maraujo, it might cost out like this:

\example{Initial design assuming a shipright with shipbuilding at rank 15 would take 45 days at 150 sp/day = 6,750 sp}

\example{Drydock fees would cost 400 sp/day and construction would take 120 days for a total drydock cost of 48,000 sp.}

\example{A maraujo takes 40 workers being paid 25 sp per day and working for 120 days. This totals to 120,000 sp. The hull cost for a maraujo is 150,000 sp.}

\example{The mast costs for a maraujo is 50,000 sp.}

\example{If we decide to have one rowing deck, we add an additional 25,000 sp.}

\example{Four weapons, 2 balistas, a flamer and an acceler are to be mounted on our maraujo:}

\begin{tabular}{@{}l l l}
\example{2x} & \example{Ballista 2,000} & \example{=  4,000 sp}\\
\example{1x} & \example{Flamer 8,000} & \example{=  8,000 sp}\\
\example{1x} & \example{Acceler 15,000} & \example{=  15,000 sp}\\
\midrule
 & \example{Total} & \example{=  27,000 sp}\\
\end{tabular}

\example{We need two attack boats. These also need to have all their steps summed, but we get to save by making both attack boats to the same design.}

\begin{tabular}{@{}lllll}
\example{Design:} & \example{8} & \example{x 150 sp/day} &  & \example{ = 1,200 sp}\\
\example{Drydock:} & \example{2} & \example{x 10 days} & \example{x 40 sp/day} & \example{= 800 sp}\\
\example{Workers:} & \example{2} & \example{x 10 wrks} & \example{x 25 sp/day} & \example{= 500 sp}\\
\example{Hulls:} & \example{2} &  & \example{ x 2,000 sp} & \example{= 4,000 sp}\\
\midrule
\example{Total} &  &  &  & \example{ = 6,500 sp}\\
\end{tabular}

\example{And lastly, the original equipment for the maraujo. This includes any non-expendable equipment and supplies. Assume a crew of 80, at a cost of 200 sp per crew member,
this totals to 16,000 sp.}

\example{So, summing all the various costs reveals the cost of building a new maraujo:}

\begin{tabular}{@{}r l}
\example{48,000 sp} & \example{Ship Design}\\
\example{120,000 sp} & \example{Labor}\\
\example{150,000 sp} & \example{Hull Cost}\\
\example{50,000 sp} & \example{Mast Costs}\\
\example{25,000 sp} & \example{Rowing Deck}\\
\example{27,000 sp} & \example{Weapons}\\
\example{6,500 sp} & \example{Attack boats}\\
\example{16,000 sp} & \example{Initial Equipment}\\
\example{442,500 sp} & \example{Total construction cost}\\
\end{tabular}

We will have to remember to figure the expendable costs of food, ammunition and replacement equipment when we want to go and actually use this ship.
\subsection{Used Ships}
Commissioning and building a new ship is both costly and time consuming. A much better option for many is purchasing an already existing vessel. Since the usable lifetime of most ships ranges from \tcdefine{8 to 50 years}, the quality and price will be largely determined by the age of the vessel, and the current availability of ships of its type. Prices vary from \result{20\%} to \result{80\%} of the original construction price. Initial repairs for a newly bought used ship can cost up to \result{30\%} of the vessel's original construction price.
\subsection{Running Costs}
Supplies must be periodically replaced. Sails last only a year or two. Onboard supplies of repair materials are consumed. Broken tools and damaged weapons must be serviced or replaced. In general, it costs \result{1\%} of the original construction cost \tcdefine{per month} to maintain the condition of a ship.
\subsection{Cargo Profits}
\begin{wrapfigure}{l}[0pt]{110pt}
\begin{normbox}[Cargo Prices]
\small
\begin{tabular}{@{} l l}
\textbf{Cargo} & \textbf{\makecell{Price/\\Ton}}\\
\midrule
Ale & 1200\\
Cloth & 2000\\
Fish & 1600\\
Grain & 1000\\
Livestock & 1600\\
Lumber & 5000\\
Oil & 1800\\
Spices & 3000\\
\end{tabular}
\end{normbox}
\end{wrapfigure}

While operating a merchant vessel can be quite lucrative, much is dependent on the business sense of the owner and the skill of the ship's officers in acquiring and disposing of the proper cargoes at the correct times. Most common bulk cargos are grains, foodstuffs, ale, lumber, livestock, prisoners, and manufactured items. While the costs of these cargoes will vary according to demand and the negotiating skills of the trader, here are the average selling price, per ton, of cargoes entering Rougtero at this time.

Transporting prisoners usually requires one ton of space per prisoner transported. Since the selling price of prisoners varies so wildly based on demand and prisoner skill set, profitability depends more on the selling skills of the merchant.
\section{Maintaining and Operating a Ship}
\subsection{Navigation}
Any crew should include at least one sailor, preferably two, with a well developed skill at navigation. Sailing from port to visible destination takes no navigation check. Sailing to a different port on the same island requires a \tcdieroll{1d6} check vs indy{navigation}. Sailing across the ocean to a nearby island requires a \tcdieroll{2d6} check, and to a far away island, a \tcdieroll{3d6} check.
\subsection{Porting}
Any port city will charge per day fees for docked ships. These fees pay the salaries of the port cargo handlers, pay for the harbormaster's office, and for periodic dredging of the docks to allow large ships to dock. These fees are based on hull size and run about \tcdefine{100 sp} per day for the largest vessels.
\subsection{Repairs}
Major repairs to a vessel's hull require it to be drydocked. Drydocking fees (like those charged during building) are applicable, as well as a \tcdefine{2000 silver} fee to pull large ships into drydock, mounting them in a work frame. The number of laborers and materials needed to accomplish repair work is dependent on the severity of the damage.
\subsection{Crew Management}
While a vessel's captain or \indy{cefo} is the ultimate responsibility for all things, the hiring, firing and management of the crew is usually left to the ship's first trader. Often on a merchant vessel, this will be the ship's owner as well. The first trader handles payment at each port of call, and checks on the status and performance of each crew member. If there is a difficulty, the first trader collects information and then presents it to the cefo for any needed disciplinary actions.
\subsection{Crew Abilities}
As a whole, the crew of a ship has been trained to work together to sail and fight for their ship. The quality of a crew's ability to fight in ship to ship combat is represented by
the crews \indy{Artillery Modifier} (\AM). The GM will assign this number to any crew dependent on the skills of the individual members of the crew, their length of service together, and their past combats. The person operating the ship has the responsibility to track the ship's Artillery Mod.
\section{Combat at Sea}
Most modern warfare takes place on Jaern's oceans. Land is generally considered too valued by any participants in a dispute to risk its ruination during combat. Nations and city
states have in the past fielded large fleets of ships to protect their land and expand their interests. Supporting and maintaining theses flees proved a large expense to these resource poor nations. Over time this forced other alternatives to appear.

In the early 79th century, the emergence of the independent maraujos as the major maritime force have helped make combat more personal. Each maraujo is an independent force of \tcdefine{30 to 200 marine warriors} whom contract their services to nations, guilds, and individuals. For small towns and villages, employing a maraujo when needed is a much more economic way to provide for defense needs. Larger cities and nations will often negotiate long term contracts for one or more maraujos for defense, or hire a large number when they wish to engage in war. Merchants and guilds often will hire these maraujos to protect and guard shipments of goods. All have learned to rely on the honor of these marines to fulfill their contractual obligations.
This codifying of warfare has resulted in a personalization of combat. Large conflicts of fleet versus fleet are much rarer than two vessels facing of against each other. Other players in this gamer are armed merchants, lawless privateers, and the Onivero. The preponderance of those who would take what they wish from others has caused even the smallest merchant to consider arming their vessels. And with the number of captured merchant vessels used as privateer craft, today's sailor will find the seas of Jaern a very dangerous place.
dangerous place.
\subsection{Ship to Ship Combat}
When two or more ships decide to enter combat, it is handled much like combat between two individuals. Your GM will determine the distance between the involved ships, and usually draw a map, or setup a table with models to show the position of each combatant. Few spells and no ship to ship weapons exceed \measure{1000 feet} in range, so this is generally the
largest range set for most combats.

Generally, one of the combatants starts by performing a combat action. A weapon shot, a spell being cast, a course change or a shouted challenge. Any action which clearly indicates the start of a combat is considered a surprise round, and those on the instigator's ship are the
only allowed to take an action during this round.

Rounds in ship combat are the same length as hand to hand combat, \tcdefine{4 seconds} each. After the surprise round is resolved, each following round starts with an initiative check
to see which ship acts first. Like normal combat, a representative from each ship rolls \tcdieroll{2d6}, and the results determine the order, lowest to highest, in which actions are
taken. The same rules on ties and cumulative bonuses also apply here.

The same phases of combat, \indy{Informational Questions}, \indy{Action Preparation}, \indy{Statement of Actions}, \indy{Results of Actions} and \indy{Outcome Phase}, are used in ship battles. In general, if there is any uncovered questions about ship combat, treat it like individual combat. To reference those rules, consult \chref{encounter} on \tcpage{encounter}.
\subsubsection{Moving the Ship}
Ships will find it necessary to maneuver during combat. Each ship has a movement rate that states the distance it can move, each round. This distance is then modified by your GM according according to wind strength and direction, ship conditions, and crew status.

Every ship can turn as well. The calling player calls out the turn direction, and amount. The GM may also modify the turn angle considering the conditions on the ship.

\begin{normboxc}[Ship Movement During Combat]
\small
\begin{tabular}{@{} l l l}
\textbf{Ship} & \textbf{\makecell{Maximum\\Feet}} & \textbf{\makecell{Turn Angle\\(degrees)}}\\
\midrule
Sail Board & 30 & 60\\
Skiff (rowed) & 10 & 45\\
Skiff (sailed) & 30 & 30\\
Dolphin/rider & 120 & 180\\
Attack Boat & 30 & 45\\
Kurujo & 50 & 20\\
Metioujo & 40 & 15\\
Maraujo & 50 & 20\\
\end{tabular}
\end{normboxc}

Given they are properly staffed, vessels under sail require \tcdefine{10 rounds} to go from a full stop to their maximum movement rate, if the wind is available and the crew ready. A
rowed vessel can come to speed in \tcdefine{3 rounds}. Stopping times are the same. \example{A Maraujo, for example, in the first round after the cefo has given the raise sails call, will accelerate from a stop to 5 feet per round, increasing its speed by 5 feet per round for ten rounds, until it is traveling 50 feet per round.}
\subsubsection{Firing Weapons}
Shipboard weapons are handled much the same way as missile weapons in normal melee combat. A ship may fire any of its weapons at and target in its line of site, and its
firing arc (weapons can't fire through the ships own rigging). After a missile fires, it takes a fixed number of rounds to reload. \example{Thus a catapult can be fired once every 4 rounds (16 seconds), the weapon is fired, and then the throwing arm is pulled back and a new stone is placed in the cup.}

TODO TABLE 

When the ship fires its weapon, roll \tcdieroll{1d20}, \result{add} the firing crew's \result{artillery mod (\AM)} and compare the result to the defense value of the target ship. If the result is equal to or higher than the defense value (\ADV), the shot succeeds in striking the target. The engineer manning the weapon can apply one option of their personal artillery skill if they wish.


\subsubsection{Critical Hits and Misses}
When a ship attempts to fire one of its weapons, examine the result of the attack roll before any bonuses or mods are added. If the die roll is \tcdefine{a 1}, it is an \indx{melee!automatic miss}\indy{automatic miss}, no hit happens, no damage is done. If the die roll is \tcdefine{a 20}, it is considered a \indx{melee!critical hit}\indy{Critical Hit}. The GM will ask the engineer (or will roll if the engineer is not a player) to roll percentiles (\tcdieroll{2d10} with one die specified as the tens' digit and one die as the ones' digit) to determine its severity, and cross reference the appropriate table for your attack type in \apref{critical-hits-artillery} on \tcpage{critical-hits-artillery}. The GM  will ignore any results that make no sense for the target vessel and have the engineer reroll until they get an appropriate result. If a ship gets a \result{sinks immediately} result, all hands aboard are killed.
\subsubsection{Individual Missiles}
Standards bows and crossbows hold little danger
f or the stru cture an d e qui pment o n an enemy ship. Th e
constantly rolling deck, and movement of both bowman and
target render conventional techniques for shooting virtually
useless. However, a hail of missile fire can cause opponents
to seek cover, an d hamper their ability to fight. And the
occasional arrow or quarrel may kill or crewman, or foul
rigging, or jam a weapon.
W h e n a n a d v e n t u r e r ( o r a n y n o n - o c c u p i e d
crewmen) fires a bow or crossbow, roll 1d20. On a 20 such a
random hit occurs, doing normal damage for that weapon to
the target ship (repr ese ntin g lo ss o f crew or d am ag e t o
rigging). The normal time must elapse for reloading these
weapons before they may be used again.
\subsubsection{Individual Spells}
Area effect spells which cause damage can be used
o n an en em y shi p if all the pro pe r r an ge and target ing
requirements are met. Fireballs and other fire based damage
spells do fire damage to the vessel, while lightning bolt and
a ny percusive s pells do impac t damage . The amount of
damage done to the ship is the same as if they were used
against a human target.
\subsubsection{Boarding Actions}
If two vessels come within 15 feet of each other,
they may drop boarding ramps to attempt to board the other
vessel. Dropping these takes one round, and locks the two
vessels together. When this happens, the GM will diagram the
two ships and place the actors and adventurers in appropriate
or random places. Combat continues as before, but the GM
will then be asking everyone for actions, not just the player
controling the vessel.
\subsection{Sink and Burn}
\subsubsection{Damage Points}
Each vessel, like each adventurer, has a damage
point total. This total represents a combination of the physical
condition of the vessel's hull, the state of its rigging and sails,
and the condition of the crew manning the vessel. As damage
is done to the vessel, the number of living crew, the ship's
ability to defend itself (its ADV), its ability to remain afloat
and its maximum movement rate are all effected.
The initial number of da mage points for a non-
damaged completed vessel is based on its hull size.

TODO TABLE 

Each ti m e a v e s s e l i s dam a g e d , e ach o f th e
components making up its DP are changed. If you express the
damage done to the vessel as a fraction, with the current DP
on top and the initial DP on the bottom, this fraction is
multiplied with each component.
If a Kurujo with 18 crew members starts with 300
DP and is currently at 150 DP. It originally had a value of 2
a dd ed int o i ts ADV for its h ull, b ut n ow th is bec omes
150/300 * 2 or 1, so its ADV decreases by 1.
With 18 original crew members, the Kurujo now
has only 9 remaining. Normally, the GM would give any
adventurers on board the target ship a 150/300 chance of
having been struck and killed in artillary fire. The GM will
assign the roll of a particular size die, and announce what is
needed to survive, and then let the player make the roll. The
GM may, at his option, decide to handle this in more detail,
taking into account the adventurer's position and actions.
This ship would normally move up to 50 feet in one
round. In its damaged condition, it now has a ma ximum
movement of 150/300 x 50 = 25 feet per round.
\subsubsection{Impact Damage}
When a weapon strikes a ship, the player or GM
directing the firing ship rolls a d i e to generate a random
amount of damage from one to the listed damage for the
weapon which h as fi red. If the weapo n damage do es not
exactly fit the size of a die, choose the next biggest die, and
ignore any rolls above the maximum. Never use multiple dice
to make this roll, as this changes the resulting distribution of
results!.
The impact damage indicated by this roll is then
immediatly subtracted from the ship's current DP total. This
represents a hole in the hull, or structural damage, or broken
masts and spars.
\subsubsection{Fire Damage}
A flaming weapon can do more t han just c ause
impact damage. When a weapon has flame damage listed, and
a strike with such a weapon succeeds, part of the target vessel
is set afire. This has no immediate effect, but adds to the
ship's fire damage, its potential damage done to the vessel
from fire. Each round a ship is in flames, its player rolls one
die to generate a random amount of damage from one to the
current fire damage and subtracts the total from the ship's
current DP. Fire fighting by the crew or via spells can lower a
ship's fire damage value.
\subsubsection{Combat Repairs}
In the heigth of combat, repairs are rarely able to be
executed quickly enough to make a large difference. But in
the case where a particular part of the ship ceases functioning
du e to damage, an e ng in eer c an attem pt t o jurr y ri g a
replacement. For example, if a critical hit takes out the helm,
an engin ee r can attempt to rig a temporary replacement
locating the ropes leading to the destroyed wheel, positioning
men to pull them and shouting instructions.
To succeed, the engineer must have two rounds to
give instructions to his assistance, spend two rounds as they
place themselves, and then make a check against his repair
skill. The GM will determine the dificulty of the repair, in this
c a s e i t w ou l d b e 3 d6 vers us re pai r to k e e p the hel m
op era tio na l un til a f te r th e ba ttl e is over. Th i s appro ach
requires the proper number of engineers for the particular
r epair , and the proper materials to be at hand. Only on e
engineer can attempt any one repair at one time.
\subsubsection{Fire Fighting}
During combat, the ship's riggers and unengaged
dolph ineers provid e one imp ortant f un cti on. They lo we r
buckets over the side, and bring up water to throw on fires.
F r o m t h e ro u nd f o ll owi ng th e on e wh e re a ri g g er or
dolphineer begins to fire fight, they take one point of fire
damage off of their ship's fire damage each round. Marines
occupied fighting fires can not perform other duties.
Mag i c ia ns c an a lso p la y a n im po rtan t r ol e i n
abating the burning of their ship. Casting an appropriate spell
to quench the fires will lower the ship's fire damage by one
for each rank of the spell. This reduction happens each round
for the spells duration. These spells include Akvovoki, Change
Temperature, Con den se , Create Wa t er, Dowse, Elemental
Mastery, Extinguish, Ice Ball, Quench, Torrent, and Water
Stream.
\subsubsection{Sinking}
Whe n a v ess el is d amage d, there is always the
chance that it may sink. When small vessels sink, they leave
those carried floating on the surface of the sea, at the mercy of
the waves and any passing sea creatures, or enemy warriors.
A l arger sinking vessel, anything l arger than an
attack boat, creates a suction which pulls anything on the
ship , and near the ship, int o the wa ter and down to the
bottom. Collapsing decks, sheared timbers, inrushing walls of
water, upended rooms and heavy furniture combine to make
survival during sinking unlikely. Even if the unfortunate crew
member was not immediatly crushed, and somehow managed
to work free of the vessel, the suction of the ships passing
would pull him to his death at the bottom of the sea. Being in
such a ship during the round it sinks, results in death.
Any time a ship's DP total is below 20 as the result
of a hit or fire damage, the player of the ship rolls a d20 and
must get the ship's DP total or less to keep the ship from
sinking that round. Any adventurers on a small ship which
sinks are left adrift in the water, and must stay afloat to
survive.
Any adventurers on a large ship when it sinks get
one round of action at the time the ship sinks. If they are not
off the ship and at least 60 feet away at the end of their action,
they go down with the ship and die. Lizards, able to breath
water, get a second round to attempt to withdraw, but even
they will be crushed and killed if they can not escape within
two rounds.
\end{multicols*}
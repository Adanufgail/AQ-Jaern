\chapter{Life on the High Seas}
\label{ch:high-seas}
A very mature planet, the land masses of Jaern have been slowly eroded away, leaving most of its surface covered in a relatively shallow ocean. Thousands of small islands
poke above the surface, leading to a proliferation of very diverse biologies and cultures. The ocean itself has a very low salt content and is home to the majority of Jaernian life. Humans and their kin ply the surface of the ocean with a plethora of different vessels, traveling from island to island, or even living at sea.
\setlength{\columnsep}{\defcolwidth}
\begin{multicols*}{2}
\section{Ocean Creatures}
\subsection{Fish}
The variety of fish swimming the seas of Jaern
seems endless. Cold blooded, finned, spined, colored, poisonous and not, there is no end to their sizes shapes and appearances. Fish form the diet staple of most people, and
employment for over one fifth of all humanoids in their harvesting. Dangerous fish include vicious \textbf{sarko}, and the very poisonous \textbf{quezl}.
\subsection{Coelenterates}
Almost as numerous as all the other life in the seas, Jaern supports a bewildering variety of these creatures. Small \textbf{Atomo}, each less than a hundredth of an inch in size, form the food for most of the grazing fish . Larger jellyfish in myriads of colors float in the top twenty feet of the water,
slowly converting light to food by different chemical  processes. The largest of these creatures, the \textbf{Flugofiso} can grow up to 60 feet in radius, and generate gases that allow it
to escape the waters and float high in the air. Many superstitions surround the sighting of these strange coelenterates. 
\subsection{Dolphins}
These creatures have evolved into highly intelligent creatures. Because of their environment and lack of manipulative limbs they are not tool builders, however they have developed a diverse and complex culture.

Physically they are very similar to Earth's dolphins, they differ in a few important ways. A dorsal stripe of color, blue for females, and yellow for males, assists in their communications by changing shade subtly, indicating general mood and inclinations. A slightly larger skull is protected by thicker bone and fleshy bumps which protect the dolphin's brain from blows and the extreme temperature found in ocean
currents. A larger tail fin and a more sleek body allow these creatures to move through waters at speeds approaching 40 mets per hour (15 MPH or 24 KPH). The life expectancy of these dolphins average from 30 to 40 years.

Dolphins communicate with each other in a highly developed language consisting of whistles, clicks and guttural stops. Some of these are outside the range of human (but not lizard) hearing. The re language itself is weak in complex verbs and time based tenses, but very extensive in descriptive terms describing water and the objects found in the oceans. Some humans can, after extensive training, understand and speak some of this speech, but the dolphin must make a conscious effort to slow its speaking speed.

Forming small groups of 8 to 16 based around social needs and friendship, dolphins are rarely seen alone. They will pair once, making a lifetime commitment to a mate. If they are capable, each pair will bear young live, in pairs and triples, once every 5-10 years. These young are raised by the group until they are old enough to interact with others close to their age, at which time they leave and form new groups.

The dolphins consider Jaern their world, and land
based creatures as strange aliens. Generally they do not concern themselves with events on land, however they will relentlessly hunt down those who damage or poison the seas and those whom would capture and enslave dolphins for their own ends. The dolphins will seek a way to punish these humans for these crimes. As for others living beneath the sea, dolphins live in peace with lizardkind, and while they can not ordinarily communicate well, will often be seen in lizard’s company.

Occasionally, a young dolphin will meet one outside their own race and form a friendship. If the other wins the dolphin’s trust and affection, the dolphin may decide to pair outside their race. This kind of relationship remains one of emotion and friendship. Maraujos often seek out places where many young dolphins swim, looking to find those whom would pair with its youngest marines. The dolphins and the marines, over time , have learned the benefits of such a pairing, which has resulted in the dolphins tolerating the nearby presence of these humans.

Once paired in this way, the dolphin will expect nearly daily attention from his partner. In return, he will devote his time, energies and abilities to help his partner. Over time, the two develop an almost telepathic sense of the other’s needs and desires. Many dolphins, and dolphineers have sacrificed even their own lives to help save their partner from harm.
\subsection{Lizards}
Evolved in the deeps of the oceans, these intelligent
creatures group together forming gatherings. They communicate verbally and sense objects and motion mainly via sound, which travels well under water. Finding dry land
uncomfortable, they largely avoid interacting with the humans and humanoids above the water.
\subsection{Balenoigajos}
Mammalian creatures of large size, these  herbivorous sea dwellers subsist on Atomo strained through their bodies as they swim. While it is very profitable to hunt
and slay these creatures for their meat and other body products, this profession has been virtually eliminated by the Onivero whom consider these creatures close friends and allies.
\subsection{Oorn}
Evolved from land trees, these plants grow branches along the water's surface and thrust large leafy growths upward toward the light. A single floater can reach
sizes of up to a few hundred feet in length. Posing a  navigational hazard to ships, standard shipping lanes in the southern waters that the Oorn usually grow, are regularly
swept clean of these plants. Especially large Oorn are sometimes used as the base of small villages inhabited by the Onivero, or by some nomads whom have emulated them.
\section{Oceangoing Vessels}
The seas of Jaern are plied by innumerable vessels of many different sizes. Ocean going transportation ranges from that for the single person, to entire ocean dwelling communities.
\subsection{Sail Boards}
Only suitable for short journeys between nearby islands, these boards are about 6 feet long and 2 foot wide. In the center, 2 feet from the front is a socket which excepts a 7
foot tall mast, bearing a single cross spur and a triangular sail, 6 foot wide at the base, coming to a point at the top of the mast. The user of this device stands behind the sail and holds the cross spur at the proper angle to catch the waves. They use their body to steer the board.

This very active and athletic way of travel restricts the journey time to about two to three hours for even the most ambitious athlete. At a speed of up to 10 mets/hour (4 MPH or 6 KPH), this restricts the normal range of this device to 30 mets (11 Mi or 18 KM).

Marines, trained in sailboarding, will use sailboards
as a last resort when their \textbf{Maraujo} has sunk. Carrying several days food, they can often go up to a week, traveling 20-40 mets a day (7.5 Mi or 12 KM), to attempt to find a place to land. Navigating while using a sailboard is particularly difficult
since they will only keep a course when they are held steady by their riders.
Sailboards without a mast and sail are called \textbf{Surfboards} and are often used by marines to land during amphibious assaults. A maraujo will sit just over the horizon from its target, and its dolphineers will ride these surfboards to shore to catch their target by surprise. Then the maraujo will close and disgorge attack boats to pick up the marines after the assault.
\subsection{Dolphins}
Dolphins will rarely concede to being used like steeds by humanoids. Normally only dolphineers and a few scattered nomads spend enough time at sea to have an opportunity to meet and pair with a dolphin. Even these people must treat their dolphin as a partner rather than a mount.

While riding a dolphin, the human lies above the dolphin and wraps their arms and legs around the dolphin. Since their arms will rarely reach, they will hold on to a length of leather in each hand, or tied about their wrists. Dolphineers will often have a suitable leather thong attached to their maroglave for this purpose. As the dolphin swims, the rider must match their body movements to the flexing of the dolphin as its tail and back arc up and down. A dolphin and rider traveling like
this can achieve speeds approaching 35 mets per hour (13.25 mi or 21.33km). Dolphins have good senses about currents, depths, and direction. They will be able to find land with little effort.

More barbaric and oppressive riders have been known to use a leather harness on the dolphin. Such a rider places their feet in the stirrups and attached their harness to the dolphin's. Such a harness is uncomfortable for the dolphin and will injure it in time.
\subsection{Skiffs}
These small boats are generally about ten to twelve feet long and 3 to 4 feet wide. They are generally constructed of wood planking and have a rudder operated manually at the rear. A mast with a single spur sports a 10 foot high triangular sail. A skiff can generally hold 6 to 8 people and can travel at speeds of ten mets per hour under sail, or 4 mets per hour rowed.
Skiffs are generally used for line of sight travel
between nearby islands, and as auxiliary craft aboard larger
crafts. The handle only relatively calm seas and will swamp
easily with waves larger than a couple of feet tall.
\subsection{Attack Boats}
Larger than skiffs, these boats are generally 25 to
30 feet long, 8 to 10 feet wide, and hold up to 30 men. These
boats are generally constructed of wood planking, pegged to
form, and covered in many layers of a heavy shellack to
waterseal and reduce friction. With no mast, they are driven
by six sets of oars mounted midship. At capacity, these boats
can be rowed at speeds of up to 8 mets per hour.
They are generally used by larger vessels to aid in
debarking and boarding. A Maraujo will often have up to four
of these to transport non dolphin riding warriors. Merchant
vessels will use these boats as life boats, and to land in places
not equipped with a proper dock.
\subsection{Kurujo}
U s e d m ai nl y a s cou r iers an d l ig h t c ar go an d
passengers, these vessels are generally 35 to 50 feet long and
displace 3,000 to 6,000 tons. With a crew of 10 to 20 sailors,
they can carry a few passengers, or a limited amount of cargo.
Two masts hoist large triangular sails, with a few smaller sails
for maneuverability. An unladen Kurujo can travel at speeds
of up to 15 mets per hour.
\subsection{Metioujo}
These wooden vessels are deep sea ships capable of
traveling long distances with large cargos. Used generally by
merchants, these carry the majority of traffic at sea. Metioujo
are sail driven, and are normally armed to ward off attacks by
privateers.
A metioujo’s displacement varies from 10,000 to
20,000 tons, and their length from 80 to 120 feet. Width, at
the center, is usually one third of the vessel’s length. There is
normally a large cargo deck accessible from one or more
cargo hatches on the main deck. Above the cargo deck is the
crew deck, containing quarters for the crew, storage for food
and operating equipment, and the crew’s mess. Above this is
the main deck, open to the sea in the middle. At the front of
the ship is the forecastle, housing the officer’s quarters. At the
r ear i s the aftc astl e, c ont ai nin g the ch ar t ro om a nd th e
weapons locker. Above the aftcastle is the pilot’s deck, where
the rudder wheel is manned, and the aft ballista is usually
located. Above the forecastle is the cefo’s deck, from where
the ship is normally commanded, and the fore ballista or
catapult is manned.
Three masts carry a variety of square and triangular
sails, and a crows nest tops the center mast. An intricate
webbing of rigging allows the ship’s rigger’s to control the
trim of the sails. The sails allow these vessels to traval as fast
as 12 mets per hour.
Mounted at the port and starboard rails at center
ship, two to four skiffs serve as lifeboats in emergencies.
\subsection{Maraujo}
T hese ar e the ve s s e ls in h abit e d by g ro up s o f
M ar ines. S imi la r t o me tiouj os, t hese shi ps ar e us uall y
narrower, lighter and faster than their mercantile counterparts.
Like the Kurujo, these vessels can traval as fast as 15 mets
per hour. Much of what would be cargo space in a metioujo is
used in the Maraujo for supply storage, ammunition storage
and dolphin tanks.
Usually sporting two or three large ship to ship
weapons on the deck, Maraujos are rigged for speed and
maneuverability. Two attack boats can be lowered from their
stowage position near the rear of the ship. Boarding ramps are
hinged into the port and starboard decking.
A 30’ lo ng and 10’ wide tan k at t he v ery keel
allows the ships compliment of dolphins to ride within the
maraujo. A waterlock with two interlocked doors, slightly left
of the keel, allows the dolphins access to the ocean. A third
safety door can be lowered from ropes at three places on the
ship, protecting the ship against sinking should the waterlock
be damaged. Closing this safety door also protects the ship
against boarders during a battle.
Crew s of t hese ves sels are p r oud , well tra ined
warriors whom are honer bound to protect their crewmates,
and their ship. Most marines will skuttle their maraujo rather
than let it fall into enemy hands. More information about
these marines can be found in chapter 10: Marines for Hire.
\subsection{Onivero Skim Boats}
Built and crewed by the Onivero, these small boats
are very unique. Only about 50’ long and 10’ wide, these
boats have four masts using over one hundred and eighty
sails. At first, the rigging seams inhabited by hoards of small
sea creatures, but on closer examination it appears that the
creatures ARE the rigging! The onivero communicate with
them telepathically, and together they control the skim boat.
One or more groups of dolphins also travel with each skim
boat.
Two hydrofoils are attached to the hull, a few feet
under the waterline at each side. When the skim boat is in the
open water, the dolphins form a bow wave, pulling the ship,
and the sails delicately take best advantage of each gust of
wind, speeding the skim boat forward. At a crucial speed, the
boat l ea ps abov e the water, r i sin g on i ts hydrofo ils , and
increases speed to 80 to 150 mets per hour. Once skimming,
the dolphins no longer need to pull it forward, and simply
trail the boat, catching up to it as they can.
The Onivero will rarely allow others on board, as
they have little space, and little patients for humans.
\subsection{Platforms}
Land is a rare commodity. With the largest and
most farmable isles being well popu l ated in recent times,
many people have turned to living off of some of the smaller
and less desirable lands. In the last century, a new kind of
migratory farmer has emerged. Rather than being tied to one
plot, he has taken his entire household, established it on a
large barge like raft, and moves from isle to isle. Tending
different crops with different growing seasons, he maximizes
his ability to produce foodstuffs.
A t fi r s t small gr ou ps of these farm ers ba nd ed
together, lashing their barges to each other, and traveling, en
masse, from one site to another. As these groups grew, they
sta r ted ne ed ing sp eci alized se r vices, p eople t o bu ild and
service tools, people to process the raw crops, merchants and
traders to sell the results to others. Eventually these grew into
entire towns and cities. Today many of these cities lay claim
to a number of home sites, traveling from one to another as
the growing season progresses.
\section{Building and Buying Ships}
\subsection{Design}
\subsection{Dry Dock Fees}
\subsection{Workers}
\subsection{The Hull}
\subsection{Masts}
\subsection{Rowing Deck}
\subsection{Weapons}
\subsection{Auxiliary Craft}
\subsection{Defense Value}
\subsubsection{Mobility}
\subsubsection{Maneuverability}
\subsubsection{Hull Condition}
\subsubsection{Hull Reinforcements}
\subsubsection{Armor}
\subsection{Initial Supplies}
\subsection{Construction Costs}
\subsection{Used Ships}
\subsection{Running Costs}
\subsection{Cargo Profits}
\section{Maintaining and Operating a Ship}
\subsection{Navigation}
\subsection{Porting}
\subsection{Repairs}
\subsection{Crew Management}
\subsection{Crew Abilities}
\section{Combat at Sea}
\subsection{Ship to Ship Combat}
\subsubsection{Moving the Ship}
\subsubsection{Firing Weapons}
\subsubsection{Critical Hits and Misses}
\subsubsection{Individual Missiles}
\subsubsection{Individual Spells}
\subsubsection{Boarding Actions}
\subsection{Sink and Burn}
\subsubsection{Damage Points}
\subsubsection{Impact Damage}
\subsubsection{Fire Damage}
\subsubsection{Combat Repairs}
\subsubsection{Fire Fighting}
\subsubsection{Sinking}
\end{multicols*}
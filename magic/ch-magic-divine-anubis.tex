\chapter{Anubis - Lord of the Dead}
\label{ch:divine-anubis}
\section{Domain}
Anubis is the guardian and protector of all souls, guiding them through life and into the true existence of death.
\section{History}
When man was first made by the gods, it is told that there was no death and old age was unknown. When people were injured, their bodies simply healed. Only the occasional hero or martyr would be taken by the gods and brought to Cielo, an infinitely large place of peace and beauty.
Man grew fruitful and multiplied greatly. Over the next few centuries gradually the world was filled up. Then things started to go downhill. Crowding and overpopulation caused strife, hunger, and pain. Since none could die, whole populations were held captive and forced to exist in pain in large refugee camps. The very land became sterile, and food even more scarce.
A young man, a hero who had rescued an entire nation from captivity, was rewarded by the gods by being sent to Cielo. He asked, “Why are so few granted this boon? Why do you revered gods and goddesses ignore the suffering and agony of your worshipers?” The gods told him to be still and take his reward without objection. A rather heated argument ensued, ending by sending the young man summarily to Cielo, after silencing him by changing his head into that of a jackal.
He found other heros and martyrs living in Cielo, and spent the next few years creating a language of gestures and movements to communicate with them. The others did this to satisfy their curiosity as to the origin of this unusual creature. Once he could be understood, he told them his story, and of the plight of the peoples of the world.
Enlisting the aid of history’s heroes, he stormed the gate to the real world and forced the guardian deity there to let them pass. The hoard of heros roamed the world, seeking out the most deserving to send onward to Cielo.
The gods gathered to punish this presumptious mortal who had ruined their paradise. They caused the land to open and swallow up Cielo beneath the ground, now calling it Infero. Its beauty became bleakness instead. The heros there degenerated and become mere specters of their former selves, unable to enjoy life for theirs had been taken. Their brash leader was branded Anubis, “he who destroys,” and was exiled to lead the dead within Infero.
The gods soon realized the benefits of a much smaller population. While the number of worshipers was much smaller, the resulting prosperity actually increased the gods’ powers. Having learned this lesson, they relented on Anubis, elevating him to godhood, and charging him to cull
the old and rescue the miserable from life. He was to maintain the population down at a manageable size, and was given absolute power over the dead to assist him in accomplishing this task.
\section{Motivation}
"Life is a shadow of true existence"
Believing that real existence does not start until death, the followers of Anubis hold that life is a place to train the soul and prepare it for its existence after death. It is the highest honor to be sent to Anubis via human sacrifice (Mind you, many followers don’t take this too literally.)
\section{Aspects}
Unlike the other gods, Anubis is rarely seen outside his home in the underworld. The existence of live souls about him pains him and causes him distress. He delegates the tasks that would take him above Infero to his various minions.
He usually appears as an 80 foot tall, jackal-headed human within his citadel. He sits on an enormous adamantine throne at the front of a huge chamber, four mets long by one met wide. There he grants an audience to all the newly dead, and passes judgement on them before assigning them their place in Infero. His pronouncements can be quite cruel, but Anubis is always just. While he can speak, he usually uses Tusparol, a sign language, to communicate with his priests.
\section{Structures of the Priesthood/Temple}
The priests and priestesses of Anubis’ Sepulchers perform all the holy and administrative tasks needed. Prisoners are assigned all duties involving manual labor. Non-priest freedmen are not allowed to work within the Sepulcher, as they do not understand Tusparol and therefore cannot communicate with the priests.
Priests do not specialize in one style of service. Instead, each priest holds a holy office, is responsible for a certain administrative duty, and must participate in the religious discussions of the Morto-Vojo, their holy books.
\subsection{Organization}
Priests are organized within the Sepulcher according to seniority and merit. Each priest is responsible for certain ceremonies or aspects of worship. These offices are arranged in a figurative tree structure, with the high priest at the top. Offices are only vacated by the death or advancement of a priest. If a priest performed so badly as to be demoted, they are expected to atone by offering themselves for sacrifice. Usually a group of about six elder priests hold themselves outside the normal assignments within the Sepulcher. These priests, called “Recenzisto,” are in charge of internal advancement and justice. They are responsible to the high priest, but in turn are in charge of appointing them.
\subsection{Requirements}
Priests of Anubis must be fairly intelligent and have a good memory. They may be of either sex and of station in life, but they must forgo their outside activities when they join the priesthood.
\subsubsection{Apprenticeship}
Any devoted follower of Anubis may apply to enter the priesthood. They are given a set of exams, administered by the Recenzisto, which test the applicant’s mental and memorization powers. If they pass these tests, they are welcomed into the Sepulcher as an acolyte.

Acolytes are responsible for attending sessions to learn Tusparol, study the Morto-Vojo, and train for the offices they may hold. Anytime during the apprenticeship, an unsuccessful candidate can be dismissed by their instructors and simply be ejected from the Sepulcher.
\subsubsection{Initiation}
When an entry level office is vacated, the Recenzisto chooses the most senior acolyte, with the proper skills, to fill it. The acolyte is sequestered deep within the catacombs beneath the Sepulcher to meditate, alone, for ten days. They must then either commit to the priesthood or leave
the Sepulcher.

If the acolyte commits, they are dressed in the proper ceremonial robes, adornments, and makeup, and led into the ordination ceremony by the high priest. The acolyte will find a willing victim, or possibly a convicted criminal, secured to the Sepulcher’s main altar. The acolyte must speak the Litany of Passing, then send the sacrifice’s spirit to Anubis in the prescribed manner. They are then a priest and accepted as such by all members of the Sepulcher. A feast is often held to celebrate this event.
\subsubsection{Duties}
Each priest is responsible for three kind of duties. First, and most formal, each priest is assigned an office that is
tied to a particular section of a particular ceremony. The priest is responsible to know the proper litany and gestures,
must maintain any needed supplies, and supervise any subordinate priests.

Each priest is also involved in one aspect of the daily running of the Sepulcher. They will usually have a staff of one or more prisoners to accomplish these tasks, such as food supply, building maintenance and construction, supply acquisitions, care of the resident undead, the publicity office, and care of the Morto-Vojo volumes.

Lastly, each priest must devote at least one day in six to the ongoing discussions of the contents of the Morto-Vojo. The priest examines examples and discusses how situations should be handled. Occasionally the results will be significant enough to be relayed to the central Sepulcher D’mort. The Sepulcher D’mort issues replacement pages for the Morto-Vojo every two to three years, perhaps even a completely new volume.

Outsiders occasionally petition the Sepulcher to lend a priest to assist in investigating murders. Due to their religious practices, Anubian priests are very useful in solving such foul deeds. Often the results surprise the petitioners, as the priest will use their morals in deciding the fate of the murderer or victims.
\subsection{Advancement}
The Recenzisto, reviews all vacated offices and decides the basis of both seniority and merit which priest to elevate. This is a full time job, as they hold extensive discussions on each position, and elevation usually causes a ripple of reassignments down the office tree.

They are also responsible for any inquiries into improper or unholy activities of any Anubian priest. The involved priest(s) are brought before them and the Recenzisto seeks the truth by questioning them and any other involved parties.
\subsection{Dogma}
\begin{center}“There is but one path to Paradise, but billions lead to despair...”\end{center}

Knowing there is only one correct way to do any task, the priests and followers of Anubis spend an incalculable amount of time memorizing the passages of the Morto-Vojo. This incredibly large (312 volumes at present) document describes the proper way to deal with any situation, from those as mundane as how to bathe, to the most complex, such as greeting a foreign head of state when one wishes to show displeasure, but not unfriendliness.
\subsection{Traditions}
\subsubsection{Clothing}
Anubis’ priests adorn themselves in voluminous ceremonial robes of red and white cloth, symbolizing the strength of blood and the purity of the soul. Clothes used outside official duties within the Sepulcher are of the same color scheme, but simpler and more utilitarian.
\subsubsection{Appearance}
Priests use jewelry and very carefully applied makeup, to indicate their exact mood and situation. The rules governing appearance are so complex that little other than extremes can be noticed by those ouside the priesthood. Priests consider being seen without their proper adornment the same as being caught naked.
\subsubsection{Speech and Gestures}
Anubian priests carefully consider every statement before speaking. Control is very important to them, for they are the models of proper and good behavior to which others should adhere. They are not obtuse or deceptive; they are normally straight to the point and usually quite truthful.

The priests have developed a full language of hand gestures that allows them to impart complex and subtle meaning at a blinding speed. Called Tusparol, this language is always used when speaking priest-to-priest outside official ceremonies. Some assassins and professional soldiers have also learned this language from the few surviving defrocked priests. (The EP cost of learning Tusparol is double that of other languages, and is restricted to the priests of Anubis).
\section{Worship}
Like the priesthood, worshipers of Anubis are very structured in their spiritual duties. Several volumes of the Morto-Vojo detail the responsibilities and procedures for meeting those responsibilities.
\subsection{Sacrifices}
As detailed in Morto-vojo volume 172, followers of Anubis are expected to make one major and 4 minor sacrifices each year. The dates and type of sacrifices are related to the birth date of the worshiper. For children, these sacrifices are performed by their parents and are appropriately
down-scaled. The Sepulcher sends out reminders and schedules to assist their parish in the complex timing of these sacrifices.
\subsection{Donations}
Morto-vojo volume 83 has a schedule of donations for each of the 317 listed professions. Dates and amounts are further separated into prosperity brackets, and range from 20 to 30 percent of the worshipper’s income.
\subsection{Obligations}
These obligations are clearly stated in volumes 112 through 155 of the Morto-vojo. They describe in detail the exact conditions that activate each option. They involve service to the temple, financial support, military duty during emergencies, and the conditions under which people can be
delivered into Anubis’ embrace.
\subsection{Penance}
Volumes 16, 102 and 305 list sins and crimes against Anubis. For each sin, specific punishment is proscribed. This penance can be financial, extra spiritual guidance, public humiliation, or temporary incarceration. Never is death used as a penance since that is what all anubians seek. Sins commited to others are often redressd during the yearly Penance of the Faithful observance.
\subsection{Advice}
The Office of Public Information and Guidance in the Sepulcher is staffed by trained priests ready to answer any questions of interpretation of holy writings and how they affect normal life.
\section{Prayers}
Understanding the orginized procedures of thier diety, the followers of Anubis hold their entrities to him until the yearly Festival of Supplication
\section{Holidays and Feast Days}
The Festival of Death is held on the first day of Pim each year. Dedicated to all those who have passed on in the previous year, this is a very beautiful and enrapturing ceremony. Thousands of candles are lit, choirs sing sonorous songs about the afterlife, and a rich repast of rare and
delicious foods is served.

During the day, competitions and games are held to prove the physical and mental strength and worthiness of the worshippers. Groups and individual events are held, eliminating all but one who is declared Champion of the Festival of Death. The culmination of the evening’s celebration is when the winner is delivered to meet Anubis, in person.

The Festival of Supplication is held on first day of each year. Supplicants fill out the paperwork listing their hopes for the upcomming year and submit it to the priesthood during the morning and afternoon hours. At second bell a worship service is help where the faithful pray for Anubis’ blessings and guidance. A lottery is held, and one is chosen to be Herald. They are given the combined paperwork and at sundown they are placed on a pyre and sent to seek Anubis, alive and conscious. There it is presumed they submit the paperwork to their god in person bringing the pleas of his worshipers.

The Penance of the Faithful is held on the second Frand of the month of Irkusk each year. During this day, Anubis’ followers, like Anubis himself, maintain silence from first bell until first bell of the following day. They are tasked with redressing the wrongs of the last year. They seek out those they have wronged repaying debts, helping the harmed and otherwise making right for their sins. It is extremely bad karma for anyone to stand in their way or otherwise interfere with this holy task. Many cities, towns and villages have rules in place to punish such blasphemers.

The Festival of the Last Word, held on the first day of Kild, is the last chance to speak with the newly delivered. In the days leading to this festival the faithful submit paperwork to the priesthood requesting to speak with their loved ones or close friends whom have traveled on to Infero in the last year. A worship service is held one house before sunset. There a volunteer is declared Herald and given the list of names. They are delivered by the high priest. Arriving in Infero they present the list to Anubis’ assistant for processing.

As the light fades, the faithful seek the resting place of those they have sought. If they have prayed, and been true to Anubis, he allows their passed one to make a spiritual appearence for the short time of twilight, between the worlds of light and darkness. They may ask them questions, give assurances and otherwise gain closure from the passing of their loved ones.
\section{Relationship to Other Dieties}
Most deities are concerned with the lives of their followers. Since life is but a training ground to the real existence of death, these other gods creeds, rules and followers are mosly unimportant to the priests and worshippers of Anubis. However they do maintain a relationship to Isis and her followers considering them stewarts of the welfare of those preparing for Infero.

\section{Spells}
\subsection{Tomboloko}

\spellentry{Find Dead}{1}{Time to Cast: 1 R}{Resist Check: none}{Target: none}{Duration: 10 + 5/F M}{Area: 100 + 50/F' radius}{Effect: locate dead}{Casting this spell causes any dead or undead bodies within the area of effect to radiate a cool white light visible only to the caster. This light can be seen through any material other than iron or adamantine.}

\spellentry{Tombstone}{2}{Time to Cast: 1 M}{Resist Check: none}{Target: touch}{Duration: permanent}{Area: 1 grave marker}{Effect: finishes marker}{Casting this spell on a block of rough-cut stone allows the caster to quickly fashion a finished and inscribed grave marker. The marker can contain any markings which the caster would have been capable of enscribing with the proper enscribing tools.}
\spellentry{Grave Sight}{3}{Time to Cast: 1 M}{Resist Check: none}{Target: touch}{Duration: 1 T}{Area: 1 grave}{Effect: view grave contents}{Casting this spell on a grave marker or a grave will cause a vision to appear to the caster of the contents of the grave.}
\spellentry{Preserve Dead}{4}{Time to Cast: 1 M}{Resist Check: 4d6 vs HEA negates}{Target: 10 + 5/F'}{Duration: 24 + 12/F H}{Area: one body}{Effect: preserves dead tissue}{The caster keeps dead tissue from further decay by casting this spell. If the tissue is animate (as in undead) it can avoid the effects by making a successful RC.}
\spellentry{Grave}{5}{Time to Cast: 1 M}{Resist Check: none}{Target: 10 + 5/F'}{Duration: 10 + 5/F M}{Area: 1 grave}{Effect: opens grave}{Any non-rock ground will split open in a 7 foot by 4 foot rift of up to 6 feet deep when affected by this spell. The caster may close the rift only during the spell's duration.}
\spellentry{Grave Lock}{6}{Time to Cast: 1 M}{Resist Check: none}{Target: touch}{Duration: 20 + 10/F weeks}{Area: one grave}{Effect: protects grave}{This spell allows the caster to protect a grave from grave robbers. Any attempt to open or desecrate the grave will cause 3d8 damage points to the violator.}
\spellentry{Vervakadavro}{7}{Time to Cast: 5 R}{Resist Check: none}{Target: 30 + 10/F'}{Duration: 2 + 1/F H}{Area: one dead body}{Effect: allows movement}{A dead, but whole, body can be given movement by this spell. After the casting, the dead body will follow the caster's simple orders involving movement. The animated body cannot manipulate objects or be given orders about the future, as the spell is only }
\spellentry{Coffin}{8}{Time to Cast: 10 M}{Resist Check: none}{Target: touch}{Duration: instantaneous}{Area: 1 coffin}{Effect: crafts coffin}{Given a sufficient amount of wood, this spell will quickly fashion a box suitable for internment of a body. The workmanship will be equivalent to what the caster could do normally with the proper tools, but the magic crafts the coffin quickly and efficien}
\spellentry{Grave Ward}{9}{Time to Cast: 2 M}{Resist Check: none}{Target: touch}{Duration: 40 + 20/F weeks}{Area: one grave}{Effect: protects grave}{This spell allows the caster to protect a grave from grave robbers. Any attempt to open or desecrate the grave will cause 6d6 damage points to the violator.}
\spellentry{Regenerate Dead}{10}{Time to Cast: 10 M}{Resist Check: none}{Target: touch}{Duration: instantaneous}{Area: 1 body}{Effect: restores decay}{The caster can take a whole, but decomposed, dead body and cause its tissues to regenerate, leaving the body in a healthy, but still dead, state. This spell cannot be used on the living or undead.}
\spellentry{Shrine}{11}{Time to Cast: 10 M}{Resist Check: none}{Target: touch}{Duration: instantaneous}{Area: 1 monument}{Effect: make grave marker}{Cast upon a suitable amount of loose rocks and stones, this spell will fashion an appropriate monument for a grave. The workmanship will be only what the caster is capable of, but the monument will be completed by the end of the spell.}
\spellentry{Grave Curse}{12}{Time to Cast: 10 M}{Resist Check: none}{Target: touch}{Duration: special}{Area: special}{Effect: curses defiler}{The priest says this warding over a recently (less than one year) buried person. While touching the dirt of the grave, incanting the deceased's name, and visualizing the circumstances of death, the priest places a ward upon the grave. When anyone attempts}
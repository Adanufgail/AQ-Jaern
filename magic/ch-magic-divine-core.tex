\chapter{Divine Core Magic}
\label{ch:divine-core-magic}
\section{Organization}
All priests, except those who worship T’or, have access to certain basic magics in addition to the spell groups specific to their sects.

Ceremonies are the magical component of the standard ceremonies performed by all sects. While these magics are shared, the actual form of the ceremony always differs from sect to sect. The spells have few noticeable effects, but they are very valuable from a roleplaying point of view.

Revocation spells are used to cancel another priest’s magic. They have no effect on elemental magic.

Blessings allow the priest to lay his hope for good fortune on deserving followers of his flock and favored friends. All blessings last no longer than 24 hours, or until the time they take effect. The priest can only cast blessings on others, not himself. When the priest casts a blessing on one of his own faith, it works automatically. When cast on one outside the faith, the GM will ask the target to succeed a check against the target’s PWR of a number of dice reflecting the GM’s opinion of the target’s piety and similarity of creed. Only one blessing can be on a target at one time. Blessings may not be used in conjunction with the Defer spell.

Fabrication spells allow the caster to create and manipulate various objects and holy writs.

Detections are used to inform the caster of different things around them. These spells also allow the priest to reach into a someone's mind and learn their inner self.

Influence spells are used by priests to show others the way of their faith, and guide them along the true path.

Bind spells are the means by which a priest manufactures magical items and creates permanent or delayed spell effects.

Defer spells are the means by which a priest manufactures divine items or delayed spell effects.

\section{Divine Core Spells}
\renewcommand{\labelenumii}{\arabic{enumii}.}
\begin{tabular}{@{} p{0.25\linewidth} p{0.25\linewidth} p{0.25\linewidth} p{0.25\linewidth}}
\begin{enumerate}
	\item Ceremonies
	\begin{enumerate}
		\item Worship
		\item Consecrate Item
		\item Dedication
		\item Unification
		\item Last Rites
		\item Ordination
		\item Excommunication
		\item Atonement
		\item Mortify
		\item Sanctify
		\item Forbiddance
		\item Miracle
	\end{enumerate}
\end{enumerate} &
\begin{enumerate}
	\setcounter{enumi}{1}
	\item Revocation
	\begin{enumerate}
		\item Revocation 1
		\item Revocation 2
		\item Revocation 3
		\item Revocation 4
		\item Revocation 5
		\item Revocation 6
		\item Revocation 7
		\item Revocation 8
		\item Revocation 9
		\item Revocation 10
		\item Revocation 11
		\item Revocation 12
	\end{enumerate}
\end{enumerate} &
\begin{enumerate}
	\setcounter{enumi}{2}
	\item Defer
	\begin{enumerate}
		\item Defer 1
		\item Defer 2
		\item Defer 3
		\item Defer 4
		\item Defer 5
		\item Defer 6
		\item Defer 7
		\item Defer 8
		\item Defer 9
		\item Defer 10
		\item Defer 11
		\item Defer 12
	\end{enumerate}
\end{enumerate} &
\begin{enumerate}
	\setcounter{enumi}{3}
	\item Blessings
	\begin{enumerate}
		\item Divine Grace
		\item Deter Fate
		\item Abate Fatigue
		\item Optimize Onslaught
		\item Enhance Potential
		\item Defy Injury
		\item Augment Task
		\item Attract Fate
		\item Forestall Conflict
		\item Second Chance
		\item Abate Outcome
		\item Share Grace
	\end{enumerate}
\end{enumerate} \\
\begin{enumerate}
	\setcounter{enumi}{4}
	\item Fabrications
	\begin{enumerate}
		\item Create Water
		\item Speak The Word
		\item Create Bread
		\item Capture The Word
		\item Create Fish
		\item Create Meat
		\item Transfer The Word
		\item Create Holy Water
		\item Create Holy Symbol
		\item Create Fervor
		\item Produce Vestments
		\item Produce Truth
	\end{enumerate}
\end{enumerate} &
\begin{enumerate}
	\setcounter{enumi}{5}
	\item Detections
	\begin{enumerate}
		\item Detect Divinity
		\item Uncover Affection
		\item Detect Life
		\item Unveil Fear
		\item Detect Captivation
		\item Share Joy
		\item Discern Motivation
		\item Disclose Sin
		\item Reveal the Past
		\item Divulge Virtue
		\item Commune
		\item Manifest Destiny
	\end{enumerate}
\end{enumerate} &
\begin{enumerate}
	\setcounter{enumi}{6}
	\item Influences
	\begin{enumerate}
		\item Revoke Fear
		\item Instill Fear
		\item Share Vision
		\item Attention
		\item Paralyze
		\item Curse
		\item Revoke Curse
		\item Enthrall
		\item Devotion
		\item Disenchant
		\item Quest
		\item Divine Word
	\end{enumerate}
\end{enumerate}
\end{tabular}
\pagebreak

\section{Ceremonies}

\spellentry{Worship}{1}{1 Minute}{None}{None}{1 Minute}{Audible Distance}{Draw diety's attention}{This spell is used in conjunction with worship ceremonies. The priests cast this to gain the attention and favor of their deity.}
\spellentry{Consecrate Item}{2}{1 Minute}{None}{Touch}{Permanent}{50 Pound item}{Make item holy}{This spell is used to dedicate an item to the priest's deity. This is normally done on many of the implements and tools used during worship ceremonies. This will cause the item to have a faint glimmer when viewed with the Witchsmeller and Detect Divinity spells.}
\spellentry{Dedication}{3}{2 Minutes}{None}{Touch}{Permanent}{30 + 10/Finesse Foot radius}{Make place holy}{This spell dedicates a place and makes it holy to the priest's deity. This can fail for various reasons related to the place's past.}
\spellentry{Unification}{4}{3 Minutes}{Willing Target only}{Touch}{Permanent}{Target}{Dedicates worshiper}{Unification is the process of making a person acceptable for worship of a deity. When someone declares that they wish to worship a deity, when they have meet all the other requirements, the priest uses this spell to alert the deity to the existence of the new worshiper. This ceremony is also often used to indicated that children have "come of age" and are full worshipers.}
\spellentry{Last Rites}{5}{5 Minutes}{None}{LOS 10 Feet}{Instantaneous}{Target}{Puts soul to rest}{This spell allows the soul of a newly dead worshipper of a deity to return to their deity. If this spell is not cast, the soul still inhabits the dead body, in pain and powerless. Eventually such souls are either collected by Anubis, the God of the Dead, or drift into the Kurago. While a soul still inhabits the body, the body can be used for many gruesome purposes, including the creation of undead creatures.}
\spellentry{Ordination}{6}{10 Minutes}{Willing Target only}{Touch}{Permanent}{Target}{Inducts follower as a priest}{This ceremony is used to induct a worshiper into the priesthood. Check the appropriate deity section for the actual form of the ceremony.}
\spellentry{Excommunication}{7}{15 Minutes}{None}{Special}{Permanent}{Target}{Expel worshipper}{When a follower must be expelled from the flock, this ceremony is used to withdraw the protection of the deity from the wrongdoer. The target need not even be present for this cerimony. The priest had better be sure of their reasoning. If the deity's opinion is that the target is expelled wrongly, it is the priest who is excommunicated.}
\spellentry{Atonement}{8}{20 Minute}{Willing Target only}{Touch}{Permanent}{Single target}{Renews commitment}{An excommunicated follower can be brought back into the fold. After they meets other requirements, this spell renews their connection to their deity.}
\spellentry{Mortify}{9}{1 Hour}{None}{Special}{Special}{Target}{Punish heretic}{When anyone commits a heinous crime against a temple, the priests may perform a ceremony of Mortification. The ceremony lays a great curse upon the target, invoking the power of the deity to punish the wrongdoer. The target need not even be present to use this spell. It must be cast by 12 or more priests in unison to have effect. Temples must be careful to cast Mortify only when needed: doing so frivously raises the ire of the priests' own diety.}
\spellentry{Sanctify}{10}{1 Day}{None}{Touch}{Permanent}{1 Structure}{Dedicates temple}{This is used to dedicate a new temple or shrine to a deity.}
\spellentry{Forbiddance}{11}{1 Hour}{5d6 vs WIL negates}{LOS 250 Feet}{Permanent}{Target}{Prevents action}{This spell is similar to Quest, but rather than causing the target to perform an action, it prevents the target from performing a stated action.}
\spellentry{Miracle}{12}{1 Minute}{None}{Special}{Special}{Special}{Affect a miracle}{This ceremony is used by the priest to ask their deity to assist them to perform any stated miracle. The success of the miracle is not dependant on the spell power, but on the whim or will of the deity. Adjudication of the miracle depends on the circumstances and the GM's discretion.}

\section{Revocation}

\spellentry{Revocation}{1-20}{3 Rounds}{None}{LOS 80 + 20 / Finesse Feet}{Instantaneous}{1 Magical Effect}{Nullify divine power}{}

\subsection{Revoking Spells}

The Revocation group is different from others in that the spell remains the same throughout, except that each successive spell is more powerful than the previous. It can affect both spells and items; potions and other single use items are considered as spells for revocation purposes.

The mechanics for determining if a revocation is successful are simple. The player rolls a number of d6 equal to the rank of the revocation their character is casting and totals them. The GM rolls a number of d6 equal to the rank of the effect the caster is attempting to revoke and totals them. The higher total wins, i.e., if the player's total is higher the targeted effect is negated; if the GM's total is higher the revocation fails.

\subsection{Affecting Items}

Each magical item has at least two components: a spell efffect and a Defer. These are specified in the item's description. To temporarily suspend the effect of an item (or draw charges from a charged item), the caster attempts to revoke the spell effect. Dice are rolled as described under \secref{magic-elemental-core-revoking-spells}. If the player’s total is higher the Revocation works and the difference between their and the GM's totals is the number of rounds the effect is negated, or the number of charges drawn.

To permanently cancel an item the caster must revoke the Defer spell. Dice are rolled, and if the player’s total is higher than the GM's the item is made permanently non-magical.

The divine Revocation spell can only be used to revoke divine spells.

\pagebreak

\section{Defer}

\spellentry{Defer}{1-20}{1 Minute}{None}{Touch}{2 Hours}{1 Spell}{Delay spell effects}{}

Casters use the Defer spell to delay the effects of another spell. The caster picks some non-living, non-magical object that they can hold in one hand and lift and cast the Defer spell on the object. This places a magical field about the object that can hold one other spell of a rank equal to or less then the rank of the Defer spell used. Immediately after this, the caster casts the other spell into the same item. The magical field created by the defer ensnares this second spell, hold within the field. 

As long as the defer spell lasts, the caster can touch the item and direct the ensnared spell to discharge, having the same effect as if cast normally. The spell’s release from the defer takes a single round, despite the casting time of the spell. If the defer spell expires, without triggering its contained spell, all magic energies involved dissipate harmlessly.

If the caster wants someone other than themselves to be able to release and direct the spell, they may finesse the original defer to include a trigger that others can use. The number of finesses determines what kind of trigger is used, as follows:

\begin{tabular}{@{} l l}
0 & Thought Activated for the original caster\\
1 & Mechanical\\
2 & Spoken Word or Phrase\\
3 & Thought Activated by anyone\\
4 & Activated by a condition
\end{tabular}

Spells cast with an unfinessed defer can only be thought activated by their original caster, when they are touching the item. One finesse allows the spell to be triggered by any moving part of the item. Two finesses allow the spell to be triggered by a keyword or phrase. Three finesses allow the spell to be thought activated by anyone holding the item. The fourth finesse allows the caster to state the single condition that will activate the item. This condition can only describe a state of being, and can only be based on the position, condition or existence of physical objects or energies within 10 Feet of the item.

The caster may choose to target the spell either at the time of casting the defer spell, or when it is triggered. If it is targeted when triggered, the person triggering the spell can direct it mentally, just as if they were the caster. If it is targeted when the defer is originally cast, the caster gives targeting information relative to the position and orientation of the object used for the defer spell.

Defer spells used on items with already existing defer spells cast by a different caster don’t work. Recasting a new defer spell on your deferred spell extends the duration of the original defer spell by an additional 2 hours.

The divine Defer spell can only be used to contain divine spells.

\pagebreak

\section{Blessings}

\spellentry{Divine Grace}{1}{2 Minutes}{4d6 vs WIL negates}{Single creature}{24 Hours/Until used}{Target}{Raise \% for DI}{For a period of 24 hours after this blessing is laid on a target by a priest, the target is in a state of divine grace. If the target attempts to call upon a deity for intervention and fails, they can ignore the roll and roll again to check for success. Once used, the blessing ends, and further attempts are made as normal.}
\spellentry{Deter Fate}{2}{2 Minutes}{4d6 vs WIL negates}{Single creature}{24 Hours/Until used}{Target}{Avoid one selection}{For a period of 24 hours after this blessing is laid on a target by a priest, one random selection can be avoided. When the GM starts to choose which adventurer is effected by an event by using a random die roll, a target with this blessing expends it by asking to be excluded from the pool of possible targets. The GM then makes the selection, ignoring the target.}
\spellentry{Abate Fatigue}{3}{2 Minutes}{4d6 vs WIL negates}{Single creature}{Instantaneous}{Target}{Borrow against rest}{The priest lays their hands on the target, helping it recuperate. This has the effect of one night's rest, healing the target and regenerating their spell ability as if they had rested for eight hours (or twelve for a soulless humanoid such as an elf). This is borrowing against the target's future recuperation powers; the next rest period will have no effect. This ability cannot be used multiple times in a row without first taking the un-restorative rest.}
\spellentry{Optimize Onslaught}{4}{2 Minutes}{4d6 vs WIL negates}{Single creature}{24 Hours/Until used}{Target}{Increase hit chance}{For a period of 24 hours after this blessing is laid on a target by a priest, on any physical attack roll that the target fails they may choose to ignore the first roll and attempt the roll again. Once this option is taken, this blessing expires.}
\spellentry{Enhance Potential}{5}{2 Minutes}{4d6 vs WIL negates}{Single creature}{24 Hours/Until used}{Target}{Increase chances}{For a period of 24 hours after this blessing is laid on a target by a priest, the target can choose to use the blessing on any one dice roll. When they do this, they make the roll twice, and the higher of the two totals is the true result.}
\spellentry{Defy Injury}{6}{2 Minutes}{4d6 vs WIL negates}{Single creature}{24 Hours/Until used}{Target}{Ignore one attack}{For a period of 24 hours after this blessing is laid on a target by a priest, the target can ignore the damage from a single physical hand to hand attack. After the GM announces the adventurer has been hit, and before they announces the result, the player can state they are expending the blessing to ignore the damage.}
\spellentry{Augment Task}{7}{2 Minutes}{4d6 vs WIL negates}{Single creature}{24 Hours/Until used}{Target}{Raise \% on skill check}{For a period of 24 hours after this blessing is laid on a target by a priest, when the target attempts a check against a skill and fails, the player can expend this blessing to ignore the result. They then roll the same check again and abide by this new result.}
\spellentry{Attract Fate}{8}{2 Minutes}{4d6 vs WIL negates}{Single creature}{24 Hours/Until used}{Target}{Attract one selection}{For a period of 24 hours after this blessing is laid on a target by a priest, one random selection can be attracted. When the GM starts to choose which adventurer is effected by an event by using a random die roll, a target with this blessing expends it by asking to be the affected one. The GM then acts as if that character is the selected target.}
\spellentry{Forestall Conflict}{9}{2 Minutes}{4d6 vs WIL negates}{Single creature}{24 Hours/Until used}{Target}{Interrupt conflict}{In the next 24 hours, the target can expend this blessing by calling out "Stop in the name of" the blessing's deity. This can occur any time actions are being declared. The GM gives results of the already stated actions ending the current round. All present spend a round taking no actions, but events such as falling rocks or duration spells continue. Then the GM starts a round giving the target and their allies advantage.}
\spellentry{Second Chance}{10}{2 Minutes}{4d6 vs WIL negates}{Single creature}{24 Hours/Until used}{Target}{Raise \% on RC}{For a period of 24 hours after this blessing is laid on a target by a priest, when the target attempts a RC and fails, the player can expend this blessing to ignore the result. They then roll the same RC again and abides by its result.}
\spellentry{Abate Outcome}{11}{2 Minutes}{4d6 vs WIL negates}{Single creature}{24 Hours/Until used}{Target}{Minimize one result}{The target expends this blessing when the GM is rolling one result which will effect the target. The effect then occurs to the target as if the GM rolled the minimum on each die of the effect. For example, a fireball that would do 8d6 of damage does only eight points of damage to the target. This does not change the effect on others sharing the same outcome.}
\spellentry{Share Grace}{12}{2 Minutes}{4d6 vs WIL negates}{Single creature}{24 Hours/Until used}{Target}{Bless a group}{A priest uses this spell in combination with any of the other blessings of up to rank 8. The second blessing then can effect as many people as the priest's CSE stat. The combined casting time of the two spells is sequential (4 minutes total).}

\pagebreak

\section{Fabrications}

\spellentry{Create Water}{1}{1 Round}{None}{Touch}{1 Minute}{2 people/1 day}{Creates water}{Upon casting this spell, and striking the ground or a rock, a stream of clean, pure water begins flowing. There will be enough water to sustain two people for one day.}
\spellentry{Speak The Word}{2}{1 Round}{None}{Caster}{1 + 1/Finesse Minutes}{Hearing}{Reproducing holy writ}{While concentrating on a holy writ (a speech), the caster casts this spell which then allows them to issue forth the writ, letter perfect and in the voice of the original speaker.}
\spellentry{Create Bread}{3}{2 Rounds}{Negates}{Touch}{Permanent}{2 + 1/Finesse people/day}{Changes stone to bread}{This spell transmogrifies stones or other earthen objects into bread. There will be enough to satisfy the needs of two people for one day, plus an additional person per finesse.}
\spellentry{Capture The Word}{4}{1 Round}{None}{Caster}{5 + 5/Finesse Minutes}{Hearing}{Records speech}{This spell allows the caster to memorize the words of a speaker so they can later repeat them using Speak The Word. This spell cannot be used to capture the magical speech of others using Speak The Word.}
\spellentry{Create Fish}{5}{4 Rounds}{Negates}{Touch}{Permanent}{4 + 1/Finesse person/day}{Changes water to fish}{When cast upon a vessel of water, the water is transmogrified into enough fish to sustain four people for one day.}
\spellentry{Create Meat}{6}{8 Rounds}{Negates}{Touch}{Permanent}{8 + 2/Finesse people/day}{Changes plant matter to meat}{Transmogrifies any touched non-living vegetable material into enough fresh red meat to sustain eight people for one day.}
\spellentry{Transfer The Word}{7}{1 Minute}{Willing target}{Touch}{10 + 5/Finesse Minutes}{Target}{Transfers holy writ}{A priest uses this magic to teach a magical writ to a willing colleague. The priest casts this spell and the target goes into a deep, hypnotic trance. The priest then casts Speak The Word and intones the writ to transfer. When the target recovers from the trance, they know the holy writ.}
\spellentry{Create Holy Water}{8}{1 Minute}{Negates}{Touch}{Permanent}{1 + 1/Finesse Liters}{Sanctifies water}{This spell will sanctify up to one pint water, making it holy.}
\spellentry{Create Holy Symbol}{9}{1 Round}{None}{0}{Permanent}{1 symbol}{Creates a holy symbol}{The priest must clench their fist while intoning the spell. At the end of the casting time they'll be grasping a newly created holy symbol of their faith. Its composition depends on the finesses applied at the time of casting: 0) wood, 1) iron, 2) silver, 3) gold, or 4) platinum.}
\spellentry{Create Fervor}{10}{2 Minutes}{4d6 vs WIL negates}{0}{20 + 10/Finesse Minutes}{Hearing}{Create religious zeal}{The caster uses this spell in combination with Speak The Word to intone a holy writ to a group of people. Each person is allowed the RC; any who fail will embrace the writ and get enthusiastic.}
\spellentry{Produce Vestments}{11}{2 Rounds}{None}{Touch}{Permanent}{Target}{Creates new robes}{This spell allows the caster to create a new set of robes for themselves. What they were previously wearing is destroyed in the process. All valuables and magic are unaffected. The robes can be of any design, but must use non-precious materials.}
\spellentry{Produce Truth}{12}{1 Round}{None}{Caster}{1 + 1/Finesse Minutes}{Hearing}{Verifies truth}{When a priest uses this spell, anything they say will be completely accepted as the truth by their listeners. If the caster attempts to tell a lie, a half-truth, or even an intention to mislead while this spell is in effect, they must make an RC of 6d6 vs HEA or die.}

\pagebreak

\section{Detections}

\spellentry{Detect Divinity}{1}{1 Round}{None}{Caster}{10 + 5/Finesse Minutes}{LOS 200 Feet}{Shows divinity}{This spell causes divine things to glow with a white aura. The stronger the holiness, the brighter the light appears. Any holy relic or artifact can be easily detected with this spell. Priests have a faint glimmer, and prophets shine brightly.}
\spellentry{Uncover Affection}{2}{2 Rounds}{4d6 vs WIL negates}{Touch}{1 + 1/Finesse Minutes}{Target}{Reveals affection}{Gives the priest a vision of the person for whom the target feels the most affection. If no vision forms, the target either bears no affection for anyone or made the RC. In some cases the vision can be of the target.}
\spellentry{Detect Life}{3}{1 Round}{None}{Caster}{1 + 1/Finesse Minutes}{LOS 200 Feet}{Shows living things}{Anything the priest sees after casting this spell which is alive is surrounded by a faint blue glimmer.}
\spellentry{Unveil Fear}{4}{2 Rounds}{4d6 vs WIL negates}{Touch}{1 + 1/Finesse Minutes}{Target}{Reveals worst fear}{Gives the priest a vision of the target's worst fear.}
\spellentry{Detect Captivation}{5}{1 Round}{None}{Caster}{2 + 1/Finesse Minutes}{LOS 200 Feet}{Reveals charmed creatures}{People and creatures under the influence of mind-controlling spells are outlined by an orange aura with this spell.}
\spellentry{Share Joy}{6}{2 Rounds}{4d6 vs WIL negates}{Touch}{1 + 1/Finesse Minutes}{Target}{Reveals most joyful event}{A vision of the event in the target's past which brought them the most happiness appears to the priest when this spell is used.}
\spellentry{Discern Motivation}{7}{1 Round}{5d6 vs WIL negates}{Caster}{3 + 3/Finesse Minutes}{LOS 200 Feet}{Reveals motivation}{A glimmer appears around all creatures and peoples with an INT greater than 6 for the spell's duration. The glow varies from deep red to brilliant white, or any shade in between. The priest intuitively knows what motivations the colors represent.}
\spellentry{Disclose Sin}{8}{2 Rounds}{4d6 vs WIL negates}{Touch}{2 + 2/Finesse Minutes}{Target}{Shows worst sin}{Gives the priest a vision of the target's (in the target's opinion) most heinous sin.}
\spellentry{Reveal the Past}{9}{1 Minute}{4d6 vs WIL negates}{Touch}{10 + 10/Finesse Minutes}{Target}{Shows past actions}{Allows the priest to show a vision of some event in the target's past. The caster visualizes the proper time and day, and casts this spell upon the target. A vision appears in the air for all to see, of the events of that time.}
\spellentry{Divulge Virtue}{10}{2 Rounds}{4d6 vs WIL negates}{Touch}{3 + 3/Finesse Minutes}{Target}{Reveals Virtue}{The priest receives a vision of the target's most virtuous act (in the target's opinion) after casting this spell.}
\spellentry{Commune}{11}{10 Minutes}{None}{None}{1 question}{None}{Deity answers question}{The priest's deity may deign to answer one question, which must be asked by the priest who cast the spell. Take care casting this; the answer may not be without cost.}
\spellentry{Manifest Destiny}{12}{10 Minutes}{4d6 vs WIL negates}{Touch}{5 Minutes}{Target}{Shows destiny}{This spell should be used with great caution, for it will reveal a vision, for all to see, of the final, unalterable destiny of its target.}

\pagebreak

\section{Influences}

\spellentry{Revoke Fear}{1}{1 Round}{4d6 vs WIL negates}{LOS 20 + 10/Finesse Feet}{Instantaneous}{Target}{Negates fear}{This spell causes its target to lose all cause for unreasonable fear, magical or not, and regain control of their actions.}
\spellentry{Instill Fear}{2}{1 Round}{3d6 vs WIL negates}{LOS 30 + 10/Finesse Feet}{1 + 1/Finesse Minutes}{Target}{Causes fear}{The target of this spell suddenly becomes unreasonably fearful of all people and creatures about him. The exact nature of the fear is up to the person playing the target to roleplay.}
\spellentry{Share Vision}{3}{1 Round}{4d6 vs WIL negates}{Caster}{10 + 10/Finesse Minutes}{30 + 10/Finesse Foot radius}{Share a vision}{The priest casts this spell prior to any other spell or effect which would give them a vision. The vision may then be seen by all within the area of effect.}
\spellentry{Attention}{4}{1 Round}{4d6 vs WIL negates}{Caster}{1 Minute}{Clear hearing}{Forces others to listen}{Grabs the attention of those within clear hearing of their voice and make them listen to what they has to say. It does not affect their opinion of the caster or their message.}
\spellentry{Paralyze}{5}{1 Round}{4d6 vs WIL negates}{LOS 60 + 20/Finesse Feet}{2 + 1/Finesse Minutes}{Target}{Immobilizes target}{The target becomes incapable of voluntary muscle movement. They collapses and cannot move or speak, but are still conscious and able to see and hear around them.}
\spellentry{Curse}{6}{2 Rounds}{Special}{LOS 80 + 40/Finesse Feet}{Special}{1 + 1/Finesse target(s)}{Lays a curse}{This spell allows the caster to lay a curse upon the target. The caster can affect one of the following: attack rolls against them, damage taken, or resistance checks made. The target makes an RC of 4d6 vs their lowest stat to resist the curse. The priest must phrase the curse in game terms, not as changes to rolls. The curse stays in effect until it has successfully affected the cursed target. If successful, the curse causes that effect's die roll to be affected by no more than 30\% of the maximum value of the roll (rounded to the nearest whole number). This cannot cause more than the maximum die roll.}
\spellentry{Revoke Curse}{7}{1 Minute}{Special}{Touch}{Permanent}{Target}{Lifts a curse}{This spell allows the caster to lift a curse from an item or object only. The caster must make an RC of 4d6 vs CSE to succeed.}
\spellentry{Enthrall}{8}{3 Rounds}{3d6 vs WIL negates}{Caster}{10 + 5/Finesse Minutes}{Clear hearing}{Influence others}{Like Attention, this spell forces others to listen to the priest's words. It also causes the targets to make an RC or believe, at least temporarily, what the priest says. When the duration expires, its effects slowly fade over one hour.}
\spellentry{Devotion}{9}{1 Minute}{3d6 vs CSE negates}{Touch}{Permanent}{Target}{Aligns target to caster's faith}{This spell allows the priest to force someone to be devoted to the cause of the priest's deity. This spell doesn't change the target's personality or style, but alters their purpose.}
\spellentry{Disenchant}{10}{1 Minute}{4d6 vs CSE negates}{LOS 10 + 5/Finesse Feet}{Permanent}{Target}{Breaks faith}{The priest casts this to cause the target to waver and break in their devotion to a cause, person, or deity. This must be used with caution, for many deities will take this personally.}
\spellentry{Quest}{11}{1 Round}{5d6 vs WIL negates}{LOS 120 + 60/Finesse Feet}{Special}{1d6 targets}{Quests}{This spell allows the caster to charge the target(s) with a possible task (chosing an impossible task causes the spell to fail). The target must actively seek to complete this quest, or begin suffering damage daily until the quest is completed. This spell is not curable with the revoke curse, but the target can be released by a diety. The damage suffered each day is the sum of the number of days ignored. IE. the first day they ignore it, they begin to suffer 1 DP of damage daily. The second day they ignore it they begin to take 3 damage (1 + 2) per day; on the 3rd ignored day they take 6 (1 + 2 + 3), on the 4th they take 10 (1 + 2 + 3 + 4) DP of damage daily, and so on. Damage occurs at the start of the new day, unless the player is resting, in which case it occurs as soon as they wake up. Choosing to work on completing the quest does not prevent damage, only prevents it increasing.}
\spellentry{Divine Word}{12}{1 Round}{6d6 vs CSE negates}{Caster}{1 Hour}{Clear hearing}{Forces obedience}{This spell requires no motions. The priest simply utters a few words in the form of a command. All within hearing must make the RC or follow that command for the spell's duration.}
\chapter{Skills}
\label{ch:skills}
This chapter contains a list of all the standard skills, where they are applicable, and how they are adjudicated. Please be aware the the GM may modify any check as he sees fit depending on the circumstances. If a skill does not list a specific die roll, the GM will assign an appropriate number of dice to check against the skill.

\setlength{\columnsep}{\defcolwidth}
\begin{multicols*}{2}
\skillentry{Accounting}{INT}{Auger}{130}{4}{Keeping track of accounts and expenditures is important to the merchants and the larger establishments of Karfelon. Creative accounting also can be profitably employed among the less ethical.}	
\skillentry{Acrobatics}{AGI}{Marine}{200}{2}{This skill is useful for gymnastic flips, jumps, leaps, and swings. An adventurer can jump into, or backflip out of, melee. He could jump from a second story window into the street and gain a free round on an opponent. Acrobatics cannot be performed in any sort of metal armor.}	
\skillentry{Acting}{INT}{Nomad}{100}{2}{An adventurer with acting skill has some understanding of how to assume a role and stay in character. The GM must assign difficulties and appropriate checks for this skill based on circumstances.}	
\skillentry{Ambidextrous}{AGI}{Warrior}{150}{2}{An adventurer can use either hand as his primary attack hand. Adventurers, by default, have the same "handedness" as their player. If the adventurer's primary hand is incapacitated or damaged, this skill will let him fight using his secondary hand as if it were his primary hand. Normally, using a weapon in the secondary hand causes the adventurer to suffer a -4 penalty on any to "to strike" roll. Roll \tcdieroll{1d6} for each minus you wish to cancel against the skill rank. If successful, subtract the number of dice rolled from the -4 penalty.}	
\skillentry{Ambush}{INT}{Auger}{150}{2}{If your adventurer knows a person or group is coming, and if they are totally unaware of his presence or intentions, he can set an ambush for them. The GM will determine the amount of time necessary to set the ambush. Roll the skill rank or less on \tcdieroll{2d6}. If successful, the adventurer gets \tcdieroll{1d3} free rounds before normal combat begins.}	
\skillentry{Analyze Trap}{INT}{Auger}{150}{N/A}{An adventurer uses this skill to learn the workings of a set or unset trap. Most traps require a \colorbox{white!40!GreenYellow}{2d6} check against this skill to analyze successfully. If successful, subtract two dice from any attempt to disarm that trap. The GM may set a different roll for success depending on the difficulty of the trap.}	
\skillentry{Animal Calling}{HEA}{Auger}{80}{2}{Ever needful of quick gratification, this skill has been the boon to many lonely travelers. Out in the forests or fields, the sound of the skill user's voice lures unsuspecting woodland and field animals to the side of the caller, ready to become target practice, dinner, a new floor rug, or to meet even a more distasteful fate.}	
\skillentry{Animal Husbandry}{CSE}{Auger}{120}{3}{Domesticated animals (horses, cows, sheep, pigs, jkarr'n, ichitle, etc.) are needed in large numbers for food, transportation and work. Understanding how to raise these creatures successfully is a profitable profession.}	
\skillentry{Animal Training}{WIL}{Nomad}{200}{N/A}{The ability to train mammals to perform on command is regulated by the rank of this skill. It is useful in adventuring, or as a profession.}	
\skillentry{Archeology}{INT}{Auger}{100}{N/A}{This skill lets your adventurer use archeological methods and techniques, but does not give any historical information. Studying days or months on a site, he can determine patterns of lifestyle, diet, wealth, and other generalizations about the ancient inhabitants. On a spot examination, if you succeed the check against this skill, the race, time era, and other simple elementary data about the ancient inhabitants can be learned.}	
\skillentry{Architecture}{INT}{Auger}{65}{3}{Architecture involves the planning and design of buildings. An architect can also give advice on structural weaknesses and suggest repair options, if feasible. Architects always keep building plans as references for future work.}	
\skillentry{Armor Smithing}{INT}{Auger}{65}{2}{This skill is necessary to create and repair armor of all types. This includes some knowledge of working leather, iron, copper, fabrics, sewing and fastenings. Creating good quality armor of normal manufacture usually requires a two dice check, though temporarily repairing damaged armor in the field usually requires a one die check (GM discretion).}	
\skillentry{Arson}{INT}{Auger}{50}{2}{This is the ability to set fire to something and make it appear to have occurred naturally. The GM sets the number of \tcdieroll{d6} for the player to roll against the skill. If successful, anyone investigating the scene of the fire must make a PER check at \tcdieroll{2d6} more than usual to detect the arson.}	
\skillentry{Artillery}{INT}{Marine}{200}{2}{Ballistas, catapults and other artillery weapons are complex to load, aim and fire. An engineer or other adventurer may increase his skill using these weapons. This skill are a combination of different loading and firing styles that gives the artilleryman flexibility. Any ONE of these options can be used in one round. Since most of these weapons are very similar to use, this one skill can be applied to using any of artillery piece. The Quickload option lets the artilleryman load his weapon faster. Each round he is loading an artillery piece, or directing a team loading the weapon, if he can roll \tcdieroll{2d6} and get his artillery rank or less, that counts as two rounds of loading. He can not load and fire in the same round using this option. Knowing just how much to overwind, change aperture sizes or otherwise stress his weapon, the experienced artilleryman can make a Long Shot. He chooses the number of dice for an attempted shot, and rolls them. If the total matches or is less than his artillery skill, then his range increases 25\% of the original range for each die rolled. But this is not without danger. Any "roll to hit" of one indicates that he has overstressed his artillery piece, and it falls to pieces, possibly injuring those about it. A steady hand and a good eye help the artilleryman make a difficult Lethal Attack. Choosing a number of dice, the player rolls those dice. Getting a total of his artillery skill or less makes the number he needs to get a Critical Hit on his "roll to strike" one less for each die he rolled. If he rolls three dice and succeeds, he will score a critical hit on a 17 or better on his "roll to strike". Being able to very carefully set the angle and elevation of his weapon relative to the target, and in spite of the rolling deck, is what allows an artilleryman to make a Precise Attack. The player picks a number of dice, making the check against against his weapon skill. If this succeeds he may add 2 for each dice used in the check to the value of his "roll to strike" during this round.}	
\skillentry{Artistry}{CSE}{Auger}{80}{4}{Painting, sculpting, dancing, or executing any form of artistic expression requires this skill to avoid being laughed out of town as a rube or charlatan.}	
\skillentry{Assassination}{AGI}{Warrior}{500}{N/A}{This skill represents an organized and prepared attempt to kill a target. The preparations must be arranged at least one hour prior to the attack. Guards, security precautions, disguises, access to the site before the attempt and the accessibility of the target will all influence the number of dice which the GM will assign to the attempt. For physical melee attacks, if successful, the attacker rolls on the assassination column of the appropriate critical wound table. If the check fails, the attacker makes a normal melee roll instead. For other styles of killings, like poisonings and "arranged" accidents, the GM will determine the results of a success or a failure.}	
\skillentry{Astrology}{INT}{Nomad}{250}{RESERVED}{The trained astrologer looks at the position of the stars and the planets in the night sky. Some believe that future events can be divined by someone with this skill. The astrologer states what he is attempting to divine, and the GM assigns a number of \tcdieroll{d6} to roll against the skill rank for him to convince onlookers that what he says will come to pass.}	
\skillentry{Astronomy}{INT}{Auger}{115}{N/A}{Looking at the skies and cataloging the movements of the stars and planets is the province of the astronomer. This skill is needed to understand the movements of the heavens, and is often learned by sages, navigators, and nomad fortune readers.}	
\skillentry{Balance}{AGI}{Marine}{50}{2}{This ability is used to walk thin ledges, ropes, narrow walkways, etc. without falling. The GM may also ask for a check against balance when an adventurer tries many physical maneuvers that would test the balance of a normal person.}	
\skillentry{Barber}{AGI}{Auger}{15}{2}{Barbers cut hair and perform other personal grooming services. Wealthy citizens frequently train their own barber, or hire one skilled in these duties. Many free-lance barbers work out of the Natatorium.}	
\skillentry{Barristry}{INT}{Auger}{115}{RESERVED}{The barristry skill is useful if legal representation is needed, or if proper legal documents must be executed and filed. Such services can be quite costly, and many barristers grow rich from the legal woes of others.}	
\skillentry{Bartending}{CSE}{Auger}{30}{2}{The art of mixing palatable combinations of liquers and listening endlessly to repetitive conversation is important to those who must tend bars at various inns and taverns over the whole of Jaern.}	
\skillentry{Belching}{HEA}{Marine}{100}{2}{This skill has been brought to a fine art by bored marines who have been eating the onboard cooking too long. A high rank in this skill allows the talented belcher to sound off entire songs.}	
\skillentry{Binding}{CSE}{Auger}{50}{3}{When binding a target, choose how many more dice to add to the check for escape. Roll that many \tcdieroll{d6}; if you get your adventurer's skill rank or less, the bound person must add that many dice to a \tcdieroll{2d6} check against the escape skill to break free.}	
\skillentry{Blacksmithing}{STR}{Auger}{65}{3}{A blacksmith is skilled in forming utilitarian items from iron, such as nails, horseshoes, chains, etc. He is also experienced at attaching iron fittings to leather, and thus can fabricate and adjust horse tack and dolphin harnesses.}	
\skillentry{Bludgeon}{AGI}{Auger}{165}{N/A}{Bludgeon is the ability to knock an unsuspecting target unconscious by striking him from behind with a blunt instrument. The bludgeoner must position himself behind the target without the target having heard, seen or being aware in any way of the attacker. The GM will adjudicate the difficulty of the bludgeon based on factors such as alertness of the target, prevailing light conditions, and other distractions. If successful, roll on the Bludgeon Critical Hit Table in Appendix D. If the skill check fails, roll "to strike" against the target. If this roll succeeds, the target takes damage as normal (resolving all proper criticals and modifiers). If the roll "to strike" fails, the target of the bludgeon gets one free round to act before combat continues.}	
\skillentry{Boarding}{AGI}{Marine}{100}{2}{This is the ability to move safely and quickly from one ship to another, especially to a hostile vessel. This may involve lots of rope swinging, careful jumping, and rigging running, which are all encompassed in this skill.}	
\skillentry{Botany}{INT}{Auger}{30}{N/A}{Botany is necessary to understand and implement the requirements for healthy plant growth. A botanist can advise on proper fertilization, watering, pruning, etc., as well as diagnose and perhaps cure plant diseases.}	
\skillentry{Brewing}{INT}{Auger}{80}{RESERVED}{This skill has been carefully handed down, father to son, since antiquity. The secrets of using just the right materials to assure the proper fermentation of the grains have been closely held by those in the brewers guild. This profession is highly profitable to one who can find a way to gain admittance to its ranks.}	
\skillentry{Bricklaying}{INT}{Auger}{50}{2}{Mixing, layiing and mortering bricks is vital to the construction of new buildings and public works. The bricklayer is in demaind on all such construction projects, both making the bricks and laying them out carefully allong the wall lines and areas specified by the architech.}	
\skillentry{Build Trap}{INT}{Auger}{250}{N/A}{Anyone wanting to build and arm mechanical traps should have this skill. Magical effects can be bound to such traps using the rules for creating magical items.}	
\skillentry{Butchering}{CSE}{Auger}{30}{2}{This skill is necessary for the efficient cutting of an animal carcass into usable meat. A butcher (i.e., one who uses this skill as a profession) can identify spoiled meat before others can, and is an expert at putting a razor sharp edge on a knife.}	
\skillentry{Camouflage}{CSE}{Auger}{50}{2}{This skill is the deliberate and specific concealment of one or more persons in the wild. A substantial amount of time may be required to gather and prepare all the needed materials. As a guide, allow \tcdieroll{3d6}+3 turns per person to be concealed.}	
\skillentry{Candlemaking}{INT}{Auger}{15}{2}{Basic candlemaking involves the repetitive dipping of a fabric wick in tallow to create a longlasting light source.}	
\skillentry{Carpentry}{INT}{Auger}{50}{2}{This skill is crucial to the construction of wooden objects, buildings, or vehicles. The proper use of tools, materials and knowledge of their joinings allow the hard working carpenter to make a reasonable living from his craft. Attempts to create or repair wooden items can be ajudicated by checks against this skill.}	
\skillentry{Cartography}{INT}{Marine}{100}{3}{Cartography is the making, care, reading, copying, and interpretation of maps. A successful \tcdieroll{1d6} skill check is required to read the basics of a map well enough to follow established paths and roadways. Without this skill, a \tcdieroll{4d6} check versus INT is required. To travel off the beaten path with the aid of a map, a player needs to make a successful \tcdieroll{2d6} skill check. (As well as a successful Orienteering check to keep from getting lost.)}	
\skillentry{Cartwrighting}{INT}{Auger}{50}{3}{A cartwright posses special carpentry skills to assemble and repair carts and wagons. He cannot make wooden wheels, however.}	
\skillentry{Climbing}{STR}{Marine}{100}{2}{This skill is used for climbing ropes, scaling rough walls, etc. The GM will set the difficulty of the check based on the circumstances of the climb.}	
\skillentry{Cobbling}{INT}{Auger}{50}{2}{Cobbling involves the construction and repair of leather footwear. The amount of time required and the difficulty of the check depends on the type of footwear. A pair of rope-soled sandals take about one day, whereas a pair of thigh length, jacer hide boots with secret compartments in the heels and soles could take two weeks or more.}	
\skillentry{Composing Music}{CSE}{Nomad}{250}{0}{Creating melody, harmony and rhythm from the chaos of life all around, the composer puts quill to paper to creae music that he and others can perform.}	
\skillentry{Cooking}{INT}{Auger}{15}{2}{This skill allows the preparation of edible and attractive foods and drinks. A check of \tcdieroll{2d6} is reasonable to prepare a plain but tasty meal. A check of \tcdieroll{4d6} is appropriate to prepare a successful feast for a large crowd, or to make a very exotic dish.}	
\skillentry{Coopering}{INT}{Auger}{65}{2}{A cooper fashions watertight wooden containers, such as barrels, kegs, buckets, etc. The skill includes the ability to select the proper wood, make beveled joints, and use metal bands to clamp and reinforce the item.}	
\skillentry{Courtesan}{COM}{Auger}{115}{2}{This skill is used to please other people in various physical and sexual ways. Skill as a evening companion and well as the well directed use of lust are included within. A check of \tcdieroll{2d6} is considered when attempting to please a companion. A check of \tcdieroll{3d6} vs this skill is usual for attempting to entice someone into a situation, but remember the circumstances can cause the GM to modify this check.}	
\skillentry{Cyphering}{INT}{Auger}{115}{N/A}{Cyphering is encoded writing. It is frequently used in business correspondence, communication with the Assassins Guild, and between maraujo captains. Cyphering can be used to create or break codes. To create a code, the player selects a number of \tcdieroll{d6} to roll against the skill rank. If he gets his adventurer's skill rank or less the code is useable, and the number of \tcdieroll{d6} rolled becomes the code's difficulty rating. If the check is failed, the code is flawed and will yield gibberish or misleading statements if used. To break a code, the player rolls a number of \tcdieroll{d6} equal to his adventurer's cyphering rank. The GM rolls a number of \tcdieroll{d6} equal to the code's difficulty. The higher total wins, i.e., if the player is higher he breaks the code, and if the GM is higher, the code remains insoluble. If the player knows the key word or phrase used to construct the code, the player rolls a number of \tcdieroll{d6} equal to one less than the code's difficulty (if the code is difficulty 6, the player rolls \tcdieroll{5d6} and the GM \tcdieroll{6d6}) regardless of his adventurer's cyphering rank. The role of player and GM can of course be reversed in the above examples if an actor is trying to break a player-created code.}	
\skillentry{Dagger Fighting}{CSE}{Marine}{120}{N/A}{This weapon skill allows greater proficiency in hand-to-hand combat with a dagger. Like other weapon skills, options for Quickdraw, MultiAttack, Precise Attack, Lethal Attack, and Effective Attack can be used as described for Combat Weapon Skills.}	
\skillentry{Dagger Throwing}{CSE}{Marine}{60}{N/A}{This proficiency skill assists in targeting thrown daggers. Roll the number of dice equal to the bonus desired. If the total is equal to or less than your adventurer's skill rank, add that bonus to the "to hit" roll, but not the damage roll.}	
\skillentry{Dancing}{AGI}{Nomad}{100}{1}{The dancing skill is used to execute pleasing footwork patterns and body motions, usually to musical accompaniment. Success at this skill indicates the dancer has enjoyed the activity and has appeared to be competent at the appropriate level of difficulty depending on the assigned skill check (GM discretion, considering the intricacy of the dance).}	
\skillentry{Detect Traps}{PER}{Auger}{150}{4}{This skill allows an adventurer to observe a suspicious area and determine if, and how, it is trapped. The area must be in the adventurer's LOS, and the difficulty of the check depends not only on how cunningly the trap design is but also visibility.}	
\skillentry{Diagnosis}{INT}{Auger}{80}{RESERVED}{Someone adept at diagnosis can determine what is physically wrong with a person, though a remedy or treatment suggestion is outside the scope of this skill. Diagnosis can be developed into a very lucrative profession when used in conjunction with the nomadic herbology skill.}	
\skillentry{Disarm Trap}{INT}{Auger}{250}{N/A}{Having identified a trap by some means, this skill allows one attempt to disarm it. Average mechanical traps require a \tcdieroll{2d6} check to successfully disarm. Magic, technology, and trap difficulty can all change this check at the GM's discretion. Failure to disarm may trigger the trap (GM discretion).}	
\skillentry{Disguise}{INT}{Auger}{50}{3}{This is the ability to skillfully apply makeup, false beards, etc., and select and wear clothing to change one's outward appearance. The GM will determine the difficulty of the check based on how much the desired result varies from the physical characteristics of the adventurer. Simply adding a beard is very easy, but to disguise a human as another race can be exceedingly difficult.}	
\skillentry{Diving}{STR}{Marine}{50}{2}{Diving allows an adventurer to properly dive into water from greater heights, or into shallow water, with less chance of injury than someone without it. As a base, an unskilled person cannot dive into less than 10 feet of water safely, and will likely be injured if diving from a height of more than 10 feet. The GM determines the difficulty of the skill check. A dive from 30 feet into 10 feet of water would be a \tcdieroll{2d6} check against the skill; dives from greater heights or into shallower water increases the difficulty. The depth of the dive may also be adjusted with this skill. Normal dive depth is 10 feet, regardless of height. If the adventurer wishes he may have the depth of the dive equal the height of the dive to a maximum of 30 feet; minimum depth is two feet. Such depth changes add \tcdieroll{1d6} to the skill check.}	
\skillentry{Dodging}{AGI}{Marine}{200}{4}{This skill allows your adventurer to dodge incoming missiles. This is done in the opponent's round when the GM is about to roll "to strike" your adventurer. When the GM asks for his MDV, announce that he is dodging, and roll the number of dice equal to the number you wish to increase his MDV. If successful, state the MDV plus the number of dice rolled. Otherwise, state the MDV minus the number of dice rolled.}	
\skillentry{Dolphin Speech}{INT}{Marine}{300}{N/A}{Some believe dolphins are as intelligent as humans, but most think of them more like children. Dolphins communicate among themselves with a series of clicks, whistles, and grunts. Over time people can learn to understand and even "speak" some of the simpler "words." Trying to convey a simple idea, or understand one spoken by a dolphin, requires a successful \tcdieroll{2d6} check against this skill. Your GM will modify this roll depending on the complexity of the communication, and the current circumstances.}	
\skillentry{Dolphin Training}{CSE}{Marine}{400}{RESERVED}{Dolphins can become very intelligent and loyal mounts if skillfully trained. Dolphin trainers are highly prized and sought by Maraujo cefos to train their cavalries, and can demand and get high fees for their services.}	
\skillentry{Dolphinship}{AGI}{Marine}{200}{3}{This skill allows an adventurer to control and ride a properly trained and harnessed dolphin. Riding a dolphin at half the creature's speed through a calm sea is a \tcdieroll{2d6} check. Faster speeds, rough seas, or high speed maneuvers increase the difficulty of the check.}	
\skillentry{Drum Speak}{INT}{Nomad}{150}{N/A}{Small, specially formed drums are crafted by nomads to project sound up to three mets in fair weather (humid conditions can increase the distance carried to five mets, but heavy precipitation can cut the distance to a half a met.) This skill is used to create and translate messages and inflections via drum noises. The GM will assign a skill check based on the complexity of the message. Loud noises at the source or destination of the sound obliterate the message and cannot be overcome with this skill.}	
\skillentry{Dyeing}{INT}{Auger}{50}{2}{Dyeing is the infusion of color or colors into cloth. A dyer will know where to obtain specific dyestuffs, and can also aid in bleaching cloth.}	
\skillentry{Embalming}{CSE}{Priest}{200}{0}{Used by Priests of Anubis}	
\skillentry{Empathize}{CSE}{Auger}{20}{1}{This skill allows it possesor to carefulle listen to the tales and woes of others, seemingly concerned and sympathetic to their problems. Listening to the inflection and voice of his target, the empethiser both learns new things about them and eases their troubles and mood.}	
\skillentry{Escape}{INT}{Auger}{400}{4}{This skill is used to escape after being tied up. The player rolls \tcdieroll{2d6} against his adventurer's skill rank to work free from an average set of ropes. This roll may be modified by rope type, chains, locks and the adventurer's physical condition (GM discretion). If the check fails, more escapes may be tried, but subsequent checks are made with one additional \tcdieroll{d6} per failure.}	
\skillentry{Falconry}{WIL}{Nomad}{350}{N/A}{Training and controlling small birds of prey are delicate and difficult tasks. This skill measures the ability to train such birds over a time period, and the ability to instruct a trained bird to perform a task. A \tcdieroll{2d6} check against the skill is usual; the GM will adjust this depending on the bird's tractability, the difficulty of the task, etc.}	
\skillentry{Farming}{CSE}{Auger}{30}{2}{Farmers supply about one third the food used by Jaernian towns and cities, so this skill can be useful as a profession. Farming encompasses knowledge about planting, cultivation, and harvesting of crop plants. An adventurer with this skill might use the condition of crops as a clue to soil, weather, or unnatural conditions in a given area.}	
\skillentry{Fencing/Merchant}{CSE}{Auger}{80}{4}{This skill is necessary to avoid detection while buying or re-selling stolen goods. A \tcdieroll{2d6} skill check is normal, but the GM will adjust this based on such factors as uniqueness of the item, its recognizability, T'orite activity in the vicinity, T'orite suspicion of the fencer, etc.}	
\skillentry{Fencing}{AGI}{Marine}{350}{N/A}{This style of ritualistic combat uses small, light, flexible swords called foils or rapiers. These weapons are of virtually no use against armored opponents, or opponents with other weapons; attacks against such are at a -4 "to strike," and the fencing weapon breaks on ANY critical hit. Marines, however, often fence to resolve differences between themselves. Combatants in a fencing match wear light clothing, and arm themselves with an appropriate fencing weapon. The fencing skill rank is used as a modifier in all rolls "to strike." Foils and rapiers do \tcdieroll{1d6} DP per hit.}	
\skillentry{Fishing}{CSE}{Auger}{50}{2}{A hobby for many, fishing supplies at least half of the foodstuffs for most Jaernian cities and towns and therefore the skill can be used as a profession. This skill includes knowledge of small boats, nets, bait, location and movements of fish, and the storage and transport of caught fish.}	
\skillentry{Flagging}{INT}{Marine}{100}{N/A}{Ship-to-ship and ship-to-shore communications are often accomplished with flags. A flagger holds a flag in each hand, and moves them in patterns to signify words or concepts. Red flags are used on clear days, and white flags on overcast days for best visibility. Concepts are often abbreviated to make flagging concise and quick, but are limited in vocabulary. If a message cannot be expressed ordinarily, Paroli alphabet characters can be flagged one by one. Succeeding a \tcdieroll{2d6} check against this skill conveys most ordinary messages within the standard flagging vocabulary (GM discretion). Simple concepts are flagged twice as fast as ordinary speech, while spelling words is four times slower than speech.}	
\skillentry{Fletching}{INT}{Auger}{50}{2}{Technically speaking, fletching allows an adventurer to finish arrows by adding flight control feathers to the shaft. Fletchers work closely with arrow makers, and it is not unusual to find one or the other with the ability to completely fashion arrows from scratch. A \tcdieroll{2d6} skill check is usual to successfully manufacture arrows (checked in lots of five or ten). Manufacture of unusual missiles (larger than normal, made from non-standard materials, etc.) increases the difficulty.}	
\skillentry{Flying}{AGI}{Marine}{400}{4}{This skill allows an actor or adventurer to control the orientation and movement of their body while utilizing the spell Flight}	
\skillentry{Forestry}{INT}{Auger}{30}{2}{Forestry involves the nurturing and management of trees, including the cutting of appropriate trees for lumber to make room for new growth. A basic knowledge of different tree species, their requirements, and uses are subsumed in this skill.}	
\skillentry{Forgery}{INT}{Auger}{250}{4}{A skilled forger can duplicate signatures, papers, paper currency, or documents. Attempts at forging are made at a number of dice against the skill rank. Forging a signature is perhaps the easiest (\tcdieroll{2d6} check), while documents might be \tcdieroll{3d6} and currency \tcdieroll{5d6} (GM's discretion).}	
\skillentry{Gambling}{CSE}{Auger}{50}{2}{The gambling skill allows an adventurer to have a better chance at beating the odds in games of chance. Gambling is simulated by the GM (the "house") and the adventurer's player rolling dice against each other, the higher total winning the wager. The GM usually rolls \tcdieroll{5d6}. An adventurer with no gambling skill rolls \tcdieroll{1d6}; each rank in the skill adds a \tcdieroll{d6}. Some games of chance are far more difficult (or highly rigged in favor of the house), so the GM may roll upwards of \tcdieroll{10d6}. Also, if the GM is portraying an actor with gambling skill, he would roll twice, once for the "house" and once for the actor. Under specific roleplaying circumstances, either the player or the GM may ask to roleplay the game of chance rather than relying on dice for the outcome.}	
\skillentry{Gardening}{INT}{Auger}{15}{2}{Gardening is similar to farming, though on a smaller scale and dealing with vegetables, herbs, shrubberies, flowers, and the like rather than field crops. The skill subsumes knowledge of garden plants, their growth requirements, and uses. Gardening can be used as trade, from selling produce to tending the private gardens of the wealthy.}	
\skillentry{Glassblowing}{INT}{Auger}{50}{N/A}{A glassblower creates glass containers by blowing air through a hollow pipe into a blob of molten glass, then twirling or rolling the glass until it hardens. The skill can be a lucrative profession.}	
\skillentry{Heraldry}{INT}{Auger}{50}{N/A}{Heraldry involves the recording and awarding of coats of arms for nobles. Heralds are often consulted to settle disputes over royal bloodlines and to decide who is entitled to display a coat of arms. Thus they are quite knowledgeable about noble ancestry, including the black sheep and closeted skeletons the nobility would just as soon forget. The skill is used to ferret out and verify information that will prove a given bloodline. The GM will set the difficulty of the check based on such factors as the availability and accessibility of written documents, living witnesses, etc.}	
\skillentry{Herbology}{INT}{Nomad}{250}{RESERVED}{Herbologists collect, classify, purify, and sell reagents derived from plants. The work is delicate and exacting; done improperly a potentially beneficial reagent can become a deadly poison. An herbologist can serve as a physician (of sorts) to cure minor ailments (dyspepsia, headache, diarrhea, etc.), though he cannot restore lost DP with his nostrums and extracts.}	
\skillentry{Herding}{CSE}{Auger}{30}{1}{The herding skill is used to control groups of domesticated animals, such as cattle, sheep, etc.}	
\skillentry{Hiding}{AGI}{Auger}{50}{3}{The adventurer can use available cover (walls, corners, rubbish, furniture) to avoid being seen, or to conceal an object. A \tcdieroll{2d6} check against the skill is usual, but the GM will modify this based on the size of the person or object to be hidden versus the type and amount of cover available. It is possible that the result will be something less than full concealment. A \tcdieroll{2d6} check against PER is normal to notice something. The GM may adjust PER rolls if the person or object is quarter concealed (+\tcdieroll{1d6}); half concealed (+\tcdieroll{2d6}), or fully concealed (+\tcdieroll{3d6}). This skill can be used under any lighting conditions.}	
\skillentry{Horse Training}{WIL}{Auger}{150}{N/A}{A horse trainer attempts to curb the wildness of a horse to make it comfortable around people, but breaking and training wild horses for riding and farming duties can be hazardous if you don't know what you're doing. This skill is in great demand around any large village or town. A \tcdieroll{2d6} check is normal, though the GM will adjust this based on the fractiousness of the beast in question. Failure might result in injury to the adventurer (GM discretion).}	
\skillentry{Horsemanship}{CSE}{Auger}{100}{2}{This is the ability to ride a horse, or to handle a team of horses. A standard check of \tcdieroll{1d6} applies to riding a horse at a trot, or driving a wagon pulled by two horses at a walk, for an hour. Your GM will set checks for any other actions your adventurer attempts on horseback.}	
\skillentry{Hunting}{PER}{Auger}{70}{2}{This ability allows its possessor to find, capture or kill small animals in the wild. A hunter can find animal spoor and trace it to their current location. For a hunter to find enough food for one day, he must roll one \tcdieroll{d6} for each person to feed against his rank in this skill.}	
\skillentry{Hypnosis}{WIL}{Nomad}{300}{N/A}{This is the ability to use some object or technique to place a willing target into a hypnotic trance. The hypnotist can cause the target to recall events clearly, perform any short, non-combat action, or implant subliminal suggestions about actions to be taken up to one week in the future. While the subject can be instructed not to remember questions or actions, he cannot be forced to do anything to which he would strongly object in his normal mental state. The GM must set the skill check based on the circumstances and the difficulty of the request. Simple actions might be a \tcdieroll{2d6} check, whereas implanting suggestions could be a \tcdieroll{3d6} or \tcdieroll{4d6} check.}	
\skillentry{Identify Minerals}{INT}{Auger}{15}{2}{Someone with this skill can look at a rock sample and identify any minerals or metal ores it contains (\tcdieroll{1d6} check). Determining quality and quantity raises the difficulty of the skill check (GM discretion).}	
\skillentry{Identify Plant}{INT}{Auger}{20}{2}{Identify plant is used to determine what a plant is (\tcdieroll{1d6} check for common plants; \tcdieroll{2d6} and higher for rarer flora). Whether the adventurer can recognize the use of a given plant is a \tcdieroll{3d6} check. A use check can be ignored if the adventurer has previous experience with the plant in question (GM discretion).}	
\skillentry{Identify Spell}{PER}{Mage}{200}{3}{This skill enables an adventurer or actor to identify certain parameters of any spell he sees cast. It in no way gives him any specific information about how that spell is cast or used. Roll \tcdieroll{1d6}; if the roll is equal to or less than the skill rank the spell type is discovered. Now roll another \tcdieroll{d6} and add it to the first roll. If the total is equal to or less than the skill rank, the spell group name is revealed. Now roll another \tcdieroll{d6} and add it to the total of the first two rolls; the spell rank can be discovered if the total of the three dice are equal to or less than the skill rank. The fourth \tcdieroll{d6} is rolled and added to the first three to reveal the number of finesses used, as long as the total of the four dice do not exceed the skill rank. The identification process ends whenever the dice total exceeds the skill rank. Dice for Skill Check 1 Identify type [elemental type or specific deity] 2 Spell group name 3 Rank of spell 4 Exact finesses in use}	
\skillentry{Immobilize}{STR}{Marine}{400}{N/A}{A quick blow to certain body areas can immobilize an opponent. The area (solar plexus, neck, etc.) must be unprotected, and certainly unarmored. Roll \tcdieroll{2d6}. If the total is equal to or less than the skill rank, the opponent collapses and cannot take any actions for \tcdieroll{3d6} rounds.}	
\skillentry{Innkeeping}{CSE}{Auger}{50}{2}{Innkeeping is necessary to the successful management of an inn, or in any situation requiring someone to provide food and lodging for a large group of people.}	
\skillentry{Instrumental Music}{CSE}{Nomad}{100}{N/A}{This skill allows its possessor to use one musical instrument; it must be rebought for each additional instrument. The difficulty of the music being played, and the audience it is played to, are considered by the GM when assigning dice for checks against this skill.}	
\skillentry{Instrumental Smithing}{INT}{Nomad}{200}{RESERVED}{This skill allows one to create musical instrumentals. Working with leather, metal, hide, and wood are all common to the instrument smith. The smith has knowledge of musical theory and the crafting of sounds from natural material. This skill can not be purchased at any rank higher then 3 above the possessor's highest instrumental music skill.}	
\skillentry{Jesting}{CSE}{Nomad}{100}{2}{Jesting is the ability to make other people laugh. It can involve slap-stick, sarcasm, abuse, or singing. This skill is complemented by the Juggling, Acrobatics, Singing, and the Instrumental Music skill. The GM may ask for ranks in these other skills to adjust the success of Jesting.}	
\skillentry{Jeweler}{INT}{Auger}{50}{N/A}{A jeweler is adept at fashioning adornments of precious metals and also the setting of gemstones in such jewelry.}	
\skillentry{Jousting}{STR}{Warrior}{300}{3}{Jousting is the formal, non-lethal combat between mounted opponents. Each jouster rolls a number of \tcdieroll{d6} equal to his jousting rank; the higher total wins the match.}	
\skillentry{Juggling}{AGI}{Nomad}{100}{2}{Throwing and tossing objects into the air and retrieving them is always an amusing skill. The distance of the toss, the number of objects, and the danger of what is being thrown are all considered by the GM when he sets a difficulty for a check against this skill.}	
\skillentry{Jumping}{STR}{Marine}{50}{2}{This skill is used for performing physical jumps of more than ordinary distance, height, or speed.}	
\skillentry{Knitting}{AGI}{Auger}{30}{N/A}{Knitting is the looping of thread or yarn with special needles to make garments. The more intricate or fine the work, the more difficult the skill check.}	
\skillentry{Lance}{CSE}{Warrior}{360}{N/A}{Because of the nature of this combat weapon, skill in its use precludes using the Added Attack and Quickdraw options. Lethal, Precise and Effective attacks are legal.}	
\skillentry{Landscaping}{INT}{Auger}{30}{2}{A landscaper can design and implement a formal garden or any pre-planned planting area.}	
\skillentry{Laundering}{CSE}{Auger}{15}{1}{This skill allows it possesor to clean clothing, furs, and hides. Items cleaned include clothing, linens, towels, rugs, and about any other item made of cloth. This skill is also used to clean specific stains and freshen specific cloth types.}	
\skillentry{Leather Working}{INT}{Auger}{80}{2}{This skill involves the sewing of clothing or items from pieces of leather.}	
\skillentry{Lip Reading}{PER}{Auger}{50}{RESERVED}{The adventurer must succeed a \tcdieroll{2d6} check vs this skill to interpret what is being spoken by another humanoid without having to hear. The lip reader must be fluent in the language being spoken to use this skill.}	
\skillentry{Listen}{PER}{Auger}{50}{2}{This reflects the extra training required to notice, and perhaps recognize, faint noises that would normally go unheard.}	
\skillentry{Locksmithing}{INT}{Auger}{80}{N/A}{This skill is used to craft locks and make or duplicate keys.}	
\skillentry{Marathon Running}{HEA}{Auger}{65}{2}{This skill allows an adventurer to run at a measured pace for a great length of time without fatigue. The GM asks for a \tcdieroll{1d6} check against the skill at the end of the first hour of running. At the end of the second hour the check is \tcdieroll{2d6}, etc. As soon as a check is failed, the runner must stop and rest one hour before continuing.}	
\skillentry{Masonry}{STR}{Auger}{50}{2}{A mason is skilled at building structures from cut stone and bricks. He is knowledgeable about the types of stone suitable for such work, and the proper mortar mix to bind them together.}	
\skillentry{Massage}{AGI}{Auger}{75}{2}{Skilled in the ease of muscle pain and stiffness, the masseuse aides their target in releaving the tightness and pain of the days work. Physical manipulation of tightned muscles, application of potent oils and liquids, and aromatic burning of helpful vapors are all part of the techniques used to relieve their target's pains.}	
\skillentry{Metal Smithing}{INT}{Auger}{150}{3}{Metal smithing is the ability to manipulate and build things out of silver, gold, copper, bronze, tin and lead. Fastenings, jewelery, nails, fixtures and parts for other craftsmens projects are some of the obvious things produced by the metal smith.}	
\skillentry{Military Construction}{CSE}{Auger}{80}{N/A}{This skill is necessary for the proper construction of siege engines (catapults, ballistas, etc.) and effective defensive positions.}	
\skillentry{Mimicry}{PER}{Nomad}{250}{4}{This skill is used to reproduce the sound of any human voice that its user has heard and memorized. Success is normally achieved with a \tcdieroll{2d6} check against this skill.}	
\skillentry{Mining}{STR}{Auger}{30}{2}{Someone with mining skill knows the proper procedure to dig a shaft into earth or stone and construct the necessary shoring to prevent collapse of the mine shaft.}	
\skillentry{Money Changing}{INT}{Auger}{65}{3}{Knowledge of foreign coinage, the ability to translate values, calculate interest and fees, and the ability to interact with other money changers all go into this skill. Being able to identify a foreign coin could be a \tcdieroll{2d6} check, while calculating compound interest on an overdue loan might be a \tcdieroll{3d6} check.}	
\skillentry{Mountain Climbing}{AGI}{Auger}{80}{3}{This is the skill to use to climb up and down the cliffs, hills, and mountains. Climbing alone, without equipment, up a 45 degree slope requires a \tcdieroll{2d6} check once per hour. Equipment, slope, and weather conditions can modify the difficulty and frequency of a check.}	
\skillentry{Moving Silently}{AGI}{Auger}{100}{4}{An adventurer with this skill has a better chance of approaching without being heard. The noiser the terrain underfoot, the more difficult the check.}	
\skillentry{Musical Composition}{INT}{Nomad}{250}{N/A}{Creating new music is a difficult skill. This skill should be combined with the instrumental music skill for a greater chance of success. The test of a new piece of music is how well it is received by its first audience. When a new piece is presented, a \tcdieroll{2d6} check against this skill is normal.}	
\skillentry{Navigation}{INT}{Marine}{150}{4}{Navigation involves being able to read sea charts, determine location by the position of Onra and the stars, understand the affects of wind and currents on plotting a course, etc.}	
\skillentry{Net Handling}{AGI}{Warrior}{100}{2}{Weilding a 6 foot long net with his non-weapon hand, the user swings and flings the net to defend himself nd entrap his prey. For each die he rolls against the skill, he gets to add a +2 to his attempt to grapple to net his opponent. Once netted, an apponent must roll \tcdieroll{4d6} vs agility to fling the net aside, or \tcdieroll{5d6} vs strength to tear the net apart. Each failed attampt adds one die to future attempts while still netted. While netted, all attacks on the target are at a +4 to succeed, or one die less on skill checks (like pummeling).}	
\skillentry{Non-verbal casting}{CSE}{Mage}{300}{N/A}{Spell casting normally requires the use of hand motions and words to focus and target the magical energies. Making a check of \tcdieroll{2d6} against this skill allows the caster to cast his spell without the use of his voice. A mage who has lost the use voice, or is gagged, would find this skill very useful.}	
\skillentry{Oar Mastery}{INT}{Marine}{200}{2}{This skill allows your adventurer to control and command banks of galley slaves. This includes the ability to correctly power the ship, knowledge of how to maintain the short and long term health of the rowers, and how to control and restrain the rowers. While not the most glamorous job on board, everyone knows a ship without a good oar master is useless in combat.}	
\skillentry{One hand casting}{AGI}{Mage}{150}{N/A}{Normally any spell with a casting time of over one melee requires the use of both hands. Making a check of \tcdieroll{2d6} allows the caster to cast his spell with one hand. A mage who has lost the use of one hand would find this skill very useful.}	
\skillentry{Opening Locks}{INT}{Auger}{65}{N/A}{An adventurer with this skill may be able to open a lock without the key.}	
\skillentry{Orienteering}{CSE}{Auger}{30}{2}{This skill is very useful to prevent becoming lost. An adventurer with orienteering can always find due north, and thus know which way to travel to his destination.}	
\skillentry{Painting}{INT}{Marine}{50}{2}{This skill is the ability to use painting tools and paint to coat large objects such as ship hulls and exterior or interior walls.}	
\skillentry{Pickpocketing}{AGI}{Auger}{80}{4}{Pickpocketing is necessary to remove objects from a person's clothing without being caught. The GM will determine how many dice to use based on the circumstances of the encounter, size and location of the item to be filched, etc.}	
\skillentry{Pimping}{CSE}{Auger}{80}{3}{A judge of good looking women and men, the pimp is considered a "lay Priest" of the Erection of Scrogg, and is generally tolerated, if not accepted in any town or city if they wish not to excite the wrath of Scrogg. This skill allows the pimp to judge the potential attraction of his current and future employees, and to train them to their task. A variety of tasks will be assigned difficulties by the GM and an appropriate number of \tcdieroll{d6} can then be rolled against this skill.}	
\skillentry{Poetry}{CSE}{Auger}{65}{3}{A poet is able to craft words into rhymes capable of evoking any mood, or perhaps a scathing political commentary. The poet must state what he is writing about, and what force he wants his poetry to have, so the GM can determine the difficulty of the check.}	
\skillentry{Pottery}{CSE}{Auger}{15}{2}{The pottery skill allows the creation of pots or other containers from molded clay hardened in a kiln.}	
\skillentry{Pummeling}{STR}{Marine}{100}{2}{This skill is used to repeatedly punch a standing opponent in melee. Making a \tcdieroll{2d6} check vs this skill inflicts \tcdieroll{1d4} damage points on your opponent and knocks him to the ground.}	
\skillentry{Puppeteering}{INT}{Nomad}{150}{2}{Creating and using small hand puppets to stage plays to entertain both children and adults is a common skill among many nomads. These plays are often used to teach morals to young people. Nomads often ask for donations after a play is complete.}	
\skillentry{Pyrotechnics}{INT}{Nomad}{100}{N/A}{The handling of flammable powders and devices to produce sparks, flames, sounds, and smoke is a delicate and dangerous skill. The pyrotechnist explains what he wishes to do, and the GM determines the materials cost and assigns a skill. These powders cannot cause great explosions, and are very hard to trigger precisely.}	
\skillentry{Repair}{CSE}{Marine}{250}{N/A}{This skill enables an actor or adventurer to fix such things as mechanical linkages, complex rigging, water clocks, devices with pulleys, ropes and wheels, or items based on a similar technology.}	
\skillentry{Rigging Running}{AGI}{Marine}{100}{2}{This skill allows an actor or adventurer to move quickly through a ship's rigging by jumping, climbing, and sliding. It also encompasses adjusting knots, rope tensions and sail positions to properly trim a ship.}	
\skillentry{Rope Making}{INT}{Marine}{50}{2}{This skill allows an adventurer to make proper rope from any suitable material (e.g., plant fibers, hair, yarn, etc.).}	
\skillentry{Rowing}{STR}{Marine}{100}{2}{This skill is required to properly row a boat with two oars. It might also be applied to rowing in unison with others.}	
\skillentry{Saddlemaking}{INT}{Auger}{30}{2}{Saddlemaking is the skill needed to meld wood, leather, and metal fittings into a seat comfortable to both man and mount.}	
\skillentry{Sail Falling}{AGI}{Marine}{150}{2}{This skill allows your adventurer to safely fall 100 feet or less to the deck in one round. The adventurer jumps in the direction of the nearest sail with knife in hand. Thrusting the knife into the material of the sail, he hangs from it and executes a controlled fall as the knife slices the rough canvas. The check is \tcdieroll{1d6} for each 20 feet of height. If the check fails, your adventurer suffers \tcdieroll{1d6} damage points per 20 feet fallen and loses one round of action. For that round he lies flat on the deck. He may get up the next round, which is his action for the round.}	
\skillentry{Sail Making}{INT}{Marine}{50}{N/A}{A sailmaker has the skill to design sails to the proper size for a vessel, then transfer the patterns to canvas, cut the panels, and assemble them. This is a very lucrative profession on a planet where the main means of transportation is by ship.}	
\skillentry{Sailing}{CSE}{Marine}{50}{2}{Sailing involves holding a course with the rudder and trimming the sails to catch the prevailing wind, to in turn drive a ship at an optimum speed. This skill is useful for long journeys, passage through rough waters or storms, or handling damaged ships.}	
\skillentry{Scribing}{INT}{Priest}{200}{N/A}{This ability is used to copy manuscripts, take dictation, and record happenings. It involves much more than simply writing down the appropriate words in the correct language. Proper fonting, illustrations, indexing, and cross-referencing are crucial to historical, professional, and technical scribing.}	
\skillentry{Sculpting}{CSE}{Auger}{65}{3}{This skill allows someone to chisel statuary or other objects from stone, or craft such items in clay or wax.}	
\skillentry{Seduction}{COM}{Auger}{100}{3}{Attracting other people for use as sexual toys has long been an art practiced by the followers of Scrogg. The proper clothing, the right walk, the correct affected accent and the appropriate scent are all parts of this skill. The ability to attract any specific person will be assigned a difficulty and dice roll by the GM.}	
\skillentry{Set Traps/Snares}{INT}{Auger}{250}{3}{This gives the ability to set a trap or snare to capture or injure something or someone. The GM will assign the difficulty based on such things as size, intricacy, how well hidden it's to be, damage it can do, etc.}	
\skillentry{Shadows}{AGI}{Auger}{50}{4}{This skill can be used to attempt to hide in moonlight, very poor lighting conditions, and underground. The actor or adventurer must be at least 20 feet away from those he is hiding from, and they must be unaware of his presence. A \tcdieroll{2d6} check will normally allow him to remain hidden. Any movement will likely reveal his presence, or the GM may require a \tcdieroll{3d6} or \tcdieroll{4d6} check to maintain the cover. This skill can not be used in daylight.}	
\skillentry{Ship Building}{INT}{Marine}{300}{RESERVED}{Directing the construction of ships, from the smallest dinghy to the largest merchant ship, takes a keen knowledge of specialized construction techniques, materials, labor management, accounting, and finance management. The building of seaworthy ships can only be learned from experienced shipwrights, and is a very profitable profession.}	
\skillentry{Singing}{COM}{Nomad}{50}{2}{Pleasing others with song can save an adventurer from the most difficult situations. The difficulty of the song and the difficulty of the audience are both considered when assigning a skill check.}	
\skillentry{Skating}{AGI}{Auger}{30}{2}{Skating gives an adventurer the ability to move swiftly over frozen water on ice skates. The movement rate is doubled if a \tcdieroll{1d6} check is made. Changes in direction while moving also require a \tcdieroll{1d6} check. Fancy maneuvers or attempts to go faster require more difficult checks.}	
\skillentry{Slave Handling}{CSE}{Auger}{35}{3}{Knowing how to evaluate slaves, how to buy and sell them, how to keep them healthy and strong, and how to manage and control them are all facets of this skill. Slave handlers are in great demand by the merchant class, by the rich, and by the large temples for managing their necessary staffs of slaves.}	
\skillentry{Sleight of Hand}{AGI}{Auger}{30}{4}{This is used to perform minor feats of "magic," usually prefaced by the phrase, "The hand is quicker than the eye . . ."}	
\skillentry{Smuggling}{CSE}{Auger}{200}{4}{This is the ability to bring goods or people into an area undetected, usually for illegal purposes.}	
\skillentry{Snorkeling}{STR}{Auger}{15}{2}{This skill allows an adventurer to swim while scanning the bottom, or dive to depths of 15 feet without need for extra air. Proper snorkeling equipment is required, of course.}	
\skillentry{Spelunking}{AGI}{Auger}{150}{3}{This is a climbing ability usually used in underground caverns. It is useful for climbing in any situation involving wet rock and darkness.}	
\skillentry{Sprinting}{STR}{Auger}{50}{2}{This is your adventurer's ability to run at a much faster pace for a short duration. Normally, you roll \tcdieroll{1d6} versus this skill for each 10 foot per round increase in movement rate he attempts. This roll is automatically modified by the same number of dice that his AGI is modified, according to his armor. Repeat the check each minute; if failed the adventurer can not attempt this skill again until after they have rested for ten minutes. Sprinting cannot be combined with Marathon Running.}	
\skillentry{Stalking}{CSE}{Auger}{150}{2}{Stalking is the ability to stealthily approach a place where something (or someone) may be hiding, and planning a way to kill or capture it. The GM determines the difficulty of the stalk and assigns a number of \tcdieroll{d6} for the player to roll. If successful, the stalker has reached his chosen position.}	
\skillentry{Stone Smithing}{INT}{Auger}{100}{3}{Stone smithing is the ability to manipulate and build things out of cut marble, basalt, slate and quartz. Floors, walls, supports, stairways and parts for other craftsmen's projects are some of the obvious things produced by the stone smith.}	
\skillentry{Surfing}{AGI}{Marine}{50}{2}{Riding the wave crests to shore while standing on wooden boards is a favorite marine tactic to land in force from ships anchored just off shore. More recently it has become a sport practiced by adolescents and young adults at beaches everywhere.}	
\skillentry{Swimming}{STR}{Marine}{20}{2}{Swimming forward in calm water normally requires a \tcdieroll{1d6} check against this skill to succeed. Water temperature, flow, roughness, armor, and carried equipment can affect the difficulty of this check. Water Breathing makes this check two dice easier.}	
\skillentry{Tackling}{AGI}{Marine}{120}{2}{An adventurer can knock his opponent to the ground, if he gets a running start. The tackler must make a \tcdieroll{2d6} check vs this skill; if successful he and his opponent are knocked down and the tackler gets an immediate free round. After the free round, initiative is determined and combat proceeds normally. The GM may modify the number of dice for different sized opponents.}	
\skillentry{Tailoring}{INT}{Auger}{50}{2}{Tailoring involves the sewing of fabric to make clothing, or items such as bags, from cloth.}	
\skillentry{Tanning}{INT}{Auger}{30}{2}{This skill is needed to turn raw animal hides into leather. The better the tanning, the more supple and better quality the leather will be.}	
\skillentry{Target Magic}{AGI}{Mage}{200}{N/A}{Target Magic allows the caster of elemental or divine spell to maneuver for line of sight and finish casting in one round. The spell must be a one round spell, or be in its last round of casting, and if the player succeeds at a two dice check against this skill, his adventurer jostles about and he gets an additional roll to determine line of sight. The roll will be at the same odds as a requested line of sight roll during the informational questions portion of the round. The player does not appreciable change position, but is just jostling to obtain a shot. If the player fails the roll, he aborts the casting of the spell, not consuming the appropriate units.}	
\skillentry{Tattooing}{PER}{Nomad}{200}{N/A}{A tattoo artist uses metal needles and colored inks to create designs, pictures and words on the skin of his subjects. Tattoos are used often by nomads, Priests of T’or and by Akravojo Warriors and the talented tattooist is in high demand. Also tattooing is considered an art form, and its best practitioners are revered and may demand any price for their work.}	
\skillentry{Taxidermy}{INT}{Auger}{65}{N/A}{Taxidermy is the preservation of deceased creatures by removing organs and chemically preserving the body.}	
\skillentry{Teaching}{INT}{Priest}{100}{N/A}{Normally one can teach a skill to someone else at any rank up to four ranks less then his own rank in that ability. This teaching skill allows its possessor to teach the next four ranks up to his own rank in the target skill. To use this skill, the teaching actor must spend the time attempting to teach the target skill to his target. At the end of this time, he must make a check of one dice for each rank above four below his rank against the rank of this skill. If he succeeds, the target gains the additional rank. If he fails, he must go through the teaching time from the start to attempt again. The last rank he can teach is his own rank, and this requires a \tcdieroll{4d6} check against the rank of his teaching skill. The teacher can never teach above his own rank in the target skill.}	
\skillentry{Tent Making}{INT}{Auger}{80}{2}{Tent making is the fabrication of portable shelters from animal hides or heavy fabric.}	
\skillentry{Torture}{CSE}{Auger}{65}{4}{Causing pain is a fine skill to reduce the strongest man to a state of submission. Talented torturers can cause captives to divulge knowledge or confess crimes, even those not actually committed. Truth is valuable to many, and one with this skill can always find gainful employment.}	
\skillentry{Toy Making}{INT}{Auger}{65}{2}{Toy making is primarily the working of wood (though other materials may be used) into shapes to amuse children. Toys capable of complex movements require more difficult skill checks.}	
\skillentry{Tracking}{PER}{Auger}{150}{2}{Following the spoor of animals and the tracks of man is a useful skill in the wild. Fresh tracks can usually be followed by making a \tcdieroll{2d6} check. Time, rain, and conscious efforts to mask a trail can make these checks more difficult.}	
\skillentry{Trapping}{CSE}{Auger}{50}{2}{Trapping is the setting of snares or metal spring traps to capture small animals, generally for their fur. It can be a lucrative profession.}	
\skillentry{Tumbling}{AGI}{Marine}{100}{2}{This skill allows your adventurer to reduce the amount of damage taken in a fall. Subtract his rank in this skill from any damage taken from a fall.}	
\skillentry{Ventriloquism}{CSE}{Nomad}{200}{N/A}{Throwing your voice to appear to come from another place is a strange skill which is often combined with Puppeteering to give puppets an apparent voice.}	
\skillentry{Verbal Casting}{CSE}{Priest}{300}{N/A}{Normally spell casting requires hand motions to focus and target the magical energies. Making a check of \tcdieroll{2d6} vs this skill allows the caster to cast his spell with just his voice. A mage who has lost the use of his hands, or is bound, would find this skill very useful.}	
\skillentry{Veterinary}{CSE}{Auger}{150}{RESERVED}{A veterinarian is skilled in the care of animals and the diagnoses and treatment of animal diseases.}	
\skillentry{Water Skiing}{AGI}{Auger}{50}{2}{This ability is needed to travel behind Jaernian hydro-sails or dolphins, on water, skis.}	
\skillentry{Weapon Smithing}{INT}{Auger}{50}{2}{Weapon smithing is the ability to craft any weapon from metal and wood. The GM must set the difficulty, depending on how complex or difficult the weapon would be to make.}	
\skillentry{Weaving}{INT}{Auger}{30}{3}{Weaving involves the tedious process of interlocking numerous strands of yarn together on a loom to make cloth, rugs, wall hangings, etc. The more intricate the design, or the tighter the weave, the more difficult the check.}	
\skillentry{Wheelwright}{CSE}{Auger}{50}{2}{A wheelwright is expert in the crafting of wheels for carts, carriages, or wagons. These can be simple wooden disks (\tcdieroll{1d6} check) or carriage wheels of fancy design (\tcdieroll{3d6} or more). Wheelwrights also know how to apply iron rims to wheels to prolong a wheel's life.}	
\skillentry{Wine Making}{INT}{Priest}{250}{N/A}{The production of wine has always been the province of the Priesthood. Some of the best wines come from the Priesthood of Isis, because of their knowledge of living things, and its uses in deadening the senses of their patients. Others claim the best wines come from the cellars of the Solarium, where secret fermentation techniques involving the sun and much glassware give the wines a sweeter flavor.}	
\skillentry{Wrestling}{CSE}{Marine}{180}{N/A}{This skill combines several different styles of unarmed hand to hand combat which can aid an adventurer when grappling an opponent. Any one of these options may be used in a single combat round. Clobbering is using your fists and feet to knock an opponent to the ground, increasing an adventurer's chance to succeed at "grapple to hold" and opponent. When the player has decided to "grapple to hold", he may at the same time roll \tcdieroll{1d6} for each bonus of 2 that he wants to add to his grapple roll. If he succeeds the roll, he adds the sought number to his \tcdieroll{d20} roll to grapple, and uses the sum to determine the success of the "grapple to hold". When an adventurer attempts to "grapple to throw" an opponent, he may use this skill to Heave the opponent farther than would be originally adjudicated by the game master. The player rolls \tcdieroll{1d6} for each additional 5' of distance he wishes his adventurer to heave his opponent. The GM still adjudicates the resulting distance and damage (probably adding damage from a fall from a successful heave). Pummel can only be used after an opponent has been successfully "grappled to hold" in a previous round, and while that opponent is still held. The player decides how many \tcdieroll{d6} of damage he wishes his adventurer to do to his opponent by kneeing, punching, slapping and jabbing him. Then the player rolls twice that number of \tcdieroll{d6}. If he matches or gets less than the rank in this skill, he does the desired number of \tcdieroll{d6} of damage to his opponent. Like a pummel, a Cosh can only be performed after the opponent is still being held. The adventurer attempts to knock the opponent unconscious by hitting him in the head, or hitting his head against the ground. The Player rolls \tcdieroll{4d6} vs this skill, and if successful, the opponent is dazed or unconscious for \tcdieroll{1d6} rounds (rolled by the GM). A skillful wrestler knows a number of holds which make it more difficult for his opponent to break free. To Pin his opponent, the player rolls one \tcdieroll{d6} for each dice he wants to increase his opponent's attempts to break free. If he succeeds, the opponent attempts to break the hold are that many dice more difficult until the next time the adventure has a round. The player needs to recheck this hold every round for it to stay in effect. When an adventurer has been held by being the target of a successful "grapple to hold", and then attempts to break free, he can Struggle to make the attempt easier. For each die he wishes to reduce the check by, he must roll a \tcdieroll{d6}. If the check succeeds, he can lower how many dice he rolls to break free by the same number of dice.}	
\skillentry{Writing}{INT}{Auger}{15}{RESERVED}{This is creative writing, not writing a grocery list. The writer must state what he is writing about, and what force he wants his writing to have so the GM can determine the difficulty of the check.}	
\skillentry{Zoology}{INT}{Auger}{50}{3}{This skill is used to identify and care for animals. It encompasses such things as knowledge of a creature's life cycle, breeding habits, food preferences, etc.}	
\end{multicols*}